% Options for packages loaded elsewhere
\PassOptionsToPackage{unicode}{hyperref}
\PassOptionsToPackage{hyphens}{url}
%
\documentclass[
]{article}
\usepackage{lmodern}
\usepackage{amssymb,amsmath}
\usepackage{ifxetex,ifluatex}
\ifnum 0\ifxetex 1\fi\ifluatex 1\fi=0 % if pdftex
  \usepackage[T1]{fontenc}
  \usepackage[utf8]{inputenc}
  \usepackage{textcomp} % provide euro and other symbols
\else % if luatex or xetex
  \usepackage{unicode-math}
  \defaultfontfeatures{Scale=MatchLowercase}
  \defaultfontfeatures[\rmfamily]{Ligatures=TeX,Scale=1}
\fi
% Use upquote if available, for straight quotes in verbatim environments
\IfFileExists{upquote.sty}{\usepackage{upquote}}{}
\IfFileExists{microtype.sty}{% use microtype if available
  \usepackage[]{microtype}
  \UseMicrotypeSet[protrusion]{basicmath} % disable protrusion for tt fonts
}{}
\makeatletter
\@ifundefined{KOMAClassName}{% if non-KOMA class
  \IfFileExists{parskip.sty}{%
    \usepackage{parskip}
  }{% else
    \setlength{\parindent}{0pt}
    \setlength{\parskip}{6pt plus 2pt minus 1pt}}
}{% if KOMA class
  \KOMAoptions{parskip=half}}
\makeatother
\usepackage{xcolor}
\IfFileExists{xurl.sty}{\usepackage{xurl}}{} % add URL line breaks if available
\IfFileExists{bookmark.sty}{\usepackage{bookmark}}{\usepackage{hyperref}}
\hypersetup{
  pdftitle={Exam3},
  pdfauthor={Ronald Long},
  hidelinks,
  pdfcreator={LaTeX via pandoc}}
\urlstyle{same} % disable monospaced font for URLs
\usepackage[margin=1in]{geometry}
\usepackage{color}
\usepackage{fancyvrb}
\newcommand{\VerbBar}{|}
\newcommand{\VERB}{\Verb[commandchars=\\\{\}]}
\DefineVerbatimEnvironment{Highlighting}{Verbatim}{commandchars=\\\{\}}
% Add ',fontsize=\small' for more characters per line
\usepackage{framed}
\definecolor{shadecolor}{RGB}{248,248,248}
\newenvironment{Shaded}{\begin{snugshade}}{\end{snugshade}}
\newcommand{\AlertTok}[1]{\textcolor[rgb]{0.94,0.16,0.16}{#1}}
\newcommand{\AnnotationTok}[1]{\textcolor[rgb]{0.56,0.35,0.01}{\textbf{\textit{#1}}}}
\newcommand{\AttributeTok}[1]{\textcolor[rgb]{0.77,0.63,0.00}{#1}}
\newcommand{\BaseNTok}[1]{\textcolor[rgb]{0.00,0.00,0.81}{#1}}
\newcommand{\BuiltInTok}[1]{#1}
\newcommand{\CharTok}[1]{\textcolor[rgb]{0.31,0.60,0.02}{#1}}
\newcommand{\CommentTok}[1]{\textcolor[rgb]{0.56,0.35,0.01}{\textit{#1}}}
\newcommand{\CommentVarTok}[1]{\textcolor[rgb]{0.56,0.35,0.01}{\textbf{\textit{#1}}}}
\newcommand{\ConstantTok}[1]{\textcolor[rgb]{0.00,0.00,0.00}{#1}}
\newcommand{\ControlFlowTok}[1]{\textcolor[rgb]{0.13,0.29,0.53}{\textbf{#1}}}
\newcommand{\DataTypeTok}[1]{\textcolor[rgb]{0.13,0.29,0.53}{#1}}
\newcommand{\DecValTok}[1]{\textcolor[rgb]{0.00,0.00,0.81}{#1}}
\newcommand{\DocumentationTok}[1]{\textcolor[rgb]{0.56,0.35,0.01}{\textbf{\textit{#1}}}}
\newcommand{\ErrorTok}[1]{\textcolor[rgb]{0.64,0.00,0.00}{\textbf{#1}}}
\newcommand{\ExtensionTok}[1]{#1}
\newcommand{\FloatTok}[1]{\textcolor[rgb]{0.00,0.00,0.81}{#1}}
\newcommand{\FunctionTok}[1]{\textcolor[rgb]{0.00,0.00,0.00}{#1}}
\newcommand{\ImportTok}[1]{#1}
\newcommand{\InformationTok}[1]{\textcolor[rgb]{0.56,0.35,0.01}{\textbf{\textit{#1}}}}
\newcommand{\KeywordTok}[1]{\textcolor[rgb]{0.13,0.29,0.53}{\textbf{#1}}}
\newcommand{\NormalTok}[1]{#1}
\newcommand{\OperatorTok}[1]{\textcolor[rgb]{0.81,0.36,0.00}{\textbf{#1}}}
\newcommand{\OtherTok}[1]{\textcolor[rgb]{0.56,0.35,0.01}{#1}}
\newcommand{\PreprocessorTok}[1]{\textcolor[rgb]{0.56,0.35,0.01}{\textit{#1}}}
\newcommand{\RegionMarkerTok}[1]{#1}
\newcommand{\SpecialCharTok}[1]{\textcolor[rgb]{0.00,0.00,0.00}{#1}}
\newcommand{\SpecialStringTok}[1]{\textcolor[rgb]{0.31,0.60,0.02}{#1}}
\newcommand{\StringTok}[1]{\textcolor[rgb]{0.31,0.60,0.02}{#1}}
\newcommand{\VariableTok}[1]{\textcolor[rgb]{0.00,0.00,0.00}{#1}}
\newcommand{\VerbatimStringTok}[1]{\textcolor[rgb]{0.31,0.60,0.02}{#1}}
\newcommand{\WarningTok}[1]{\textcolor[rgb]{0.56,0.35,0.01}{\textbf{\textit{#1}}}}
\usepackage{graphicx,grffile}
\makeatletter
\def\maxwidth{\ifdim\Gin@nat@width>\linewidth\linewidth\else\Gin@nat@width\fi}
\def\maxheight{\ifdim\Gin@nat@height>\textheight\textheight\else\Gin@nat@height\fi}
\makeatother
% Scale images if necessary, so that they will not overflow the page
% margins by default, and it is still possible to overwrite the defaults
% using explicit options in \includegraphics[width, height, ...]{}
\setkeys{Gin}{width=\maxwidth,height=\maxheight,keepaspectratio}
% Set default figure placement to htbp
\makeatletter
\def\fps@figure{htbp}
\makeatother
\setlength{\emergencystretch}{3em} % prevent overfull lines
\providecommand{\tightlist}{%
  \setlength{\itemsep}{0pt}\setlength{\parskip}{0pt}}
\setcounter{secnumdepth}{-\maxdimen} % remove section numbering

\title{Exam3}
\author{Ronald Long}
\date{7/9/2020}

\begin{document}
\maketitle

\begin{Shaded}
\begin{Highlighting}[]
\NormalTok{knitr}\OperatorTok{::}\NormalTok{opts_chunk}\OperatorTok{$}\KeywordTok{set}\NormalTok{(}
  \DataTypeTok{warning =} \OtherTok{TRUE}\NormalTok{, }\CommentTok{# show warnings}
  \DataTypeTok{message =} \OtherTok{TRUE}\NormalTok{, }\CommentTok{# show messages}
  \DataTypeTok{error =} \OtherTok{TRUE}\NormalTok{, }\CommentTok{# do not interrupt generation in case of errors,}
  \DataTypeTok{echo =} \OtherTok{TRUE}  \CommentTok{# show R code}
\NormalTok{)}
\end{Highlighting}
\end{Shaded}

\hypertarget{exam-3}{%
\section{Exam 3}\label{exam-3}}

\hypertarget{question-1}{%
\subsection{Question 1}\label{question-1}}

Clear the environment

\begin{Shaded}
\begin{Highlighting}[]
\KeywordTok{rm}\NormalTok{(}\DataTypeTok{list=}\KeywordTok{ls}\NormalTok{(}\DataTypeTok{all=}\OtherTok{TRUE}\NormalTok{))}
\end{Highlighting}
\end{Shaded}

\hypertarget{question-2}{%
\subsection{Question 2}\label{question-2}}

Use the tidycensus package to find the inequality Gini index variable,
there are multiple parts

\begin{Shaded}
\begin{Highlighting}[]
\CommentTok{# a)}
\KeywordTok{library}\NormalTok{(tidycensus)}
\KeywordTok{library}\NormalTok{(tidyverse)}
\end{Highlighting}
\end{Shaded}

\begin{verbatim}
## -- Attaching packages ---------------------------------------------------------------------------------------------- tidyverse 1.3.0 --
\end{verbatim}

\begin{verbatim}
## v ggplot2 3.3.2     v purrr   0.3.4
## v tibble  3.0.1     v dplyr   1.0.0
## v tidyr   1.1.0     v stringr 1.4.0
## v readr   1.3.1     v forcats 0.5.0
\end{verbatim}

\begin{verbatim}
## -- Conflicts ------------------------------------------------------------------------------------------------- tidyverse_conflicts() --
## x dplyr::filter() masks stats::filter()
## x dplyr::lag()    masks stats::lag()
\end{verbatim}

\begin{Shaded}
\begin{Highlighting}[]
\KeywordTok{suppressMessages}\NormalTok{(}\KeywordTok{library}\NormalTok{(bit64))}


\CommentTok{# b)}
\NormalTok{v15 <-}\StringTok{ }\KeywordTok{load_variables}\NormalTok{(}\DataTypeTok{year =} \DecValTok{2015}\NormalTok{,}
                      \StringTok{'acs5'}\NormalTok{)}


\NormalTok{gini_}\DecValTok{2015}\NormalTok{ <-}\StringTok{ }\KeywordTok{get_acs}\NormalTok{(}\DataTypeTok{geography =} \StringTok{"state"}\NormalTok{,}
                 \DataTypeTok{variables =} \KeywordTok{c}\NormalTok{(}\DataTypeTok{GINI =} \KeywordTok{c}\NormalTok{(}\StringTok{"B19083_001"}\NormalTok{)),}
                 \DataTypeTok{year =} \DecValTok{2015}\NormalTok{)}
\end{Highlighting}
\end{Shaded}

\begin{verbatim}
## Getting data from the 2011-2015 5-year ACS
\end{verbatim}

\begin{Shaded}
\begin{Highlighting}[]
\NormalTok{v10 <-}\StringTok{ }\KeywordTok{load_variables}\NormalTok{(}\DataTypeTok{year =} \DecValTok{2010}\NormalTok{,}
                      \StringTok{'acs5'}\NormalTok{)}
\NormalTok{gini_}\DecValTok{2010}\NormalTok{ <-}\StringTok{ }\KeywordTok{get_acs}\NormalTok{(}\DataTypeTok{geography =} \StringTok{"state"}\NormalTok{,}
                        \DataTypeTok{variables =} \KeywordTok{c}\NormalTok{(}\DataTypeTok{GINI =} \KeywordTok{c}\NormalTok{(}\StringTok{"B19083_001"}\NormalTok{)), }\DataTypeTok{year =} \DecValTok{2010}\NormalTok{)}
\end{Highlighting}
\end{Shaded}

\begin{verbatim}
## Getting data from the 2006-2010 5-year ACS
\end{verbatim}

\begin{Shaded}
\begin{Highlighting}[]
\NormalTok{inequality_panel <-}\StringTok{ }\KeywordTok{bind_rows}\NormalTok{(gini_}\DecValTok{2010}\NormalTok{, gini_}\DecValTok{2015}\NormalTok{)}

\KeywordTok{library}\NormalTok{(data.table)}
\end{Highlighting}
\end{Shaded}

\begin{verbatim}
## 
## Attaching package: 'data.table'
\end{verbatim}

\begin{verbatim}
## The following object is masked from 'package:bit':
## 
##     setattr
\end{verbatim}

\begin{verbatim}
## The following objects are masked from 'package:dplyr':
## 
##     between, first, last
\end{verbatim}

\begin{verbatim}
## The following object is masked from 'package:purrr':
## 
##     transpose
\end{verbatim}

\begin{Shaded}
\begin{Highlighting}[]
\KeywordTok{setnames}\NormalTok{(inequality_panel, }\StringTok{'estimate'}\NormalTok{, }\StringTok{'gini'}\NormalTok{)}
\KeywordTok{setnames}\NormalTok{(inequality_panel, }\StringTok{'NAME'}\NormalTok{, }\StringTok{'state'}\NormalTok{)}

\KeywordTok{head}\NormalTok{(inequality_panel)}
\end{Highlighting}
\end{Shaded}

\begin{verbatim}
## # A tibble: 6 x 5
##   GEOID state      variable  gini   moe
##   <chr> <chr>      <chr>    <dbl> <dbl>
## 1 01    Alabama    GINI     0.47  0.003
## 2 02    Alaska     GINI     0.412 0.006
## 3 04    Arizona    GINI     0.453 0.002
## 4 05    Arkansas   GINI     0.459 0.003
## 5 06    California GINI     0.469 0.001
## 6 08    Colorado   GINI     0.455 0.003
\end{verbatim}

\hypertarget{question-3}{%
\subsection{Question 3}\label{question-3}}

reshape the inequality panel wide, so that gini values for 2010 and 2015
hve their own columns

\begin{Shaded}
\begin{Highlighting}[]
\NormalTok{ inequality_wide <-}
\StringTok{    }\NormalTok{inequality_panel }\OperatorTok
\StringTok{    }\KeywordTok{pivot_wider}\NormalTok{(}\DataTypeTok{id_cols =} \KeywordTok{c}\NormalTok{(}\StringTok{"2010"}\NormalTok{, }\StringTok{"2015"}\NormalTok{), }\CommentTok{# unique IDs}
                \DataTypeTok{names_from =} \StringTok{"year"}\NormalTok{, }\CommentTok{# names for new wide vars}
                \DataTypeTok{values_from =} \StringTok{"gdp_current"}\NormalTok{, }\CommentTok{# data to put in new wide vars}
                \DataTypeTok{names_prefix =} \StringTok{"year"}\NormalTok{ )}
\end{Highlighting}
\end{Shaded}

\begin{verbatim}
## Error: Can't subset columns that don't exist.
## x Column `year` doesn't exist.
\end{verbatim}

\begin{Shaded}
\begin{Highlighting}[]
\KeywordTok{head}\NormalTok{(inequality_wide)}
\end{Highlighting}
\end{Shaded}

\begin{verbatim}
## Error in head(inequality_wide): object 'inequality_wide' not found
\end{verbatim}

\hypertarget{question-4}{%
\subsection{Question 4}\label{question-4}}

Reshape the inequality\_wide to long format

\begin{Shaded}
\begin{Highlighting}[]
\NormalTok{inequality_long <-}
\StringTok{    }\NormalTok{inequality_wide }\OperatorTok
\StringTok{    }\KeywordTok{pivot_longer}\NormalTok{(}\DataTypeTok{cols =} \KeywordTok{starts_with}\NormalTok{(}\StringTok{"year"}\NormalTok{), }\CommentTok{# use columns starting with "year"}
                 \DataTypeTok{names_to =}\StringTok{"year"}\NormalTok{, }\CommentTok{# name of new column, on the basis it starts with}
                 \DataTypeTok{names_prefix =} \StringTok{"year"}\NormalTok{, }\CommentTok{# part of string to drop, would carry}
                 \DataTypeTok{values_to =} \StringTok{"gdp_current"}\NormalTok{, }\CommentTok{# where to put numeric values}
                 \DataTypeTok{values_drop_na =} \OtherTok{FALSE}\NormalTok{) }\OperatorTok\StringTok{ }\CommentTok{# don't drop NAs}
\StringTok{    }\KeywordTok{filter}\NormalTok{(}\OperatorTok{!}\NormalTok{(current_amount}\OperatorTok{==}\DecValTok{0}\NormalTok{)) }\CommentTok{# drop observations with no disb}
\end{Highlighting}
\end{Shaded}

\begin{verbatim}
## Error in eval(lhs, parent, parent): object 'inequality_wide' not found
\end{verbatim}

\hypertarget{question-5}{%
\subsection{Question 5}\label{question-5}}

show the r code that inequality\_panel and inequality\_long have the
same number of observations

\begin{Shaded}
\begin{Highlighting}[]
\KeywordTok{str}\NormalTok{(inequality_long)}
\end{Highlighting}
\end{Shaded}

\begin{verbatim}
## Error in str(inequality_long): object 'inequality_long' not found
\end{verbatim}

\begin{Shaded}
\begin{Highlighting}[]
\KeywordTok{str}\NormalTok{(inequality_panel)}
\end{Highlighting}
\end{Shaded}

\begin{verbatim}
## tibble [104 x 5] (S3: tbl_df/tbl/data.frame)
##  $ GEOID   : chr [1:104] "01" "02" "04" "05" ...
##  $ state   : chr [1:104] "Alabama" "Alaska" "Arizona" "Arkansas" ...
##  $ variable: chr [1:104] "GINI" "GINI" "GINI" "GINI" ...
##  $ gini    : num [1:104] 0.47 0.412 0.453 0.459 0.469 0.455 0.482 0.436 0.535 0.471 ...
##  $ moe     : num [1:104] 0.003 0.006 0.002 0.003 0.001 0.003 0.003 0.005 0.005 0.002 ...
\end{verbatim}

\hypertarget{question-6}{%
\subsection{Question 6}\label{question-6}}

collapse the inequality dataframe by state to obtain a single mean

\begin{Shaded}
\begin{Highlighting}[]
\NormalTok{inequality_collapse <-}
\NormalTok{inequality_long }\OperatorTok
\KeywordTok{group_by}\NormalTok{(state) }\OperatorTok\StringTok{ }\CommentTok{# tell R the unique IDs}
\KeywordTok{summarize}\NormalTok{(}\KeywordTok{across}\NormalTok{(}\KeywordTok{where}\NormalTok{(is.numeric), sum)) }\OperatorTok\StringTok{ }\CommentTok{# summarize numeric vars by sum}
\end{Highlighting}
\end{Shaded}

\begin{verbatim}
## Error: <text>:6:0: unexpected end of input
## 4: summarize(across(where(is.numeric), sum)) %>% # summarize numeric vars by sum
## 5: 
##   ^
\end{verbatim}

\hypertarget{question-7}{%
\subsection{Question 7}\label{question-7}}

Produce a map of the United States that colors in the state polygons by
their mean gini scores from inequality\_collapse

\begin{Shaded}
\begin{Highlighting}[]
\KeywordTok{library}\NormalTok{(easypackages)}
\KeywordTok{packages}\NormalTok{(}\StringTok{'rio'}\NormalTok{, }\StringTok{'tidyverse'}\NormalTok{, }\StringTok{'googlesheets4'}\NormalTok{, }\StringTok{'labelled'}\NormalTok{, }\StringTok{'data.table'}\NormalTok{,}
         \StringTok{'varhandle'}\NormalTok{, }\StringTok{'ggrepel'}\NormalTok{, }\StringTok{'geosphere'}\NormalTok{, }\StringTok{'rgeos'}\NormalTok{, }\StringTok{'viridis'}\NormalTok{, }\StringTok{'mapview'}\NormalTok{,}
         \StringTok{'rnaturalearth'}\NormalTok{, }\StringTok{'rnaturalearthdata'}\NormalTok{, }\StringTok{'devtools'}\NormalTok{, }\StringTok{'rnaturalearthhires'}\NormalTok{,}
         \StringTok{'raster'}\NormalTok{, }\StringTok{'sp'}\NormalTok{, }\StringTok{'sf'}\NormalTok{, }\StringTok{'ggsflabel'}\NormalTok{, }\StringTok{'Imap'}\NormalTok{)}
\end{Highlighting}
\end{Shaded}

\begin{verbatim}
## Loading required package: rio
\end{verbatim}

\begin{verbatim}
## Loading required package: googlesheets4
\end{verbatim}

\begin{verbatim}
## Loading required package: labelled
\end{verbatim}

\begin{verbatim}
## Loading required package: varhandle
\end{verbatim}

\begin{verbatim}
## Loading required package: ggrepel
\end{verbatim}

\begin{verbatim}
## Loading required package: geosphere
\end{verbatim}

\begin{verbatim}
## Loading required package: rgeos
\end{verbatim}

\begin{verbatim}
## Loading required package: sp
\end{verbatim}

\begin{verbatim}
## rgeos version: 0.5-3, (SVN revision 634)
##  GEOS runtime version: 3.8.0-CAPI-1.13.1 
##  Linking to sp version: 1.4-2 
##  Polygon checking: TRUE
\end{verbatim}

\begin{verbatim}
## Loading required package: viridis
\end{verbatim}

\begin{verbatim}
## Loading required package: viridisLite
\end{verbatim}

\begin{verbatim}
## Loading required package: mapview
\end{verbatim}

\begin{verbatim}
## Loading required package: rnaturalearth
\end{verbatim}

\begin{verbatim}
## Loading required package: rnaturalearthdata
\end{verbatim}

\begin{verbatim}
## Loading required package: devtools
\end{verbatim}

\begin{verbatim}
## Loading required package: usethis
\end{verbatim}

\begin{verbatim}
## Loading required package: rnaturalearthhires
\end{verbatim}

\begin{verbatim}
## Loading required package: raster
\end{verbatim}

\begin{verbatim}
## 
## Attaching package: 'raster'
\end{verbatim}

\begin{verbatim}
## The following object is masked from 'package:data.table':
## 
##     shift
\end{verbatim}

\begin{verbatim}
## The following object is masked from 'package:dplyr':
## 
##     select
\end{verbatim}

\begin{verbatim}
## The following object is masked from 'package:tidyr':
## 
##     extract
\end{verbatim}

\begin{verbatim}
## Loading required package: sf
\end{verbatim}

\begin{verbatim}
## Linking to GEOS 3.8.0, GDAL 3.0.4, PROJ 6.3.1
\end{verbatim}

\begin{verbatim}
## Loading required package: ggsflabel
\end{verbatim}

\begin{verbatim}
## 
## Attaching package: 'ggsflabel'
\end{verbatim}

\begin{verbatim}
## The following objects are masked from 'package:ggplot2':
## 
##     geom_sf_label, geom_sf_text, StatSfCoordinates
\end{verbatim}

\begin{verbatim}
## Loading required package: Imap
\end{verbatim}

\begin{verbatim}
## 
## Attaching package: 'Imap'
\end{verbatim}

\begin{verbatim}
## The following object is masked from 'package:purrr':
## 
##     imap
\end{verbatim}

\begin{verbatim}
## All packages loaded successfully
\end{verbatim}

\begin{Shaded}
\begin{Highlighting}[]
\CommentTok{#github}
\NormalTok{devtools}\OperatorTok{::}\KeywordTok{install_github}\NormalTok{(}\StringTok{'ropensci/rnaturalearthhires'}\NormalTok{)}
\end{Highlighting}
\end{Shaded}

\begin{verbatim}
## WARNING: Rtools is required to build R packages, but is not currently installed.
## 
## Please download and install Rtools custom from https://cran.r-project.org/bin/windows/Rtools/.
\end{verbatim}

\begin{verbatim}
## Skipping install of 'rnaturalearthhires' from a github remote, the SHA1 (2ed7a937) has not changed since last install.
##   Use `force = TRUE` to force installation
\end{verbatim}

\begin{Shaded}
\begin{Highlighting}[]
\KeywordTok{library}\NormalTok{(devtools)}
\KeywordTok{library}\NormalTok{(remotes)}
\end{Highlighting}
\end{Shaded}

\begin{verbatim}
## 
## Attaching package: 'remotes'
\end{verbatim}

\begin{verbatim}
## The following objects are masked from 'package:devtools':
## 
##     dev_package_deps, install_bioc, install_bitbucket, install_cran,
##     install_deps, install_dev, install_git, install_github,
##     install_gitlab, install_local, install_svn, install_url,
##     install_version, update_packages
\end{verbatim}

\begin{verbatim}
## The following object is masked from 'package:usethis':
## 
##     git_credentials
\end{verbatim}

\begin{Shaded}
\begin{Highlighting}[]
\NormalTok{devtools}\OperatorTok{::}\KeywordTok{install_github}\NormalTok{(}\StringTok{'yutannihilation/ggsflabel'}\NormalTok{)}
\end{Highlighting}
\end{Shaded}

\begin{verbatim}
## WARNING: Rtools is required to build R packages, but is not currently installed.
## 
## Please download and install Rtools custom from https://cran.r-project.org/bin/windows/Rtools/.
\end{verbatim}

\begin{verbatim}
## Skipping install of 'ggsflabel' from a github remote, the SHA1 (a489481b) has not changed since last install.
##   Use `force = TRUE` to force installation
\end{verbatim}

\begin{Shaded}
\begin{Highlighting}[]
\KeywordTok{libraries}\NormalTok{(}\StringTok{'rio'}\NormalTok{, }\StringTok{'tidyverse'}\NormalTok{, }\StringTok{'googlesheets4'}\NormalTok{, }\StringTok{'labelled'}\NormalTok{, }\StringTok{'data.table'}\NormalTok{,}
          \StringTok{'varhandle'}\NormalTok{, }\StringTok{'ggrepel'}\NormalTok{, }\StringTok{'geosphere'}\NormalTok{, }\StringTok{'rgeos'}\NormalTok{, }\StringTok{'viridis'}\NormalTok{, }\StringTok{'mapview'}\NormalTok{,}
          \StringTok{'rnaturalearth'}\NormalTok{, }\StringTok{'rnaturalearthdata'}\NormalTok{, }\StringTok{'devtools'}\NormalTok{, }\StringTok{'rnaturalearthhires'}\NormalTok{,}
          \StringTok{'raster'}\NormalTok{, }\StringTok{'sp'}\NormalTok{, }\StringTok{'sf'}\NormalTok{, }\StringTok{'ggsflabel'}\NormalTok{, }\StringTok{'Imap'}\NormalTok{)}
\end{Highlighting}
\end{Shaded}

\begin{verbatim}
## All packages loaded successfully
\end{verbatim}

\begin{Shaded}
\begin{Highlighting}[]
\NormalTok{USA_map =}\StringTok{ }\KeywordTok{ggplot}\NormalTok{() }\OperatorTok{+}
\KeywordTok{geom_sf}\NormalTok{(}\DataTypeTok{data =}\NormalTok{ inequality_collapse) }\OperatorTok{+}
\KeywordTok{geom_sf}\NormalTok{(}\DataTypeTok{data =}\NormalTok{ inequality_collapse, }\KeywordTok{aes}\NormalTok{(}\DataTypeTok{fill=}\StringTok{`}\DataTypeTok{Log Value}\StringTok{`}\NormalTok{)) }\OperatorTok{+}
\KeywordTok{scale_fill_viridis}\NormalTok{(}\DataTypeTok{option =} \StringTok{"viridis"}\NormalTok{) }\OperatorTok{+}
\KeywordTok{ggtitle}\NormalTok{(}\StringTok{"USA GINI Scores"}\NormalTok{) }\OperatorTok{+}
\KeywordTok{theme}\NormalTok{(}\DataTypeTok{plot.title =} \KeywordTok{element_text}\NormalTok{(}\DataTypeTok{hjust =} \FloatTok{0.5}\NormalTok{)) }\OperatorTok{+}
\KeywordTok{theme_void}\NormalTok{()}
\end{Highlighting}
\end{Shaded}

\begin{verbatim}
## Error in fortify(data): object 'inequality_collapse' not found
\end{verbatim}

\begin{Shaded}
\begin{Highlighting}[]
\KeywordTok{print}\NormalTok{(USA_map)}
\end{Highlighting}
\end{Shaded}

\begin{verbatim}
## Error in h(simpleError(msg, call)): error in evaluating the argument 'x' in selecting a method for function 'print': object 'USA_map' not found
\end{verbatim}

\hypertarget{question-8}{%
\subsection{Question 8}\label{question-8}}

WDI package to import data in GDP in current US dollars

\begin{Shaded}
\begin{Highlighting}[]
\KeywordTok{library}\NormalTok{(WDI)}
\NormalTok{GDP =}\StringTok{ }\KeywordTok{WDI}\NormalTok{(}\DataTypeTok{country =} \StringTok{"all"}\NormalTok{, }\DataTypeTok{indicator =} \KeywordTok{c}\NormalTok{(}\StringTok{"NY.GDP.MKTP.CD"}\NormalTok{),}
\DataTypeTok{start =} \DecValTok{2006}\NormalTok{, }\CommentTok{# start of foreign aid data}
\DataTypeTok{end =} \DecValTok{2007}\NormalTok{, }\CommentTok{# end of of foreign aid data}
\DataTypeTok{extra =} \OtherTok{FALSE}\NormalTok{, }\DataTypeTok{cache =} \OtherTok{NULL}\NormalTok{)}
\KeywordTok{library}\NormalTok{(data.table)}
\KeywordTok{setnames}\NormalTok{(GDP,}\StringTok{"NY.GDP.MKTP.CD"}\NormalTok{, }\StringTok{"gdp_current"}\NormalTok{)}
\end{Highlighting}
\end{Shaded}

\hypertarget{question-9}{%
\subsection{Question 9}\label{question-9}}

Deflate the gdp\_current to constant 2010 or 2015 us dollars

\begin{Shaded}
\begin{Highlighting}[]
\NormalTok{deflator =}\StringTok{ }\KeywordTok{subset}\NormalTok{(gdp)}
\end{Highlighting}
\end{Shaded}

\begin{verbatim}
## Error in h(simpleError(msg, call)): error in evaluating the argument 'x' in selecting a method for function 'subset': object 'gdp' not found
\end{verbatim}

\begin{Shaded}
\begin{Highlighting}[]
\KeywordTok{subset}\NormalTok{(deflator, deflator}\OperatorTok{==}\DecValTok{100}\NormalTok{)}
\end{Highlighting}
\end{Shaded}

\begin{verbatim}
## Error in h(simpleError(msg, call)): error in evaluating the argument 'x' in selecting a method for function 'subset': object 'deflator' not found
\end{verbatim}

\begin{Shaded}
\begin{Highlighting}[]
\NormalTok{GDP=}\StringTok{ }\KeywordTok{left_join}\NormalTok{(deflator,}
\NormalTok{deflator,}
\DataTypeTok{by=}\KeywordTok{c}\NormalTok{(}\StringTok{"year"}\NormalTok{))}
\end{Highlighting}
\end{Shaded}

\begin{verbatim}
## Error in left_join(deflator, deflator, by = c("year")): object 'deflator' not found
\end{verbatim}

\begin{Shaded}
\begin{Highlighting}[]
\NormalTok{deflated_data}\OperatorTok{$}\NormalTok{deflated_amount =}\StringTok{ }\NormalTok{deflated_data}\OperatorTok{$}\NormalTok{current_amount}\OperatorTok{/}
\NormalTok{(deflated_data}\OperatorTok{$}\NormalTok{deflator}\OperatorTok{/}\DecValTok{100}\NormalTok{)}
\end{Highlighting}
\end{Shaded}

\begin{verbatim}
## Error in eval(expr, envir, enclos): object 'deflated_data' not found
\end{verbatim}

\begin{Shaded}
\begin{Highlighting}[]
\KeywordTok{head}\NormalTok{(deflated_data)}
\end{Highlighting}
\end{Shaded}

\begin{verbatim}
## Error in h(simpleError(msg, call)): error in evaluating the argument 'x' in selecting a method for function 'head': object 'deflated_data' not found
\end{verbatim}

I thought that the 2015 year would be a good baseline because it seems
to be set at 100 for the deflator data.

\hypertarget{question-10}{%
\subsection{Question 10}\label{question-10}}

The user interface with inputs, and outputs. There is also a server that
interacts with it, make sure to execute. The server allows you to render
certain things.

\hypertarget{question-11}{%
\subsection{Question 11}\label{question-11}}

pull the pdf from mike denly's webpage

\begin{Shaded}
\begin{Highlighting}[]
\KeywordTok{library}\NormalTok{(pdftools)}
\end{Highlighting}
\end{Shaded}

\begin{verbatim}
## Using poppler version 0.73.0
\end{verbatim}

\begin{Shaded}
\begin{Highlighting}[]
\KeywordTok{library}\NormalTok{(tidyr)}
\KeywordTok{library}\NormalTok{(tidytext)}
\KeywordTok{library}\NormalTok{(dplyr)}
\KeywordTok{library}\NormalTok{(stringr)}
\KeywordTok{library}\NormalTok{(ggplot2)}

\NormalTok{armeniatext=}\KeywordTok{pdf_text}\NormalTok{(}\DataTypeTok{pdf =} \StringTok{'https://pdf.usaid.gov/pdf_docs/PA00TNMG.pdf'}\NormalTok{)}

\NormalTok{armeniatext}
\end{Highlighting}
\end{Shaded}

\begin{verbatim}
##  [1] "                                                                                                                                  PHOTO BY SEROUJ OURISHIAN\r\nGOVERNANCE IN ARMENIA\r\nAn Evidence Review for Learning, Evaluation and\r\nResearch Activity II (LER II)\r\nJANUARY 2019\r\nDISCLAIMER: The authors' views expressed in this publication do not necessarily reflect the views of the United States Agency for\r\nInternational Development or the United States Government.\r\n"                                                                                                                                                                                                                                                                                                                                                                                                                                                                                                                                                                                                                                                                                                                                                                                                                                                                                                                                                                                                                                                                                                                                                                                                                                                                                                                                                                                                                                                                                                                                                                                                                                                                                                                                                                                                                                                                                                                                                                                                                                                                                                                                                                                                                                                                                                                                                                                                                                                                                                                                                                                                                                                                                                                                                                                                                                                                                                                                                                                                                                                                                                                                                                                                                                                                                                                                                                                                                                                                                                                                                                                                                                                                                                                                                                                                                                                                                                                                                                                                                                                                                                                                                                                                                                                                                                                                                                                                                                                                                                                                                                                                                                                                                                                                  
##  [2] "This document was produced for review by the United States Agency for International Development,\r\nDemocracy, Human Rights and Governance Center under the Learning, Evaluation and Research\r\nActivity II (LER II) contract: GS10F0218U/7200AA18M00017.\r\nPrepared by:\r\nThe Cloudburst Group\r\n8400 Corporate Drive, Suite 550\r\nLandover, MD 20785-2238\r\nTel: 301-918-4400\r\n"                                                                                                                                                                                                                                                                                                                                                                                                                                                                                                                                                                                                                                                                                                                                                                                                                                                                                                                                                                                                                                                                                                                                                                                                                                                                                                                                                                                                                                                                                                                                                                                                                                                                                                                                                                                                                                                                                                                                                                                                                                                                                                                                                                                                                                                                                                                                                                                                                                                                                                                                                                                                                                                                                                                                                                                                                                                                                                                                                                                                                                                                                                                                                                                                                                                                                                                                                                                                                                                                                                                                                                                                                                                                                                                                                                                                                                                                                                                                                                                                                                                                                                                                                                                                                                                                                                                                                                                                                                                                                                                                                                                                                                                                                                                                                                                                                                        
##  [3] "CONTENTS\r\n1.      EXECUTIVE SUMMARY                                            1\r\n2.      GOVERNANCE EVIDENCE REVIEW                                   3\r\n   2.1.   GOVERNANCE, DEMOCRACY, AND ARMENIA: AN OVERVIEW            3\r\n   2.2.   CONSOLIDATION OF STATE INSTITUTIONS                        7\r\n   2.3.   SERVICE DELIVERY                                          17\r\n   2.4.   CIVIL SERVICE REFORM                                      24\r\n   2.5.   MISCELLANEOUS FACTORS THAT CONTRIBUTE TO GOVERNANCE       29\r\n   2.6.   CONCLUSION                                                31\r\n3.      ANALYSIS OF ARMENIAN V-DEM GOVERNANCE INDICATORS            33\r\n   3.1.   INTRODUCTION                                              33\r\n   3.2.   OVERALL MEASURES FOR ARMENIA                              33\r\n   3.3.   JUDICIARY                                                 35\r\n   3.4.   LIBERAL DEMOCRACY                                         36\r\n   3.5.   CORRUPTION IN ARMENIA                                     37\r\n   3.6.   DEMOCRACY & CORRUPTION ACROSS COUNTRIES                   38\r\n   3.7.   COMPARING V-DEM TO POLITY (WITH FREEDOM HOUSE IMPUTATION) 40\r\n   3.8.   CONCLUSION                                                41\r\n4.      REFERENCES                                                  42\r\n"                                                                                                                                                                                                                                                                                                                                                                                                                                                                                                                                                                                                                                                                                                                                                                                                                                                                                                                                                                                                                                                                                                                                                                                                                                                                                                                                                                                                                                                                                                                                                                                                                                                                                                                                                                                                                                                                                                                                                                                                                                                                                                                                                                                                                                                                                                                                                                                                                                                                                                                                                                                                                                                                                                                                                                                                                                                                                                                                                                                                                                                                                                                                                                                                                                                                                                                                                                                                                                                                                                                                                                                                                                                                                                                                                                                                                                 
##  [4] "  1.        EXECUTIVE SUMMARY1\r\nThis report contains a Governance sector evidence review and an analysis of governance indicators\r\nfrom the Varieties of Democracy (V-Dem) dataset.\r\nThe evidence review begins by defining governance, making appropriate distinctions with democracy,\r\nand providing background on Armenia and relevant regional comparisons. Consistent with guidance\r\nfrom the USAID Armenia Mission, we then turn to three key issue areas—the consolidation of state\r\ninstitutions, service delivery, and civil service reform—with appropriate attention to regional transitional\r\nexperiences as well as policy formulation processes. The analysis of V-Dem data displays and discusses\r\ndescriptive trends in Armenian governance and democracy (Lindberg et al., 2014). Specifically, the V-\r\nDem analysis covers measures and sub-measures of electoral democracy, the judiciary, liberal\r\ndemocracy, and corruption in Armenia over time. Additionally, the document compares Armenia’s\r\nelectoral democracy scores to those of other countries in the region, including Georgia, Ukraine,\r\nMoldova, and Kyrgyzstan.\r\nOn an institutional level, we note that there is reason for optimism in Armenia, given the parliamentary\r\nand proportional representation institutions Armenia has established. While not guaranteed, there is\r\nmuch scholarly consensus that these institutional features are the best among the available alternatives.\r\nTurning to factors that may be more easily manipulated in the short- to medium-term, both by the\r\ngovernment or international assistance providers, we identify lessons learned:\r\n• Consolidation of State Institutions\r\n    - Exercise caution in moving forward with decentralization. There is cause for concern about the\r\n        efficacy of decentralization, though we have identified both the promise and pitfalls of a\r\n        decentralization approach.\r\n    - Address petty corruption, and carefully phase-in efforts to address higher-level corruption.\r\n    - Continue with e-governance in an effort to streamline many institutional functions.\r\n    - Good institutional reform does not entail exporting the right strategy from one context and using\r\n        it in another one. Instead, good institutional reform first builds off robust political economy\r\n        analysis to inform feasible action. Robust political economy analysis should not only make use of\r\n        traditional indicators but also the dimensions of the political settlements framework, paying\r\n        particular attention to the implementation concerns stressed by the Problem-Driven Iterative\r\n        Adaptation (PDIA) approach. Thus, institutional reform is not about grand plans or overarching\r\n        solutions but about gradually undertaking reforms that are suitable to particular contexts.\r\n1 This Evidence Review was prepared by Michael Denly, Michael Findley, John Gerring, and Rachel Wellhausen, all affiliated with\r\nthe University of Texas at Austin. We would like to thank the following Research Affiliates at the University of Texas at\r\nAustin’s Innovations for Peace and Development for research assistance: Rachel Boles, Evelin Caro Gutierrez, Erica Colston,\r\nHannah Greer, Paige Johnson, Judy Lane, Amanda Long, Amila Lulo, Felipa Mendez, Tyler Morrow, Tomilayo Ogungbamigbe,\r\nMobin Piracha, JP Repetto, Ethan Tenison, Adityamohan Tantravahi, and Luisa Venegoni.\r\nUSAID.GOV                                                           GOVERNANCE IN ARMENIA: AN EVIDENCE REVIEW              |   1\r\n"                                                                                                                                                                                                                                                                                                                                                                                                                                                                                                                                                                                                                                                                                                                                                                                                                                                                                                                                                                                                                                                                                                                                                                                                                                                                                                                                                                                                                                                                                                    
##  [5] "• Service Delivery\r\n   - Participatory programs are likely to be beneficial both for governance and citizen-level satisfaction\r\n      with more democracy following Armenia’s political transition. Promising potential interventions\r\n      include opening more citizen service centers and improving the efficacy of existing ones, as well as\r\n      perhaps initiating some combination of the following: participatory/open budgeting, right to\r\n      information laws, grievance redress mechanisms, hotlines, social audits, and the introduction of\r\n      social programs. All potential interventions should have some form of an e-governance\r\n      component.\r\n• Civil Service Reform\r\n   - Armenia should be cautious about following Georgia’s “big bang” approach to civil service reform,\r\n      especially given former Georgian President Mikhail Saakashvili’s failure to adequately tackle\r\n      corruption when he subsequently served as Governor of Odessa and advisor to President\r\n      Poroshenko in Ukraine.\r\n   - Armenia could also pursue some relatively new interventions, including random inspections for\r\n      petty corruption, more e-government (e.g. biometric smart cards), testing for pro-social\r\n      motivations when hiring (but quietly), performance-based postings, and transparency in hiring and\r\n      civil servants’ tax records.\r\n• Comparisons and Process\r\n   - Regional comparisons are helpful but need to be considered carefully. Georgia, for example, offers\r\n      a model for e-governance reforms and addressing petty corruption, but is likely different than\r\n      Armenia concerning timing and scale.\r\n   - Reforms are likely to be most successful if they proceed at a moderate pace and do not alienate\r\n      important players—even if those important players ultimately impede efforts to achieve good\r\n      governance.\r\nUSAID.GOV                                                GOVERNANCE IN ARMENIA: AN EVIDENCE REVIEW     |  2\r\n"                                                                                                                                                                                                                                                                                                                                                                                                                                                                                                                                                                                                                                                                                                                                                                                                                                                                                                                                                                                                                                                                                                                                                                                                                                                                                                                                                                                                                                                                                                                                                                                                                                                                                                                                                                                                                                                                                                                                                                                                                                                                                                                                                                                                                                                                                                                                                                                                                                                                                                                                                                                                                                                                                                                                                                                                                                                                                                                                                                                                                                                                                                                                                 
##  [6] "  2.       GOVERNANCE EVIDENCE REVIEW\r\n  2.1.       GOVERNANCE, DEMOCRACY, AND ARMENIA: AN OVERVIEW\r\nBefore delving into to the specifics of governance in Armenia, it is important to define the overarching\r\nconcepts of governance and democracy, and explain how they relate, especially given that practitioners\r\nand scholars alike sometimes conflate them. In doing so, we preview which governance-related topics\r\nwe consider to be salient for in Armenia, topics that we examine in further detail in later sections.\r\nGovernance is a regime-neutral 2 process that encompasses states’ abilities to (1) make, implement, and\r\nenforce informal and formal rules and laws; and (2) deliver public goods and services to their citizens\r\n(Fukuyama, 2013; Mann, 1984). 3 In line with the 2017 World Development Report on Governance and the Law,\r\ngood governance is also not only a feature of capable public administration but also a proper balance of power\r\nbetween state and nonstate actors, including citizens, elites, and civil society (World Bank, 2017b, 3). In\r\nthis Governance Evidence Review for Armenia, we devote particular attention to the consolidation of\r\nstate institutions, public goods/service delivery, and the civil service, all of which accord with Fukuyama’s\r\n(2013) conceptualization of governance.\r\nWhereas governance primarily concerns the implementation of rules/laws and public goods/service delivery,\r\nat its core democracy entails “fully contested elections with full suffrage and the absence of massive\r\nfraud, combined with effective guarantees of civil liberties, including freedom of speech, assembly, and\r\nassociation” (Collier and Levitsky, 1997, 434). 4 Armenia’s recent shift from a semi-presidential system\r\n(Markarov, 2016) to parliamentary system, vibrant civil society, and its recent political transition of April\r\n2018 notably raise the possibility of a shift to a more fully democratic regime. These are among the\r\nreasons why The Economist named Armenia its country of the year for 2018 (The Economist, 2018).\r\nAlthough governance and democracy are conceptually distinct, 5 this Governance Evidence Review draws\r\nfrom the democracy literature for multiple reasons. Notably, democracies outperform autocracies in\r\nproviding public goods and services, including in health, education, environmental protection, nutrition,\r\nroad infrastructure, and electricity (Lake and Baum, 2001; Bueno de Mesquita et al., 2003; Bernauer and\r\nKoubi, 2009; Burgess et al., 2015; Blaydes and Kayser, 2011; Min, 2015; Lizzeri and Persico, 2004; Cao and\r\nWard, 2015; Harding and Stasavage, 2014; Besley and Kudamatsu, 2006; Kudamatsu, 2012). 6 As\r\ncompared to autocracies, democracies also facilitate more citizen-level collective action and transparency of\r\ninformation that can contribute to more equitable rule implementation and policies (Acemoglu and\r\nRobinson, 2006; Hollyer, Rosendorff, and Vreeland, 2018; Pande, 2011). Although Armenia has low\r\ncitizen-level satisfaction with government institutions and service delivery, the country boasts a highly\r\neducated and cohesive population with a penchant for effective protest—attributes that facilitate\r\n2 By “regime-neutral” we mean that governance and democracy are not the same (see Fukuyama, 2013; Nooruddin,\r\n2009).\r\n3 Governance can also mean “international cooperation of nonsovereign bodies outside the state system\r\n(international/[global] governance)” and “the regulation of social behavior through networks and other\r\nnonhierarchical mechanisms (governing without government)” (Fukuyama, 2016, 89). Both international governance\r\nand governing without government fall outside the scope of this Evidence Review.\r\n4 The Collier and Levitsky (1997, 434) procedural minimum definition of democracy is far from the only one in the\r\nliterature. For more information, readers should consult, inter alia, Schmitter and Karl (1991) and Lindberg et al.\r\n(2014).\r\n5 For a discussion, see Fukuyama (2013) and Nooruddin (2009).\r\n6 Ross (2006) finds that democracies are not superior in preventing infant mortality, but Kudamatsu (2012) suggests\r\nthat the finding is not robust to the inclusion of variables that Ross (2006) omits.\r\nUSAID.GOV                                                          GOVERNANCE IN ARMENIA: AN EVIDENCE REVIEW      | 3\r\n"                                                                                                                                                                                                                                                                                                                                                                                                                                                                                                                                                                                                                                                
##  [7] "governance and democratic consolidation (Andresyan and Derlugian 2015; EBRD; 2016; UNESCO,\r\n2018; Derlugian and Hovhannisyan, 2018).\r\nDespite the above governance-related advantages of democracies over autocracies, democracies are not\r\nnecessarily superior in all dimensions of governance. One set of challenges for young democracies such as\r\nArmenia, in particular, concerns the credibility of politicians’ policy promises and the unbiased application of\r\nthe rule of law (Keefer, 2007a,b; Keefer and Vlaicu, 2008; Scartascini et al., 2010; Brinks, Leiras and\r\nMainwaring, 2014; O’Donnell, 2004; Gehlbach and Keefer, 2011, Paturyan and Stefes, 2017). In their\r\nabsence, regimes notably suffer from the consequences of corruption (the misuse of public office for private\r\ngain), clientelism (the exchange of resources, services, and jobs for political support), low horizontal\r\naccountability (inability of the bureaucracy to keep checks on its itself), 7 and low vertical accountability\r\n(citizens do not elect politicians who enhance their welfare). In the young democracy of Armenia, all of these\r\nfeatures constitute particular challenges or risks (Paturyan and Stefes, 2017), which we address below.\r\nThe abilities of bureaucracies to consistently implement the will of those in power across a state’s\r\nterritory 8 is another area of governance in which democracies do not necessarily outperform\r\nautocracies. One reason is that a state’s overall bureaucratic implementation capacity is largely a\r\nfunction of its ability to maintain the “monopoly of violence.” 9 As is well-known, many autocracies have\r\nstrong militaries that are capable of quashing internal dissent and outside pressures. 10 As indicated by its\r\nloss of territory to Azerbaijan in 2016, Armenia’s military is rather weak, and elite co-optation hampers\r\nand defines the strength of the country’s bureaucracy (EBRD 2016; Derlugian and Hovhannisyan, 2018),\r\nThe final major dimension of governance concerns “embedded autonomy,” and democracies do not\r\nnecessarily outperform autocracies in terms of embedded autonomy either (Evans, 1995). Embedded\r\nautonomy comprises, first, the bureaucracy’s ability to hold a positive, symbiotic relationship with the\r\nprivate sector—that is, to design policies that facilitate economic growth, without resorting to\r\npatronage and favoritism. Second, embedded autonomy refers to the state’s ability to hold a\r\nmeritocratically recruited and capable civil service. Given that clientelism, corruption, and monopoly\r\ncharacterize Armenia’s public sector, the lack of embedded autonomy defines Armenia (Paturyan and\r\nStefes, 2017). With regard to the civil service, independence and lack of criteria in hiring are notable issues\r\n(OECD, 2011), as are the country’s past inability to enact meaningful reforms, such as in policing\r\n(Shahnazarian and Light, 2018).\r\nARMENIA’S HISTORICAL, DEMOGRAPHIC, AND GEOPOLITICAL CONTEXT\r\nIn discussing the menu of governance options below with a specific focus on the consolidation of state\r\ninstitutions, public goods/service delivery, and civil service reform, we make reference to the Armenian\r\ncontext throughout. Because we make sector-specific references, we first provide a brief overview of the\r\nArmenian context here. Given the direction to make region-specific comparisons, and in order for those\r\ncomparisons to be properly contextualized, in the next subsection we include background on four regional\r\ncomparison countries: Georgia, Ukraine, Moldova, and Kyrgyzstan.\r\n7 For a discussion, see O’Donnell (1998).\r\n8 A state’s capacity to implement its will across its territory is what Mann (1984) calls “infrastructural power”.\r\n9 When a state has the monopoly of violence, it means that it is the unique entity capable of legitimately exercising\r\nthe use of force in a whole territory (Weber, 1978). Challenges concerning the monopoly violence are not unique to\r\ndeveloping countries. For example, Colombia and Mexico, two current members of the OECD, continue to\r\nexperience monopoly of violence problems with factions that accumulate their power as a result of the drug trade.\r\n10 As Fukuyama (2011) recounts, drawing on Ancient China, bureaucracies found their origin in the military.\r\nUSAID.GOV                                                          GOVERNANCE IN ARMENIA: AN EVIDENCE REVIEW       | 4\r\n"                                                                                                                                                                                                                                                                                                                                                                                                                                                                                                                                                                                                 
##  [8] "Armenia is a landlocked country in the Caucasus region of Central Asia. In the modern era, Armenia first\r\ndeclared itself a republic in 1918, but this was short-lived. The Soviet Union annexed Armenia in 1922 and\r\nheld on until 1991, when Armenia achieved its independence. Today, about one-third of Armenia’s\r\nmono-ethnic population of nearly 3 million people reside in the country’s capital of Yerevan.\r\nArmenia is a primarily Christian country, though its neighbors are mostly Muslim.\r\nThe geopolitical situation of Armenia is complex. Due to serious tensions with Azerbaijan and Turkey,\r\nArmenia maintains closed borders with both countries. Competition with the Azerbaijan drove Armenian\r\nindependence more than its desire for democracy, and, in 2016, Armenia suffered some territorial\r\nlosses during its conflict with Azerbaijan (Iskandaryan, 2012; Markarov, 2016). Armenia’s tense relationship\r\nwith Turkey dates back to the Ottoman Empire, when the Ottomans subjected Armenians to forced\r\nlabor and sent Armenians to the Syrian desert in less than humane conditions. Although Turkey still\r\ndenies that an Armenian genocide took place, the issue remains very sensitive for both countries.\r\nWith its other neighbors, Georgia and Iran, Armenia has relatively positive relations. The same is true\r\nfor its near-neighbor Russia, though the strength of the Russian alliance varies (Falkowski, 2016). Despite\r\nthat Iran and Russia are not in the good graces of the West, maintaining strong relations with these\r\ncountries and Georgia is strategic for Armenia, especially since it is dependent on them for energy. By\r\nthe same token, a large portion of Armenia’s wealthy, powerful, and engaged diaspora resides not only in\r\nthese countries but also, and especially, in the West. Thus, Armenia cannot abandon Western\r\nengagement, nor can the West disengage from Armenia. Perhaps due to its tortured geopolitical\r\nsituation, Armenia never clearly articulated NATO aspirations, whereas Georgia and Ukraine (two similar\r\ncases) both did. In terms of foreign policy, Armenia needs to balance Russia, Iran, and the West, which it\r\nhas carefully done.\r\nAll reports suggest that Armenian society is one of the world’s highest levels of social cohesion. 11 These\r\nhigh levels of social cohesion are likely a product of the genocide’s effect on reinforcing group identity,\r\nthe ongoing war with Azerbaijan, a high level of linguistic and religious homogeneity, as well as a very\r\nsmall population and compact geographic territory. These factors bring Armenians together, and likely\r\nyield much—although certainly not perfect—consensus on matters of politics.\r\nTHE REGIONAL CONTEXT: GEORGIA, UKRAINE, MOLDOVA, AND KYRGYZSTAN\r\nThroughout this report, we not only discuss Armenia, but also more general theoretical and empirical\r\nliterature as well as region-specific experiences. In this section, we provide basic background on four\r\nregional comparisons, including why we chose these countries for comparison. In short, each of these\r\ncountries that underwent similar political transitions, and prior to doing so looked reasonably similar to\r\nArmenia prior to April 2018. We detail these backgrounds and comparative figures for each.\r\nTaking into account that no two countries are identical, nor necessarily highly similar, when comparing to\r\nthe broader region, Georgia, Ukraine, Moldova, and Kyrgyzstan are among the best comparison countries. Like\r\nArmenia, these four regional countries underwent political transitions, and have attempted a variety of\r\ngovernance reforms – even if the specific set of governance reforms varied across countries. Below we\r\nprovide some background on these countries from sources including V-Dem and O’Beacha´in and Polese\r\n(2010).\r\n11 Social cohesion refers to the “sticking-togetherness” of a community (Gross and Martin 1952, 553).\r\nUSAID.GOV                                                           GOVERNANCE IN ARMENIA: AN EVIDENCE REVIEW | 5\r\n"                                                                                                                                                                                                                                                                                                                                                                                                                                                                                                                                                                                                                                                                                                                                                                                                                                                                                                                                                                                                       
##  [9] "To first illustrate                 Figure 1. Plotting Regional Democracy and Corruption Comparisons\r\nsimilarities, we plot their\r\ndemocracy and corruption\r\nvalues the year before each\r\nof these countries\r\nunderwent their respective\r\ntransitions. At the broadest\r\npossible level, Figure 1\r\nshows the democracy\r\ncomponents and\r\ncorruption scores based\r\non V-Dem’s 2017 values\r\n(Lindberg et al., 2014)\r\n(Also see full V-Dem\r\nsection at the end of this\r\ndocument).\r\nTo give some sense for\r\nhow the political transition\r\naltered corruption in these\r\nfour comparison countries,\r\nTable 1 shows the corruption levels before and after transition, with Armenia’s current levels also\r\nincluded.\r\nTABLE 1: POLITICAL TRANSITIONS AND CORRUPTION\r\n                       Start Year of     Corruption 5 Years Corruption at   Corruption 2       Corruption 5\r\n                       Political         Before Political   the Year of     Years After        Years After\r\n                       Transition        Transition         Political       Political          Political\r\n                                                            Transition      Transition         Transition\r\nGeorgia                2003              .93                .92             .31                .15\r\nUkraine                2004              .90                .84             .86                .88\r\nKyrgyzstan             2005              .91                .94             .94                .91\r\nMoldova                2009              .79                .75             .69                .72\r\nArmenia                2018              .84                N/A             N/A                N/A\r\nSource: Varieties of Democracy (V-Dem)\r\nGeorgia. In 2003, Georgia experienced its own political transition, in which President Eduard\r\nShevardnadze was ousted from power. Following this, Georgia attempted a series of transitional reforms,\r\nboth domestically and in conjunction with inter- national engagement. Those reforms included a set of\r\nanti-corruption reforms: a national anti-corruption strategy, reform of the Chamber of Control of\r\nGeorgia (CCG), and civil society anti-corruption projects (Di Puppo, 2010). Georgia also reduced\r\nregulatory complexity so as to reduce the availability of rents to corrupt elites (Mungiu-Pippidi, 2016). It\r\nalso worked to digitize government service provision, developed human and institutional capacity, and\r\nUSAID.GOV                                                       GOVERNANCE IN ARMENIA: AN EVIDENCE REVIEW   | 6\r\n"                                                                                                                                                                                                                                                                                                                                                                                                                                                                                                                                                                                                                                                                                                                                                                                                                                                                                                                                                                                                                                                                                                                                                                                                                                                                                                                                                                                                                                                                                                                                                                                                                                                                                                                                                                                                                                                                                                                                                                                                                                                                                                                                                                                                                                                                                                                                                                                                                                                                                                              
## [10] "strengthened strategic planning within target national governance institutions (Schalwayk, 2009; Devlin,\r\n2010; Bennet, 2011).\r\nUkraine. In 2004–2005, Ukraine also experienced its own political transition, which began with a series of\r\nmissteps and scandals surrounding former President Kuchma. By 2004, he did not seek reelection,\r\ncreating a particularly heated competition for the presidency. Kuchma’s chosen successor, Yanukovych,\r\nnarrowly won in a contested, multi-round election, but popular discontent during the electoral process\r\nplaced huge pressure on the national government. During the same period, the Ukrainian Supreme\r\nCourt declared the runoff invalid due to corruption. In a second-round runoff, the results and Yanukovych\r\nlost to the leader of the opposition coalition, Yushchenko, marking the largely peaceful transfer of power.\r\nFollowing its political transition, Ukraine attempted a series of transitional reforms, both domestically and\r\nwith international engagement, but did not enjoy significant success (Copsey, 2010).\r\nKyrgyzstan. In 2004–2005, Kyrgyzstan also experienced a political transition. As part of its political transition,\r\nthe Akayev government lost control, after holding power for 15 years. The impetus was the parliamentary\r\nelections that the opposition, with broad public support, contested as corrupt. The next several years,\r\nin particular, were particularly tumultuous with several rounds of elections and tenuous coalition\r\ngovernments. Following its political transition, Kyrgyzstan attempted a number of reforms, though with\r\nlimited success (Lewis, 2010).\r\nMoldova. In 2009, Moldova also experienced a political transition (Mungiu-Pippidi and Munteanu, 2009).\r\nFollowing the announcement that the Communist Party had won a majority of seats in the parliamentary\r\nelections, demonstrations and riots ensued. At first, President Voronin tried to maintain a hard line, but\r\nultimately could not stabilize the government, leading to snap elections in which a coalition of parties\r\nformed an alliance, and the Communist Party became the opposition. However, Moldova’s political\r\ntransition was not particularly transformative.\r\n  2.2.      CONSOLIDATION OF STATE INSTITUTIONS\r\nCONTEXT/STATUS\r\nBefore turning to specific state institutions, we first illustrate how Armenia fares along a variety of\r\ndemocracy and governance indicators according to V-Dem. Figure 2 displays the overall measure of\r\nelectoral democracy in V-Dem as well as the various the sub-indices that contribute to this overall\r\nmeasure. We plot scores for 1990-2017. For all measures, higher values indicate more democratic\r\noutcomes.\r\nElectoral Democracy—also known as “polyarchy” (Dahl 1973) —intends to record the responsiveness\r\nof rulers to citizens, when this is “achieved through electoral competition for the electorate’s approval\r\nunder circumstances when suffrage is extensive; political and civil society organizations can operate\r\nfreely; elections are clean and not marred by fraud or systematic irregularities; and elections affect the\r\ncomposition of the chief executive of the country” (Coppedge et al., 2018). Overall, electoral\r\ndemocracy was high immediately following Armenia’s independence from the Soviet Union, but it fell\r\nmeaningfully by the mid-1990s. It hit new lows in the mid-2000s although rose again in recent years.\r\nTaking into account margins of error around each point estimate, the major takeaway is that Electoral\r\nDemocracy by 2017 was lower than levels in the early 2000s.\r\nUSAID.GOV                                                     GOVERNANCE IN ARMENIA: AN EVIDENCE REVIEW       |  7\r\n"                                                                                                                                                                                                                                                                                                                                                                                                                                                                                                                                                                                                                                                                                                                                                                                                                                                                                                                                                                                                                                                                                                                                                                                                                                                                                                                                                  
## [11] "In Figure 2, the sub-indices that contribute to the overall electoral democracy score provide further\r\ncontext on the overall trend. Taken together, these variables measure political participation, the\r\nstrength of rule of law and electoral institutions, and the threat of physical violence. First, focus on the\r\nfact that this group of indicators is all around the same level on the overall scale from 0 (worse\r\noutcome) to 1 (better outcome).\r\nThis suggests that the various         Figure 2. Democracy and Governance Measures over Time for Armenia\r\naspects of political and civil life    based on V-Dem\r\nrelevant to democracy in Armenia\r\nare all varying around a low\r\nbeginning baseline. There are\r\nthree notable exceptions. First,\r\nElectoral Contestation was\r\nextremely high at independence,\r\nin the midst of undeveloped\r\npolitical parties, but it dropped\r\nsignificantly by the early 2000s.\r\nSecond, Accountability followed\r\nthe same trend. Accountability\r\ncaptures “constraints on the\r\ngovernment’s use of power\r\nthrough requirements for\r\njustification for its actions and\r\npotential sanctions” (V-Dem\r\nCodebook). This includes\r\naccountability through elections,\r\nchecks and balances between\r\ninstitutions, and oversight by civil\r\nsociety and media. Armenia’s\r\noverall decline in electoral\r\ndemocracy from independence to\r\nthe early 2000s tracks the\r\nregime’s ability to limit these\r\nforms of accountability. The third\r\nexception is Civil Liberties, which\r\nincludes “the absence of physical violence committed by government agents and the absence of\r\nconstraints of private liberties and political liberties by the government” (V-Dem Codebook). Since\r\nindependence, Civil Liberties have been notably higher in Armenia than the other aspects of political life\r\nconsidered here, and Civil Liberties have been generally increasing over time. This increase parallels the\r\nindicator for Less Physical Violence, which indicates a considerably lower threat of physical violence\r\naround 2010. (For definitions of other variables, see V-Dem Codebook.)\r\nUntil recently, and perhaps presently continuing in the wake of Armenia’s political transition, politics in\r\nArmenia has been plagued by pervasive and endemic corruption, clientelism, and low horizontal\r\naccountability (the inability of the bureaucracy to keep checks on itself) (see Paturyan and Stefes, 2017;\r\nIskandaryan, 2012; Transparency International, 2013). Addressing these issues is critical if Armenia hopes to\r\nimprove governance and chart a course towards a more complete democracy. In seeking to address these\r\nUSAID.GOV                                                     GOVERNANCE IN ARMENIA: AN EVIDENCE REVIEW     | 8\r\n"                                                                                                                                                                                                                                                                                                                                                                                                                                                                                                                                                                                                                                                                                                                                                                                                                                                                                                                                                                                                                                                                                                                                                                                                                                                                                                                                                                                                                                                                                                                                                                                                                                                                                                                                                                                                                                                                                                                                                                                                                                                                                                                                                                                                                                                                                                                                      
## [12] "challenges, the consolidation of state institutions is a critical factor. Some institutional arrangements or\r\nchanges have already been made, and we draw attention to how those may promote or hinder better\r\ngovernance. Those include a Proportional Representation (PR) electoral system and a parliamentary\r\ngovernment system. On some of these changes, we are particularly optimistic as the literature suggests\r\nthey are among the better institutional arrangements. We also discuss other institutional factors that\r\nmay be easier to change in the short run, and that may be worth considering.\r\nFACTORS PROMOTING/HINDERING\r\nProportional Representation Electoral system. Armenia employs a PR electoral system for\r\nparliamentary elections with a combination of open and closed list ballots. 12 Seats in parliament are allocated\r\nusing the d’Hondt method, with a threshold of 5% for parties and 7% for multi-party alliances. The electoral\r\nsystem aims to assure that parties that surpass the threshold will be represented in roughly the same\r\nproportion as their nationwide vote totals.\r\nOne important effect of thresholds at the 5% level and above is to mitigate the possibility of an extremely\r\nfragmented party system, which observers are unanimous in decrying. Research shows that democratic\r\nbirth and sustenance relies on a stable political party system, notably including creation and buy-in from\r\nconservative political parties (Ziblatt, 2017). A stable party system similarly contributes to better\r\ngovernance, notably by virtue of a stable party system’s ability to contribute to public service provision and\r\neconomic growth (Hicken, Kollman and Simmons, 2016; Bizzarro et al., 2018). There is no evidence to\r\nsuggest that civil society can replace political parties, reinforcing their importance (Diamond et al., 2014).\r\nMoreover, PR systems tend to foster strong parties, which are usually viewed favorably, although some\r\nanalysts complain that party leaders are insulated from popular pressure (Samuels and Shugart, 2010).\r\nPR systems are generally governed by coalition governments, which, in turn, serve as a check on the\r\nexecutive (since the support of all parties in the coalition is necessary for the coalition’s survival, and\r\nsome are likely to object if the executive oversteps his/her bounds). While this involves some sacrifice of\r\ndecisiveness relative to a Westminster (majoritarian) pattern of governance, and a small possibility of\r\ngridlock, coalition governments have many virtues. Elections are less fraught since the meaning of victory\r\nand defeat is apt to be marginal. (At most, the policies of government will shift slightly in one direction or\r\nthe other, with middle parties providing continuity of personnel and policy.) Cabinets, composed of leaders\r\nfrom all coalition parties, are more inclusive. And the result is a political process that is probably more\r\ncapable than a Westminster system of fostering consensus on major policy initiatives. Coalition governments\r\ndo show a tendency to spend more money. However, this may be a positive characteristic if it helps to\r\nachieve greater consensus on contentious matters of policy (Gerring and Thacker, 2008).\r\nParliamentarism. Armenia’s recent constitutional reforms change it from a semi-presidential system\r\n(Markarov, 2016) to a pure parliamentary system. Research on these topics suggests that this change will\r\nhave the following impacts on politics. First, political parties are likely to be strengthened, and these\r\nparties may operate in a more programmatic (and less clientelistic) fashion. 13 Second, there will be little\r\nconflict between branches of government, so the problem of gridlock and constitutional strife is much\r\nless likely to appear. Third, in between elections there will be few checks on the exercise of power except\r\n12 Closed lists tend foster programmatic, party-centered voting, whereas open lists tend favor a personal connection between\r\nthe candidate and voter. Most analysts argue that open-lists tend to foster more patronage, clientelism, and corruption (Chang\r\nand Golden, 2007; Carey and Shugart, 1995), though there is some disagreement (see Gingerich, 2013).\r\n13 By “programmatic” we mean that the rules for distributive politics (i.e., who gets what and when) will be more\r\npublic and followed more often (Stokes et al., 2013).\r\nUSAID.GOV                                                             GOVERNANCE IN ARMENIA: AN EVIDENCE REVIEW            |   9\r\n"                                                                                                                                                                                                                                                                                                                                                                                                                                                                  
## [13] "those that arise from within the ruling coalition (as noted). So long as the ruling party or coalition\r\nmaintains its majority in parliament, there is little to stand in its way until the next election. This may\r\nseem like a recipe for unaccountable government. However, experience with parliamentary systems\r\nsuggests that electoral incentives are usually sufficient to keep rulers in line with the majority opinion\r\n(Gerring and Thacker 2008; Samuels and Shugart 2010). Fourth, parliamentary systems are more supple\r\nwhen it comes to replacing leaders, as the prime minister and members of the cabinet can be replaced at\r\nany time, and in a fully constitutional fashion, without jeopardizing the ruling government and coalition (so\r\nlong as a majority is maintained). For this reason, constitutional crises are much rarer in parliamentary\r\ndemocracies than in presidential democracies. Fifth, bureaucrats are more accountable to elected leaders\r\nbecause there is only one signal emanating from government, rather than two, and this means that\r\nunelected officials have less leeway to set policy or to resist government initiatives. Sixth, because\r\ngovernment speaks with one voice and can keep bureaucrats accountable, there is no need to saddle\r\nbureaucrats with “red tape,” constraining their room for maneuver with statutory regulations that often\r\nturn out to be rather inefficient. Finally, there is some evidence to suggest that parliamentary systems are\r\nless corrupt (Gerring and Thacker, 2004). All in all, political scientists tend to think highly of parliamentary\r\nsystems (Fish, 2006; Gerring and Thacker, 2008; Linz, 1990; Linz and Valenzuela, 1994).\r\nDemocratic Transition. Armenia is undergoing a political transition toward more democratic practices.\r\nThe long-term effects of increased electoral competition and government turnover are likely to be\r\nfavorable for reform processes. The short-term effects are harder to gauge. Many have argued that\r\ncorruption, as one example, increases in the wake of a democratic transition and then gradually\r\ndecreases thereafter (McMann et al., 2018). However, using data from countries within the post-Soviet\r\nregion, we find that political transitions are followed by lower corruption.12 Of course, increased\r\ntransparency may make corruption more visible after a democratic opening. In any case, we emphasize that\r\neven if there is a reduction in corruption attendant upon Armenia’s political transition, we cannot expect\r\nthis to come about automatically: corruption does not improve as a mechanistic function of multi-party\r\nelections. It must be fought for.\r\nJudicial Reform. The judiciary is a critical actor in establishing rule of law, securing civil liberty, and\r\nconstraining governments to abide by the constitution (O’Donnell, 2004). To further this goal, the\r\njudiciary must have the ability to manage itself (while being subject to anti-corruption laws and potential\r\nimpeachment), including an independent budget, life tenure for judges (in good behavior), and the ability\r\nto judge the constitutionality of laws and non-statutory executive actions. Recent accounts suggest that\r\nthe judiciary in Armenia is relatively weak (Paturyan and Stefes, 2017), and most likely the judicial system\r\nneeds significant reform. With that said, we refer readers to the Integrity Systems / Rule of Law Evidence\r\nReview conducted separately under this tasking (USAID, 2019b).\r\nSuccessful Prosecution, Regulation, and Privatization. Some analysts argue, though it is highly\r\ndisputed, that establishing the rule of law constitutes a necessary first step to promote democratic\r\nreform (O’Donnell and Schmitter, 1986; Diamond et al., 2014; Mungiu-Pippidi, 2015, 2016). Regardless of\r\nwhere analysts side on the sequencing debate, simply having numerous laws, combined with popular\r\nmobilization, is unlikely to change behavior. Instead, one critical first step toward achieving successful\r\nprosecution—reducing regulatory complexity—may be critical to success. Privatization of state-owned\r\nenterprises may play a role toward establishing successful regulation under some circumstances, but\r\nadvocating for such a strategy is unequivocally not the right course for aid agencies due to the failure of\r\nthe Washington Consensus (Williamson, 1993, 2000; Rodrik, 2006).\r\nUSAID.GOV                                                    GOVERNANCE IN ARMENIA: AN EVIDENCE REVIEW      |  10\r\n"                                                                                                                                                                                                                                                                                                                                                                                                                                                                                                                    
## [14] "Natural Resources. Natural resource rents generally pose a significant obstacle to the consolidation of\r\nstate institutions in less consolidated democracies (Menaldo, 2016). For example, Mexico’s state-owned\r\noil company, PEMEX, long served as a source of resources for the once dominant PRI political party to\r\nfacilitate vote-buying and the exchange of patronage of jobs (Greene, 2007). Since Armenia does not\r\nhave an extensive store of valuable minerals such as oil, gas, and metals (Iskandaryan, 2012), Armenia is\r\nnot susceptible to the so-called resource curse. When countries do not have strong institutions, resource\r\nwealth can fuel authoritarianism, civil war, and exchange rate woes (Dutch Disease), thereby hindering\r\ngood governance and the development of democracy (Ross, 2015; van der Ploeg, 2011).\r\nREFORM/INTERVENTION EXAMPLES\r\nReforming Institutions. As we describe throughout this Governance Evidence Review, Armenia has an\r\noverall institutional setting that is more based on personal relations and elite-level divisions of rents 14\r\nthan the unbiased, impersonal application of the rule of law. 15 In such a setting, mere replication of\r\nWestern institutions in a manner that is devoid of country context is unlikely to yield good outcomes\r\n(Scott, 1998; Pritchett and Woolcock, 2004; Andrews, Pritchett and Woolcock, 2017). Institutional\r\nreform is more likely to succeed in such a setting when aid agencies and policymakers identify (a) the\r\nleaders with sufficient clout who can serve as change agents (Grindle, 2012); (b) the specific institutions\r\nthat are amenable to change; and (c) the correct time horizon or “policy window” to enact the reform\r\n(Kingdon, 1995; Levy, 2014).\r\nFor many reasons, an essential way to start any potential institutional reform is by conducting formal\r\npolitical economy analysis—also known as stakeholder analysis. One reason is that, as the World Bank’s\r\nexperience with policy/structural adjustment reforms shows, success with institutional reform is mostly\r\na function of political economy considerations (Dollar and Svensson, 2000). Second, especially in weaker\r\ninstitutional settings, difficult-to-change social norms are often stronger than any imposition of a new\r\nrule, law, or monitoring program, regardless of the strength of its enforcement mechanism (Acemoglu\r\nand Jackson, 2017; World Bank, 2008; Dizon-Ross, Dupas, and Robinson, 2017; Dhaliwal and Hanna,\r\n2017). Third, actors outside the scope of the potential institution under consideration for reform, such\r\nas political parties, often derail institutional reform (Cruz and Keefer, 2015).\r\nA well-done political economy analysis delves well beyond identifying the winners, key players, and\r\ninstitutional arrangements, as well as all of their capacities to implement the reform. Other salient\r\naspects to examine notably include political parties, “interests, incentives, rents/rent distribution,\r\nhistorical legacies, prior experiences with reforms, social trends, and how all of these factors effect or\r\nimpede change” (World Bank, 2011: 1; Cruz and Keefer, 2015). Perhaps most importantly, though, a\r\nwell-done political economy analysis must identify the losers from the proposed reform as well as their\r\nwillingness and ability to block it. If the winners appear weak and/or the losers appear strong, then the\r\nreform will likely not succeed. Similarly, any institutional reform requires capable actors and a\r\n14 By “rents”, we mean “returns [or assets] that exceed the opportunity cost of resources that might otherwise be deployed in\r\na competitive market” (Levy, 2014: 22). In more competitive markets and political environments, competition fosters “creative\r\ndestruction” of monopolists’ rents (Levy, 2014: 23).\r\n15 Acemoglu and Robinson (2012) would refer to Armenia’s institutional setting as an “extractive institution.” North, Wallis,\r\nand Weingast (2009) would use the term “limited access social order.” Fukuyama (2011, 2014) would use the term\r\n“neopatrimonialist”. Although each of these frameworks are slightly different—and their scopes eclipse that of this study—all of\r\nthe frameworks converge on the idea of a regime not having impersonal application of the rule of law, as originally conceived by\r\nMax Weber (1978).\r\nUSAID.GOV                                                          GOVERNANCE IN ARMENIA: AN EVIDENCE REVIEW               |  11\r\n"                                                                                                                                                                                                                                                                                                                                                                                                                                                                                                                                                          
## [15] "sustainable funding source that is not subject to elite capture. If either the latter or the former is missing\r\nor weak, it may be more advantageous to pursue another reform.\r\nIdentifying promising institutional reforms can be challenging, but there are strategies that aid agencies\r\nand policymakers can employ. One prominent strategy relies on the identification of “weak links” or\r\n“binding constraints”: that is, the constraints that can unlock potential gains in many other areas (Rodrik,\r\n2007; Hausman, Rodrik, Velasco, 2008; World Bank, 2012). However, there are some notable\r\nchallenges to the binding constraint framework. First, its primary domain of application is that of\r\neconomic growth, for which it works best applies when there is a dense, overlapping network\r\ninstitutions to support it—something that may not exist in a low institutional environmental (Hidalgo,\r\nKlinger, Barbasi, and Hausmann, 2007). Second, it may not be feasible to tackle the most pressing,\r\nbinding institutional constraint. Bureaucrats’ behavior or logistical challenges may be too difficult to\r\novercome, for example (World Bank, 2012). When policymakers fail to achieve reforms in areas of\r\ncrucial importance to citizens, it may also yield the side effect of increasing cynicism about the policy\r\nprocess (So et al., 2018). In turn, that cynicism can create political cleavages for populist outsiders, who\r\ntend to be more authoritarian and have less respect for democratic norms (Levitsky and Ziblatt, 2018).\r\nAnother well-known strategy of institutional reform is that of New Public Management. Its basic premise\r\nis to cut costs, deregulate, reduce inefficiencies, induce more participation from lower levels of\r\ngovernment and clients, and offer alternative service provision options (Peters, 2008). Essentially, even\r\nthough the exact definition is contested, NPM is about reducing red tape and attenuating the monopoly\r\npower of government. Although NPM was very popular during the1990s and part of the 2000s, it—and\r\nthe Washington Consensus that provided a platform for it—has since fallen mostly out of favor. Beyond\r\nthe legitimacy problems that the Washington Consensus created for aid agencies (Rodrik, 2006), NPM\r\nnecessarily entails creating a principal-agent problem (Peters and Pierre 2018: 140-145). More\r\nspecifically, NPM divorces the policy decision-making process from the implementation of programs.\r\nWhile that may produce good results in some contexts, it makes discerning who is responsible when\r\nthings go wrong more difficult. Most pertinently for this Governance Evidence Review, an edited volume\r\nfrom the eminent public administration scholar Peters (2008) suggests that NPM was less successful in\r\nRussia and the former Soviet Republics, including Armenia.\r\nNowadays, the two most useful frameworks to guide institutional reforms are the Problem-Driven\r\nIterative Adaptation (PDIA) approach of Andrews, Pritchett, and Woolcock (2013) and the political\r\nsettlements framework of Khan (2017). The latter is especially useful for institutional reform feasibility\r\nanalysis because, unlike New Institutional Economics, 16 the political settlements frameworks treats\r\npower and institutions separately (Behuria, Buur, and Gray, 2017). The political settlements framework\r\nthus allows for dynamic investigation of how power shapes institutional configurations over time as well\r\nas during critical junctures (see Capoccia and Kelemen, 2007; Capoccia, 2016).17 The three dimensions\r\nof political settlements also lend quite smoothly to the analysis of institutional reform: horizontal power\r\n(power of excluded groups relative to dominant ones); vertical power (power of dominant factions in\r\n16 New Institutional Economics (NIE) essentially stresses how political and economic institutions shape economic and political\r\nbehavior and outcomes (e.g., Acemoglu, Johnson, and Robinson, 2005). NIE thought has produced some of the most famous\r\nwork in the whole discipline of economics, including from Nobel Laureates such as Elinor Ostrom, Oliver Williamson, Ronald\r\nCoase, and Douglas North. For representative works, see Ostrom (1990), Williamson (1985), Coase (1960), and North (1981).\r\n17 Within the historical institutionalist tradition, critical junctures refer to periods of very rapid change that later creates sticky,\r\nhard-to-break path dependence. There is no specific minimum or maximum length to these critical junctures; they can last for\r\nmany years or a very short amount of time. For more on historical institutionalism, see, for example, Capoccia and Kelemen\r\n(2007).\r\nUSAID.GOV                                                                 GOVERNANCE IN ARMENIA: AN EVIDENCE REVIEW                |  12\r\n"                                                                                                                                                                                               
## [16] "the ruling coalition relative to other less powerful ones also in the ruling coalition); and political\r\nsettlement financing (i.e., notably, between elites inside and outside the ruling coalition) (Behuria, Buur,\r\nand Gray, 2017). Each of these dimensions gets at the heart of Armenia’s challenges with its captured\r\nbureaucracy (Paturyan and Steffes, 2017).\r\nThe Problem-Driven Iterative Adaptation (PDIA) approach of Andrews, Pritchett, and Woolcock (2013)\r\nis a very strong response to what Andrews (2015) calls Solution- and Leader-Driven Change (SLDC).\r\nWhereas SLDC is about identifying solutions to institutional reform up and having the solutions guide\r\ninterventions, PDIA is about focusing on the problems and “muddling through…with experimentation\r\nand trial and error” (Andrews, 2015: 197).18 The rationale for the change of focus, according to\r\nAndrews, Pritchett, and Woolcock (2017), is that states have very different capacities to implement\r\ninstitutional reforms. Those capacities tend to be very context- and sector-specific, too.\r\nIn summary, institutional reform is not about identifying a solution ahead of time, such as streamlining to\r\navoid duplication or revising mandates. Before proceeding with institutional reform, aid agencies and\r\npolicymakers need to undertake political economy analysis. When doing so, they should not only\r\nexamine the traditional tenets of political economy analysis but the three dimensions of the political\r\nsettlements framework, paying particular attention to implementation capacity constraints. The actual\r\nimplementation of reforms should be gradual and involve the engagement of all relevant actors, too.\r\nQuick, rapid, and drastic institutional reforms have a tendency to cause chaos and backfire, particularly\r\nin weaker institutional environments such as Armenia (Bersch. 2016, 2019; Acemoglu and Jackson,\r\n2017).\r\nAnti-Corruption Agencies. Anti-corruption agencies often play a key role in governance reform\r\n(Recanatini, 2011; Søreide, 2014). However, we caution that their impact is contingent upon several\r\nconditions that are difficult to achieve: political independence, adequate funding and staffing, investigatory\r\npower, hotlines allowing for anonymous reporting of corrupt activities, and an amnesty for whistle-blowers.\r\nCountry-level experiences are generally mixed, with some standout success cases (Hong Kong, Singapore),\r\nand lots of cases that are hard to classify (Dixit, 2018). For more on anti-corruption agencies, we refer\r\nreaders to the Integrity Systems and Rule of Law Evidence Review also under this tasking (USAID,\r\n2019b).\r\nDecentralization. While we expect that addressing low-level petty corruption and e-governance are\r\nimportant opportunities for Armenia to pursue, there are risks in how Armenia proceeds. We offer the\r\nfollowing with respect to decentralization.\r\nIn theory, decentralization offers the promise of bringing government closer to citizens. In particular, it\r\nallows citizens more potential to request their leaders for the specific mix of public goods/services that\r\nthey need at the local level. By the same token, the second- generation of fiscal federalism/decentralization\r\nliterature stresses the difficulty of achieving the promise of decentralization (Oates, 2005). Political economy\r\n(incentives) and corruption challenges are difficult to overcome, and the proper design of decentralization\r\nrequires many overlapping government features and structures to work seamlessly in tandem (Weingast,\r\n2014; Mookherjee, 2015; Rodden, 2004).\r\n18 Although Andrews, Pritchett, and Woolcock (2013, 2017) do not reference Easterly (2006) directly, the contrast between\r\nPDIA and SLDC seems to mirror’s Easterly’s (2006) call for “searching” for solutions at a small-scale, as opposed to the\r\napproach of Sachs (2005) and the Millennium Development Goals (MDGs), which involve large-scale “planning”.\r\nUSAID.GOV                                                         GOVERNANCE IN ARMENIA: AN EVIDENCE REVIEW              | 13\r\n"                                                                                                                                                                                                                                                                                                                                                                                                                                                                                                                                                                                                                                                                                                                                                                                                                                                                                                                                          
## [17] "There is no one-size-fits-all form of decentralization that suits all countries. For decentralization to be\r\nsuccessful, though, there must be a credible commitment from the national government entailing: (i)\r\nsufficient resources to assist with the processes, hire new people, and create new facilitating\r\nstructures/institutions, not mere devolution of responsibility to the subnational level; (ii) clear procedures\r\nand legislation, especially concerning taxes; (iii) a fair intergovernmental transfer system, based on clear\r\ncriteria such as population thresholds and poverty or other needs-based measures; and (iv) audits, a free\r\npress, access to information, and informed voters to hold politicians and the bureaucracy accountable (Díaz-\r\nCayeros, 2006; Bird and Smart, 2002; Lessmann and Markwardt, 2010; Boffa, Piolatto and Ponzetto, 2016;\r\nFerraz and Finan, 2008).\r\nEvidence from young democracies similar to Armenia suggests that sufficient implementation on all of the\r\nabove criteria is very challenging. With respect to resources, fiscal imbalances often imperil the\r\nimplementation of decentralization programs. Since decentralization funding decisions are largely made at the\r\nnational level, decentralized decision-makers largely have no recourse when national-level politicians do not\r\nfund the programs (Bird and Smart, 2002; Díaz-Cayeros, 2006). For example, the Honduran Law of\r\nMunicipalities states that the national government is to transfer 11% of national revenue to municipalities.\r\nHowever, the national government frequently only transfers 8% of revenue due to fiscal imbalances, and\r\nthe decentralized decision-makers are powerless to change the situation.\r\nAs decentralization brings more resources to subnational governments, it yields a question concerning\r\nhow they will spend the resources. On this score, the global literature highlights the effect of political\r\nbudget cycles, detailing precisely how politicians manipulate intergovernmental transfers and other\r\nresources from the central government to alter the outcomes of subnational elections (Klomp and de\r\nHaan, 2013). 19 Notably, discretionary expenses on employment, grant programs (e.g., conditional cash\r\ntransfers), and infrastructure tend to increase during electoral cycles (Remmer, 2007; Keefer and Khemani,\r\n2005).\r\nBeyond elections, building state capacity to monitor all of the additional government functions that\r\ndecentralization creates is difficult, particularly with respect to taxation. Capacity to tax is generally very\r\nuneven across most states’ territories (Besley and Persson, 2013; Kiser and Karceski, 2017). Performance\r\npay for tax collectors, even if effective from a tax revenue perspective, also might bring about costs such\r\nas increased bribery (Khan, Khwaja and Olken, 2016).\r\nThere are several reasons to be skeptical of the potential benefits of decentralization in Armenia. First,\r\nArmenia is a very small country, both demographically and geographically, making the government already\r\nclose and presumably accessible to the people. Second, Armenia is a highly homogeneous country, so\r\ncitizens living in different regions do not experience linguistic or cultural barriers and probably do have\r\nradically divergent policy preferences (e.g., for taxes and spending). As such, little improvement in the\r\nmatch of preferences to policy can be expected from decentralization (Rodden, 2004). Third,\r\ndecentralization in a small country means that regional and/or local governments that gain power will have a\r\nvery narrow tax base, making it difficult to raise revenue and increasing the likelihood of capital flight (Cai\r\nand Treisman, 2004). Fourth, very small units will lose economies of scale, meaning that some government\r\nservices may be more expensive, or may be under-provided (Treisman, 2007; Blom-Hansen et al, 2016).\r\nFifth, insofar as regulatory authority is devolved to local or regional authorities it may create a confusing\r\n19 See evidence from, for example, Brazil (Brollo et al., 2013), Mexico (Timmons and Garfias, 2015), Portugal (Veiga and\r\nVeiga, 2013), Spain (Sole´-Olle´ and Sorribas-Navarro, 2008), and Ghana (Banful, 2011).\r\nUSAID.GOV                                                          GOVERNANCE IN ARMENIA: AN EVIDENCE REVIEW          |  14\r\n"                                                                                                                                                                                                                                                                                                                                                                                                                                                                                                                                                                                                                                                             
## [18] "patchwork of laws—discouraging investment, erecting a barrier to the free movement of goods and\r\nservices, and ultimately raising prices. Sixth, small and weak local governments are more susceptible to\r\n“capture” by powerful oligarchs or businesses, who exercise power in a region (Bardhan and Mookherjee,\r\n2000; Bardhan, 2002). Seventh, decentralization is likely to increase regional inequalities across the country\r\n(Bardhan and Mookherjee, 2006). Eighth, the search for competent elected leaders, bureaucrats, judges,\r\nNGO staff, and all the people required to make a government work well is more difficult at local and regional\r\nlevels: human capital tends to be scarcer on the periphery, and local/regional wages are generally not as\r\nattractive as those in the capital (where the central government is headquartered). So, decentralization\r\ngenerally means that the quality of government personnel declines. Ninth, decentralization makes\r\ngovernment more fragmented and thus more complicated to operate and to monitor, weakening\r\naccountability mechanisms between leaders and electors (Boffa, Piolata, and Ponzetto, 2016).\r\nWhile the literature suggests pessimism about decentralization, there are some arguments suggesting\r\nthat decentralization may not be so problematic, though only under certain conditions. First, the policy\r\nto be decentralized should be clearly delineated, avoiding confusion and potential conflicts between the\r\nnational government and local governments over who is responsible for the policy, and allowing citizens\r\nto clearly assign blame if the policy is not successfully implemented.\r\nSecond, the policy should be funded by revenue that is raised locally. In this fashion, the locality is not\r\ndependent upon the vicissitudes of the national budget and of national politics. Too often, funding for\r\nlocal programs is cut by national governments eager (or compelled) to balance their budgets. Or, it is\r\nallocated in a partisan manner in order to pay off clients of the ruling party. Note, however, that raising\r\nrevenue at local levels is difficult as capital is mobile. A subnational government that raises the tax rate\r\non citizens or businesses is liable to instigate capital flight, with the result that it will raise less revenue\r\nthan intended and may damage its economy and its tax base.\r\nThird, subnational governments must have the revenue and administrative capacity to implement the\r\npolicy. Note that subnational governments are generally less attractive to well-trained bureaucrats and\r\nspecialists, who prefer higher-paying jobs located in the capital. Some subnational governments are apt\r\nto be cash-strapped, probably because they are situated in a poor region and thus have a weak tax base.\r\nThey may struggle to raise the necessary funds to implement the policy, and even if they have the\r\nfunding they may not have a sufficiently large and well-trained staff to implement that policy.\r\nFourth, subnational governments must be subject to hard-budget constraints, enshrined in statutory or\r\nconstitutional law. Absent this constraint, subnational governments will be free to engage in deficit-\r\nspending, with the knowledge that the national government will bail them out at a later date. Over time,\r\ndeficits will take a serious toll on the national economy and the national budget (Rodden, 2002).\r\nIf these four conditions are in place, we may expect that a decentralized policy will be successful. That\r\nsuccess will be augmented if, in addition, preferences vary on that policy and these policy preferences\r\nare regionally specific. That is, those who prefer Policy A live together in Region A, while those who\r\nprefer Policy B live together in Region B. This way, everyone’s preferences are maximized and they do\r\nnot have to suffer under a uniform policy imposed by the national government.\r\nWhile decentralization is a common and increasingly attractive policy in many countries, the scholarship\r\noffers mixed expectations with the preponderance of the evidence suggesting pessimism about the\r\nbenefits of decentralization. In a geographically and politically concentrated country such as Armenia, for\r\nUSAID.GOV                                                     GOVERNANCE IN ARMENIA: AN EVIDENCE REVIEW       |  15\r\n"                                                                                                                                                                                                                                                                                                                                                                                                                                                                                                                                                                                                                                                                                     
## [19] "example, it is somewhat unclear what the benefits of decentralization would be. With that said, under\r\ncertain conditions decentralization could be useful, and we urge taking seriously these conditions.\r\nAmnesty and Transitional Justice. Even though the causal evidence regarding the effectiveness of\r\ntransitional justice programs is not conclusive (McCargo, 2015), granting amnesty in exchange for the\r\nconfession of minor crimes (e.g., petty corruption) committed under the old regime might serve as a\r\nway to coopt opposition and to move forward without wholesale lustration (O’Donnell and Schmitter,\r\n1986). After implementing the program, Armenia could then introduce set an example by placing some in\r\njail, levying large fines on others, etc. By the same token, threats of prosecution should not become a\r\ntactic for blackmailing the opposition or putting opponents in jail.\r\nCampaign Finance. Most countries regulate political finance and many offer public subsidies to political\r\nparties or candidates. Proponents of political finance regulations claim that public money reduces\r\ncorruption in politics, and some evidence seems to support this thesis. Political finance subsidies and\r\naccompanying regulations may reduce the influence of private money in politics and increase legal and media\r\nsanctions for corrupt behavior (Hummel, Gerring and Burt, 2018).\r\nLEARNING FROM REGIONAL/SIMILAR COUNTRY EXPERIENCE\r\nGeorgia and Estonia achieved some governance improvements in part by reducing regulatory complexity. In\r\nthe process, elites lost access to public sector revenue streams that they were capturing for their own\r\npersonal gain, which hurt the integrity of the public sector. Part of the reason for the governance\r\nsuccesses of these regimes included their “big-bang” approach: that is, to undertake the reforms rather\r\nquickly (Mungiu-Pippidi, 2016; Rothstein, 2011). Powering through a big-bang approach, however, yields\r\nrisks for political stability in weak institutional environments (Bersch 2016, 2019). Therefore, it is\r\ndifficult to know whether a “big- bang” approach toward reducing regulatory complexity could yield\r\nfruitful results for Armenia.\r\nUkraine’s 2004-2005 political transition, Kyrgyzstan’s 2005 transition, and Moldova’s transition of 2009\r\ngenerated lots of protest and subsequent hope from citizens about potential institutional change (Ó\r\nBeacháin and Polese, 2010). Although Ukraine succeeded at creating an opposition, its political transition\r\nmostly replaced one elite with another, which did not promote many institutional reforms (Copsey,\r\n2010). Similarly, Kyrgyzstan’s political transition neither rid the country of its elite, nor did it not\r\npromote fundamental change (Lewis, 2010). For its part, Moldova has not diminished the influence of its\r\noligarchs/“mafia”, has not fulfilled citizen requests for a fairer electoral system, and still does not have a\r\nfunctioning opposition party (Nemtsova, 2016; Munteanu, 2018). Overall, the Ukrainian, Kyrgyz, and\r\nMoldovan experiences suggest that Armenia needs to be careful to properly manage expectations from\r\nits political transition; otherwise, disappointment could lead to institutional retrogression or decay.\r\nTAKEAWAYS\r\nIt is our belief that Armenia is in a relatively good position institutionally, and there is cause for optimism.\r\nMuch research suggests that parliamentary governments with proportional representation electoral systems\r\n— and further with the 5% thresholds to prevent party fragmentation — are the most prone to healthy\r\ngovernance and democratic development. Moreover, Armenia’s population, especially the youth, are heavily\r\ninvolved and eager to have better governance.\r\nGiven these positive institutional and societal situations, Armenia is poised to make good progress if it\r\ncan effectively and wisely approach corruption, develop their bureaucratic functions such as e-governance, and\r\nUSAID.GOV                                                     GOVERNANCE IN ARMENIA: AN EVIDENCE REVIEW        | 16\r\n"                                                                                                                                                                                                                                                                                                                                                                                                                                                                                                                                                                                                                                                                                                                                                                                                                                                                                                                                      
## [20] "approach institutional decentralization cautiously. Indeed, for the many reasons discussed above, we are\r\ndubious of the idea that decentralization in Armenia would enhance the quality of democracy or\r\ngovernance. Armenia might be better served by maintaining its unitary system of government.\r\nThe Armenian government and international actors need to carefully set an agenda for progress that is both\r\nambitious, but cognizant of political realities. This will involve instituting measured changes, including\r\nappropriately and cautiously dealing with the most corrupt and most influential players in Armenia, which\r\nmay require a longer time-horizon, but should yield steady progress toward establishing a foundation for\r\nmedium to long-term success.\r\n  2.3.       SERVICE DELIVERY\r\nCONTEXT/STATUS\r\nPublic services include “goods funded and/or directly provided by the state to improve the welfare of citizens”\r\n(Lieberman, 2018). Examples of public services include water, sanitation, electricity, environmental protection,\r\nand education—essentially, state-led services that individual citizens “consume”, in the economic sense of the\r\nword. Although many citizens of advanced democracies receive government benefits such pensions and social\r\ninsurance, these are not public services. All public services are also not necessarily public goods, which must\r\nbe nonrival (one person’s consumption does not affect availability for others) and nonexclusive (impossible to\r\nprovide without making available to mostly everyone) (Pyndick and Rubenfeld, 2009). Given the significant\r\ndisparities in provision of public services both across and within states, most public services classify as club\r\ngoods (low rivalry, high excludability) or private goods (high rivalry, high excludability. 20 Armenia notably\r\nsuffers from corruption and in differential access to government services in rural areas (World Bank, 2015;\r\nWorld Bank, 2018).\r\nMeasures of Service Delivery. There do not exist unique measures of service delivery across sectors. The\r\nclosest such measure is the World Bank’s Worldwide Governance Indicator’s Government Effectiveness score\r\n(Kaufman, Kraay, and Mastruzzi, 2018). The Government Effectiveness variable measures a variety of factors,\r\nincluding information about service delivery as well as the civil service and bureaucracy more generally. As\r\nthis information may be relevant, Figures 3 and 4 offers some background on government effectiveness in\r\nArmenia relative to other measures of governance, as well as compared to other countries in the region.\r\nFigure 3 shows Government Effectiveness, along with three other measures: Regulatory Quality, Rule of Law,\r\nand Control of Corruption. Each of these additional measures contains some useful information about service\r\ndelivery and civil service reform, but only indirectly. Figure 4 shows just the Government Effectiveness\r\nmeasure but illustrates Armenia relative to the regional comparisons. In both cases, the figures plot global\r\nrankings for Armenia and its regional comparison countries.\r\n20 Rarely do public services fall within the rubric of common resources (high rivalry, low excludability). For a discussion of\r\ncommon resources, see Ostrom (1990).\r\nUSAID.GOV                                                           GOVERNANCE IN ARMENIA: AN EVIDENCE REVIEW                  | 17\r\n"                                                                                                                                                                                                                                                                                                                                                                                                                                                                                                                                                                                                                                                                                                                                                                                                                                                                                                                                                                                                                                                                                                                                                                                                                                                                                                                                                                                                                                                                                                                                                                                                                                                   
## [21] "Figure 4.\r\n   Figure 3. Armenia’s Global Ranking on Various Governance Measures\r\nUSAID.GOV                                            GOVERNANCE IN ARMENIA: AN EVIDENCE REVIEW | 18\r\n"                                                                                                                                                                                                                                                                                                                                                                                                                                                                                                                                                                                                                                                                                                                                                                                                                                                                                                                                                                                                                                                                                                                                                                                                                                                                                                                                                                                                                                                                                                                                                                                                                                                                                                                                                                                                                                                                                                                                                                                                                                                                                                                                                                                                                                                                                                                                                                                                                                                                                                                                                                                                                                                                                                                                                                                                                                                                                                                                                                                                                                                                                                                                                                                                                                                                                                                                                                                                                                                                                                                                                                                                                                                                                                                                                                                                                                                                                                                                                                                                                                                                                                                                                                                                                                                                                                                                                                                                                                                                                                                                                                                                                                                                                                                                                                                                                                                                                                                                                                                                                                     
## [22] "FACTORS PROMOTING/HINDERING\r\nEthnicity and Social Cohesiveness. Numerous studies have shown that ethnic tensions are one of the\r\nmost consistent and significant impediments to public service delivery (Alesina, Baqir, and Easterly, 1999;\r\nMiguel and Gugerty, 2005; Habyarimana et al., 2007). Although the measurement of ethnic heterogeneity or\r\n“fractionalization” 21 was an issue in earlier studies (Selway, 2011), later studies confirm ethnicity that drives\r\npublic service provision (Burgess et al., 2015; Alesina, Michalopoulos, and Papaioannou 2016). Since Armenia\r\nis a mono-ethnic country with essentially one religion and is very societally cohesive, it bodes well for the\r\ncountry’s ability to overcome public service delivery challenges.\r\nPoverty and State Capacity. Poverty and insufficient state capacity are clearly determinants of service\r\ndelivery. Armenia is a middle-income country and seems to be at level of development in which it would not\r\nbe subject to “poverty traps”: that is, at levels of poverty such that investment in services would have no\r\neffect, due to other countervailing factors (Sachs, 2005; Banerjee and Duflo, 2012). That said, if Armenia\r\ndecides to go forth with decentralization, it could cause further complications in terms of the state’s capacity\r\nto deliver services (Besley and Person, 2011; Mookerjee, 2015).\r\nDemocracy. Democracies have larger ruling coalitions than autocracies (Bueno de Mesquita et al., 2003;\r\nSvolik, 2012). Because democracies have to answer to more constituents (Manin, Przeworksi, and\r\nStokes, 1999), and patronage is expensive (Gingerich, 2013; Robbinson and Verdier, 2013; Lizzieri and\r\nPersico, 2004), democracies provide more public services to their citizens than autocracies. There is\r\nsupporting evidence from sectors such as health, education, environmental protection, nutrition, road\r\ninfrastructure, and electricity (Lake and Baum, 2001; Bueno de Mesquita et al., 2003; Bernauer and\r\nKoubi, 2009; Burgess et al., 2015; Blaydes and Kayser, 2011; Min, 2015; Lizzeri and Persico, 2004; Cao\r\nand Ward, 2015; Harding and Stasavage, 2014; Besley and Kudamatsu, 2006; Kudamatsu, 2012). As\r\nArmenia continues to democratize, it should continue to improve its provision of public services.\r\nUrban Bias and Consequences for Rural Hinterlands. There is consensus in the literature that, as\r\ncompared to urban areas, rural ones generally suffer from underprovision of public services (Brinkerhoff,\r\nWetterberg, and Wibbels, 2018; Kosec and Wantchekon, 2019). Part of the reason has to do with the so-\r\ncalled “streetlight effect”: the fact that it is easier to provide services to urban areas that are both\r\naccessible by road and have larger populations from which services providers can draw for their labor\r\npools.\r\nUrban bias is particularly a concern in states that are more authoritarian. Authoritarian regimes\r\nunderprovide public services to rural areas because, generally, it is harder for rural citizens to generate\r\nthe collective action power necessary to threaten the stability of the regime, which is the primary\r\nconcern of authoritarian rulers. By contrast, urban residents can mobilize collective action much faster,\r\nand urban collective action can quickly threaten the stability of an authoritarian regime. Consequently,\r\nauthoritarian regimes pay much more attention to urban needs and service delivery than rural ones, and\r\ntend to have higher concentrations of its citizens that live in urban areas (Bates 1981; Ades and Glaeser,\r\n1995; Pierskalla 2016; Ballard-Rosa, 2016; Kim and Urpelainen 2016). Rural-urban divides in public\r\nservice provision are generally less salient in democratic regimes. Some democracies have\r\n21 The concept of fractionalization “measures the likelihood that if two persons were selected at random, they would be from\r\nthe same ethnic group” (Bates, 2017: 57). Soviet ethnographers carried out the first analysis of fractionalization across the\r\nworld during the 1960s (see Mauro 1995), and researchers used the Soviet measurement until Alesina et al (2003) came out\r\nwith a more comprehensive measure.\r\nUSAID.GOV                                                             GOVERNANCE IN ARMENIA: AN EVIDENCE REVIEW               | 19\r\n"                                                                                                                                                                                                                                                                                                                                                                                                                                                                                                                                                                                                                                                                                                          
## [23] "malapportionement 22 in their electoral systems that accord more electoral weight to rural areas (Bayer\r\nand Urpelainen, 2016), and variation of district magnitude can produce similar effects in proportional\r\nrepresentation systems (Monroe and Rose, 2002). Additionally, many democracies belong to clubs such\r\nas the European Union, ASEAN, and Mercosur that finance public service provision in rural areas for\r\nmore needy member states. Accordingly, Armenia’s political transition and its move toward democracy\r\nshould enable the country to improve existing rural service delivery challenges.\r\nHistorical State Presence and Elite History. The presence of a pre-colonial historical state is one of\r\nthe most consistent predictors of current day economic development (Michalopoulos and Papaioannou\r\n2013; Donaldson and Storeygard 2016; Pierskalla, Schultz, and Wibbels, 2017). Since these studies proxy\r\nfor economic development through satellite-generated measures of present-day nighttime electricity, we\r\ncan be sure that the presence of a historical state contributes to public service provision as well. By the\r\nsame token, the historical presence of elites in an area can lead to less service provision (Pandey, 2010).\r\nAt the moment, it is difficult to disentangle the effect of the historical state on effect in Armenia, which\r\nhas a long history and underwent colonization from Ottoman Empire (Michalopoulos, Naghavi, and\r\nPrarolo, 2016; Derlugian and Hovhannisyan, 2018).\r\nTransparency. E-government such as in procurement (Lewis-Faupel et al., 2016), freedom of\r\ninformation requests (Escaleras, Lin and Register, 2010; Islam, 2006) and greater transparency of\r\ninformation can help citizens keep politicians accountable and lead to better service provision, particularly\r\nin young democracies such as Armenia (Hollyer, Rosendorff and Vreeland, 2018; Keefer and Khemani, 2005;\r\nBanerjee et al., 2018). We discuss this more in depth below, but generally note that Armenia should continue\r\nits implementation of its e-governance strategy, particularly given its educated population with a\r\npenchant for protest.\r\nREFORM/INTERVENTION EXAMPLES\r\nAudits. Audits can help expose corrupt politicians (Ferraz and Finan, 2008), ensure development projects are\r\ncompleted according to specification (Olken, 2007), and reduce political corruption at least in the short-term\r\n(Di Tella and Schargrodsky, 2003; Bobonis, Fuertes and Schwabe, 2016; Ferraz and Finan, 2011; Avis, Ferraz\r\nand Finan, 2018). One way to stretch limited funds, ensure fairness, and prevent politicians from\r\nmanipulating the audit process is to randomize it, for which Brazil offers a wealth of positive experience\r\n(Ferraz and Finan, 2018). Although the recent assessments of the Armenian Audit Chamber are generally\r\npositive, randomization of audits may offer one way to alter and break-up perceptions of a negative,\r\nsymbiotic relationship between elites, oligarchs and the state (OECD, 2018; Wicksberg and Hoktanyan,\r\n2013; Shahnazarian, N.d.; Iskandaryan, 2012). We provide more details in the Integrity Systems and Rule\r\nof Law Evidence Review also under this tasking (USAID, 2019b).\r\nCommunicating Directly with Politicians. Beyond what measures government can induce from the top,\r\ncitizens can utilize technology to promote better governance outcomes as well (World Bank, 2004;\r\nKosack and Fung, 2014). Evidence from Uganda indicates that when citizens can text politicians directly,\r\npoliticians are less corrupt (Buntaine et al., 2018). Although politicians may not always deliver better\r\nservices in response to more information, since service provision may be out of their control, the\r\npoliticians are ultimately more responsive to citizens (Grossman and Michelitch, 2018). If citizens provide\r\n22 Malapportionment refers to “the discrepancy between the shares of legislative seats and the shares of population held by\r\ngeographical units” (Samuels and Snyder, 2001: 652).\r\nUSAID.GOV                                                        GOVERNANCE IN ARMENIA: AN EVIDENCE REVIEW                | 20\r\n"                                                                                                                                                                                                                                                                                                                                                                                                                                                                                                                                                                                                                                                                                                                                                                                                                                                                                     
## [24] "actionable information to politicians (Grossman, Platas, and Rodden, 2018), Armenia may have success\r\nwith similar platforms, especially given the momentum from its 2018 political transition.\r\nGrievance Redress Mechanisms, Social Audits, and Hotlines. Grievance redress mechanisms\r\n(GRMs) are “locally based, formalized ways to accept, assess, and resolve community feedback or\r\ncomplaints” (World Bank, 2013: 1). Overall, GRMs provide a way to address feedback from\r\ncommunities in a meaningful way, notably at early stages of development interventions—for example,\r\nbefore minor grievances can develop into major problems. Many countries have their own GRMs, but in\r\nsome cases it may be necessary for development financiers to initiate project- or intervention-specific\r\nGRMs. That is especially the case for development projects that trigger social and environmental risk\r\nmanagement (i.e. “safeguard”) policies, such as those pertaining to indigenous peoples, involuntary\r\nresettlement, and the environment. Grievance redress mechanisms differ in their procedures and\r\nstructures, including the extent to which complaints are filed, advertised to potential beneficiaries,\r\naddressed within certain timeframes, etc. (World Bank, 2012). What is clear, though, is that having a\r\nhotline for beneficiary complaints or a formal grievance redress mechanism can alert bureaucrats to\r\nservice delivery issues and improve performance. Alternatively, aid agencies and policymakers may wish\r\nto consider social audits: “mechanisms in which information on expenditures and implementation\r\nproblems is gathered and then presented for discussion in a public meeting involving all stakeholders”\r\n(Joshi, 2013: S38). More recently, the World Bank has used social audits for project beneficiaries to take\r\npictures of project implementation. Since there does not appear to be any side effects of hotlines,\r\ngrievance redress mechanisms, or social audits, Armenia may wish to consider them, as appropriate.\r\nCommunity-Based Monitoring. The literature on community-based monitoring does not provide\r\nconclusive results, but the results are not fully negative in terms of public service delivery (Casey, 2018).\r\nFor example, a series of experiments monitoring the Ugandan health sector suggest that community-\r\nbased monitoring is most effective when monitors are more intelligent, there is less ethnic\r\nfractionalization, and there is less inequality (Björkman and Svensson, 2009; Björkman and Svensson,\r\n2010; Björkman, de Walque, and Svensson 2017). However, in an education intervention in India,\r\nBanerjee et al (2010) find that community-based monitoring through village councils had no impact on\r\nservice provision. Additionally, Olken (2007) found that community-based monitoring through village-\r\nlevel accountability meetings in Indonesia was less effective in facilitating public service provision (i.e.,\r\nroad construction) than technical audits of core samples conducted by outside firms.\r\nMany scholars and practitioners in the development community have interpreted the results from Olken\r\n(2007), Banerjee et al. (2010), and recent failures to fully replicate the Björkman and Svensson\r\nexperiments 23 as confirmation that community-based monitoring is ineffective, but Fox (2015) provides\r\na more nuanced assessment. For Olken (2007), the KDP development program that he studied had\r\nalready mobilized citizens years before his intervention, and Banerjee et al. (2010) relied on village\r\ncouncils that were comprised and determined by elites, who were susceptible to capture (see Bardhan\r\nand Mookherjee, 2006). Accordingly, as Fox (2015) points out, neither study provided a truly zero\r\nbaseline from which to judge program effectiveness, a critique in line with recent debates about the\r\npitfalls of p-values and null hypothesis significance testing in social science.24\r\n23 See a recent working paper by Raffler, Posner, and Parkinson (2018).\r\n24 Most social science studies calculate statistical significance through null hypothesis significance testing, a procedure that\r\nentails calculating p-value statistics. Studies using p-values generally set the null hypothesis equal to zero, meaning that the\r\nbaseline effect is always zero. Famously, Andrew Gelman has argued that zero cannot be the baseline for any social science\r\nUSAID.GOV                                                                 GOVERNANCE IN ARMENIA: AN EVIDENCE REVIEW              | 21\r\n"                                                                                                                                                                                                                                                                                                                                                                                                                                                                                     
## [25] "More broadly, the effectiveness of a community-based monitoring intervention—or participatory\r\nprogram more generally—depends neither just on the voice-related measures that the intervention\r\nfosters nor the success of the one particular intervention. Instead, participatory programs are more\r\nsuccessful when there is state capacity, especially in the form of horizontal accountability, 25 to hold\r\nbureaucrats and politicians accountable for acting on the findings of the participatory interventions (Fox,\r\n2015). Although horizontal accountability is a challenge for Armenia, the country does have a highly-\r\neducated and cohesive population that effectively spurred its 2018 political transition. Combining these\r\nfactors with the consensus that participatory programs tend to increase citizen-level satisfaction with\r\ngovernment (Beath et al, 2017; Casey, 2018), it indicates that community-based monitoring could be\r\nuseful for Armenia.\r\nRight to Information Laws, Freedom of Information Requests, and Transparency Laws. Under\r\nmany circumstances, politicians and bureaucrats may not improve service delivery without some sort of\r\ncommitment device. One such device can be a right to information law, also known as a transparency\r\nlaw. As a set of field experiments in India show, right to information laws can increase service delivery\r\noutputs and are almost as effective as bribery in doing so (Peisakhin and Pinto, 2010; Peisakhin, 2012).\r\nArmenia may thus wish to consider an adopting a right to information law and accompanying e-\r\ngovernance platform that makes right to information requests easy for citizens and hard for bureaucrats\r\nand politicians to ignore.\r\nCitizen Service Centers, Citizens Guide to the Budget, and Benchmarking. Citizen Service Centers,\r\nalso called one-stop-shops, one window systems, and citizen facilitation centers, are hubs that provide\r\ncitizens access to government services all in one location. The states of Sao Paolo in Brazil, Andra\r\nPradesh in India, and New Delhi in India have used Citizens Service Centers with particular success, and\r\nthey are far from the only ones (World Bank, 2011). A recent review suggests that at least 77 countries\r\nare employing Citizen Service Centers (Pfeil et al., 2016). Ideally, they should provide citizens with a\r\nclear guide to the budget and benchmarking on the amount of services that citizens receive. Armenia has\r\nstarted using Citizen Service Centers, but their effectiveness and how many functions they serve is\r\nunclear.\r\nCitizen Report Cards. Citizen report cards are surveys of citizens regarding their use of public services\r\nor government officials, usually completed with an NGO or research institute (Agarwal, Post, and\r\nVenugopal, 2013). A scorecard of members of the Ugandan parliament did not change politicians’\r\nbehavior, even with a media campaign (Humphreys and Weinstein, 2012). However, scorecards of\r\npoliticians’ corruption levels in Mexico changed voters’ behavior. Specifically, it hurt electoral prospects\r\nfor relatively corrupt mayors, but made some voters abstain from voting (Chong et al., 2015). If Armenia\r\nconsiders citizen report cards, it would be wise to take into consideration that some accountability\r\ntools like citizen report cards can have negative externalities.\r\nParticipatory Budgeting. Having citizens decide a certain portion of how local governments spend\r\ntheir money has improved perceptions of democracy and service delivery in Brazil, Mexico, and India.\r\nThe Brazil studies attribute also participatory budgeting to a drop in infant mortality, a crucial indicator\r\nintervention, given that social science does not take place in a vacuum (e.g., Gelman and Carlin, 2017). Accordingly, Gelman\r\nhas argued that social science should use more Bayesian statistics, which does not suffer from the same pitfalls of null\r\nhypothesis significance testing and p-values.\r\n25 Again, horizontal accountability refers the ability of state-level institutions to exert checks and balances on each other\r\n(O’Donnell, 1998).\r\nUSAID.GOV                                                                GOVERNANCE IN ARMENIA: AN EVIDENCE REVIEW            | 22\r\n"                                                                                                                                                                                                                                                                                                                                                                                                                                                                                                                                                                                                                                                                                                                                                                                      
## [26] "for development (Touchton and Wampler 2013; Touchton, Sugiyama, and Wampler 2017). At a smaller-\r\nscale, results were similar in terms of perceptions for community-driven development projects in\r\nIndonesia and Afghanistan (Olken, 2010; Beath, 2017). Against this backdrop, Armenia may want to\r\nconsider participatory budgeting programs, especially if it pushes forward with decentralization.\r\nSocial Programs. Both Mexico and Brazil, in particular, have enjoyed particular success with their conditional\r\ncash transfer programs: Oportunidades (Mexico) and Bolsa Familia (Brazil). The idea behind these social\r\nprograms is that they target the poor, and beneficiaries have to do something in exchange (e.g., send their\r\nchildren to school and the doctor) to continue receiving the benefits. In Brazil, Bolsa Familia has not only\r\ncontributed a lowering of inequality but also the decline of clientelism and formation of true citizens (Hunter\r\nand Sugiyama 2014, Sugiyama 2016, Sugiyama and Hunter 2013, Lindert et al. 2007). 26 For Mexico, some\r\nresearch appears to suggest that Oportunidades contributed to the decline of clientelism (De La O, 2015), but\r\na new article indicates that De La O’s (2015) finding is the result of a coding error (Imai, King, and Velasco,\r\n2019). Regardless, it is clear that social programs create legitimacy and confidence in the state. These two\r\noutputs, in the very least, would be useful for the new Armenian government.\r\nLEARNING FROM REGIONAL/SIMILAR COUNTRY EXPERIENCE\r\nGeorgia’s success at limiting petty corruption in service delivery under the leadership of President\r\nMikheil Saakashvili is a point of reference for Armenia. As Table 1 (above) shows, measures of\r\ncorruption declined dramatically and immediately following Georgia’s political transition in 2004.\r\nSaakashvili orchestrated these decreases in corruption through a “big bang” approach, with many\r\nreforms at once, and lots of “learning by doing” (Di Puppo, 2010). The country also relied heavily on\r\nWestern pressure to draft an anti-corruption strategy, as well as Western support to finance a number\r\nof projects related to service delivery (Di Puppo, 2010). 27 Saakashvili and his cadre of reformers,\r\nnevertheless, undertook the majority of the work, often with the help of new information technology\r\nand e-governance, including at the Public Registry (Schalkwyk, 2010; Mungiu-Pippidi, 2016). Additionally,\r\nGeorgia created many citizen service centers to facilitate document processing (Anderson, 2018). As\r\ndescribed above, Armenia is already following in Georgia’s footsteps concerning e-governance and\r\ncitizen service centers, and should continue with these efforts to improve service delivery. It is unlikely\r\nthat Armenia can implement reforms at the same pace as Georgia, though. Armenia’s geopolitical\r\nconsiderations described in Section 1 of this report provides one constraint. Another relates to the fact\r\nthat Georgia during the Saakashvili era was a competitive authoritarian regime with less constraints on\r\nthe executive (Levitsky and Way, 2010), whereas current day Armenia will have the more robust\r\ndemocratic institutions described in Section 2.2.\r\nMoldova has improved service delivery since its political transition in 2009 with help from two e-governance\r\ntransformation projects financed by the World Bank. However, citizen-level satisfaction with service delivery\r\nremains extremely low, and political dynamics have constrained the country from making further progress\r\n(World Bank, 2017a, 2017b; Munteanu, 2018). Since Georgia was able imprison its oligarchs with high-level,\r\ntelevised arrests (Di Puppo, 2010), but Moldova has not done so, it seems to suggest that oligarch influence\r\nslows down service delivery reforms too (see also, Anderson, 2018). If so, it suggests that reforms will be\r\nslower to yield results for Armenia, a country for which around ten oligarchs own circa 50% of its wealth\r\n(USAID CDCS, 2017).\r\n26 For more the transformation from subjects to true citizens, see Fox (1994).\r\n27 For some service delivery projects, see, for example: http://www.worldbank.org/en/country/georgia/projects\r\nUSAID.GOV                                                           GOVERNANCE IN ARMENIA: AN EVIDENCE REVIEW | 23\r\n"                                                                                                                                                                                                                                                                                                                                                                                                                                                                                                                                                                                                                                                                                              
## [27] "Bribe-taking by bureaucrats has marked Ukraine’s since the country’s independence from the Soviet\r\nUnion in 1991 (Anderson, 2018). Given Georgia’s success with rooting out corruption, and the fact that\r\nMikheil Saakashvili studied in Ukraine with President Petro Poroshenko, the latter invited Saakashvili to\r\nUkraine in 2014. Initially, Saakashvili served as an advisor to President Poroshenko; then, in 2015,\r\nPoroshenko granted Saakashvili Ukrainian citizenship and appointed him as governor of Odessa. Despite\r\nSaakashvili’s considerable experience with rooting out corruption in Georgia, Saakashvili resigned his\r\npost in Ukraine in 2016, citing corruption as a primary reason for his resignation (Walker, 2016).\r\nSaakashvili’s experience in Ukraine demonstrates for Armenia that Georgia’s experience is not\r\nnecessarily replicable elsewhere.\r\nTAKEAWAYS\r\nGiven that Armenia is already undertaking many e-governance measures, and the population is educated\r\nand has a penchant for successful protest, Armenia should continue to involve the population through\r\nparticipatory programs. In particular, continuing with citizen service centers, as well as perhaps initiating\r\nsome combination of participatory/open budgeting, right to information laws, grievance redress\r\nmechanism, hotlines, and social audits constitute some promising potential interventions. Although\r\nparticipatory programs do not always work, as evidenced above, Armenia’s demographic characteristics\r\nand its political transition provide conditions to facilitate the success of participatory measures. Even if\r\nthey are unsuccessful, research generally suggests that participatory programs do increase satisfaction\r\nwith governance and democracy (Olken, 2010; Beath et al., 2017; Chase and Labonne 2011; Besley et al.,\r\n2005). If Armenia’s political transition does not bring about the expected levels of change, which is a real\r\nrisk, it could lead the population to sour on democracy and facilitate the success of populist candidates\r\nthat ultimately bring back authoritarianism (see, Levitsky and Ziblatt, 2018). In this light, it is important\r\nto keep the population engaged with the government.\r\n   2.4.      CIVIL SERVICE REFORM\r\nCONTEXT/STATUS\r\nNo modern state can succeed without some form of a civil service. 28 A competent civil service with\r\nmeritocratic recruitment and promotion contributes to political stability (Bai and Jia, 2016), less\r\ncorruption (Charron et al., 2017), and is a principal factor that underpins a state’s ability to impersonally\r\napply the rule of law without succumbing to patrimonial and personalist leader-based pressures (Weber\r\n1978; Fukuyama 2011). Simply put, a competent civil service is integral for ensuring the implementation\r\nof policies across a territory (Mann, 1984).\r\nNowadays, most states civil services comprise positions such as functionaries, military officers, teachers,\r\npolice, etc. Unitary states such as Armenia mainly contract such positions at the national level, whereas\r\ndecentralized states such as Colombia do so primarily at the subnational level. Regardless of states’\r\nparticular systems of government, most states have a law similar to that of the United States’ Pendleton\r\nAct of 1883, requiring merit-based hiring for mid- and high-level civil service positions—usually by means\r\nof an entrance exam. Research is unanimous in underscoring the effectiveness of the Pendleton Act and\r\nthose like it in other countries to combat patronage (loyalty-based appointments) and corruption\r\n(Theriault, 2003; Grindle 2012).\r\n28 One of the reasons why many consider China to be the first modern state relates back its meritocratically-recruited and\r\nhighly competent civil service, for which it maintained a competitive recruitment exam for over 1,300 consecutive years\r\n(Fukuyama 2011; Bai and Jia, 2016).\r\nUSAID.GOV                                                            GOVERNANCE IN ARMENIA: AN EVIDENCE REVIEW           | 24\r\n"                                                                                                                                                                                                                                                                                                                                                                                                                                                                                                                                                                                                                                                                                                                                                                                                                                                                                                                                                                                                           
## [28] "Armenia undertook its first civil service legal framework reform efforts in 1994, with some re-\r\ndevelopment in 1997 and implementation in 1998 (Nemec, 2016). In 2001, Armenia adopted its Civil\r\nService Law (CSL) (OECD, 2011). The CSL not only contains a provision on meritocratic hiring and\r\npromotion along the lines of the US’s Pendleton Act, but the CSL also mandated the creation of a Civil\r\nService Council (CSC). The CSC is a public-sector body that oversees all hiring in the Armenian civil\r\nservice. The CSC has worked with and/or has receiving funding from other governments (e.g., France,\r\nEgypt, Cyprus, China) and numerous international donors (e.g., European Union, OECD, USAID, World\r\nBank, OSCE) (Davatyan, N.D.).\r\nDespite these efforts, survey evidence suggests that citizen-level trust in the government at all levels is\r\nlow (EBRD, 2016). Analysis of the Armenian civil service by the OECD (2011) highlights numerous\r\ndeficiencies, including in professionalism, independence, and codes of ethics. More recently, the Armenia\r\ngovernment updated its Civil Service Law in 2014 to facilitate more training and modern performance\r\nevaluation, thereby making it easier for competent staff to obtain promotions (World Bank, 2015).\r\nIn terms of development assistance, Armenia’s civil service has benefitted from an Institutional\r\nDevelopment Fund grant, helping to create its government e-financial management system, as well three\r\nWorld Bank public sector modernization projects. The third of these projects will close on December\r\n2020. It focuses on financial reporting and e-government. Notable inventions related to the civil service\r\ninclude: the development of an e-consular system, to reduce wait times on document processing; a new,\r\nelectronic system for pre-court trial proceedings in the judiciary sector; and an electronic case file and\r\nred flags system for the Ethics Commission (World Bank, 2015; World Bank, 2018).\r\nFACTORS PROMOTING/HINDERING\r\nLow Salaries. Although the causal evidence is far from conclusive, some analysts have shown that\r\nsalaries can play a role in attracting higher quality government employees (Dal Bó et al., 2013; Finan et\r\nal, 2017). On this score, the pay for Armenian civil servants is low relative to the private sector,\r\nespecially given the difficult entry requirements. Accordingly, the Armenian civil has experienced\r\ndifficulties hiring highly qualified and technical staff (World Bank, 2015).\r\nEducation. Without an educated population, the quality of the labor pool for civil servants deteriorates\r\nsignificantly. Some studies further suggest that educated people have less tolerance for corruption and\r\nare in a better position to resist pressure from bureaucrats and party cadres, and thus are less\r\nsusceptible to bribes and other corrupt activities (Cheung and Chan, 2008; Machin, Marie and Vujic,\r\n2011; Uslaner, 2017). Since the average adult Armenian has benefitted from 12.5 years of schooling\r\n(UNESCO, 2018), and university-level educational completion rates are on the rise (World Bank, 2018),\r\nArmenia should be able to field a high-quality civil service.\r\nBureaucratic Reform and Patronage. Bureaucratic reform is a very challenging task in most countries,\r\nand that appears to be especially the case in Armenia. By many accounts, the elite has captured\r\nbureaucracy in Armenia; patronage politics, as opposed to impersonal application of law and policy,\r\nappears to be the norm (Paturyan and Stefes, 2017; Lewis, 2017). Patronage politics is also highly\r\nexpensive, too, because a job involves a long commitment over many years (Xu, 2017; Gingerich, 2013;\r\nRobinson and Verdier, 2013). Recently, the Armenian government attempted some reforms in policing,\r\nbut only achieved moderate success (Shahnazarian and Light, 2018).\r\nUSAID.GOV                                                   GOVERNANCE IN ARMENIA: AN EVIDENCE REVIEW   |   25\r\n"                                                                                                                                                                                                                                                                                                                                                                                                                                                                                                                                                                                                                                                                                                                                                                                                                                                                                                                                                                                                                                                                 
## [29] "Election Administration. Although not a part of every country’s civil service, international electoral\r\nwatchdog bodies played a role in safeguarding democracy from election fraud (Hyde, 2011). Armenia has\r\nenjoyed success with election monitoring from outside entities (Hyde, 2007), but the country does not\r\nhave an independent electoral institute. Instituting one with the same model as Mexico’s National\r\nElectoral Institute (see Magaloni, 2010) might help ensure that electoral manipulation does not become a\r\nform of bureaucratic control (Gehlbach and Simpser, 2015).\r\nREFORM/INTERVENTION EXAMPLES\r\nPerformance Pay and Pro-Social Motivation. There are very few studies that credibly examine the\r\nimpact of performance pay with sound causal identification, but one well-done study by Khan et al.\r\n(2016) on property tax collectors in Pakistan provides some clues. Overall, the study finds that civil\r\nservants who received the performance pay treatment increased tax collection substantially. By the\r\nsame token, these same civil servants also increased their bribe amounts. Khan et al. (2016) suggest the\r\nhigher bribe amounts are a function of the higher opportunity cost associated with foregoing\r\nperformance pay. Therefore, before offering any sort of performance pay or increased salary, it is\r\nadvisable to test for civil servants’ pro-social motivation (Hanna and Wang, 2017; Besley and Ghatak,\r\n2018). With respect to the latter, numerous studies show that higher wages may not attract those with\r\nthe best values (Ashraf et al., 2014; Banuri and Keefer, 2016), a quality that is especially important in a\r\ncountry like Armenia where corruption and patronage are the equilibrium (see Fisman and Golden,\r\n2017). In a literature review of performance pay, Hasnain, Manning, and Pierskalla (2014) identified craft\r\njobs such as teachers, health care workers, and revenue administrators as the promising vocations for\r\nperformance pay. Given that health care workers and tax collectors are particularly prone to bribery,\r\nand education in Armenia is generally strong by international standards, we are hesitant to recommend\r\nperformance pay for Armenian civil servants. We also strongly suggest quietly instituting a test for pro-\r\nsocial motivation and testing for potential manipulation and cheating.\r\nReporting Structures. The literature on the effectiveness of different reporting structures within civil\r\nservice is only incipient, but two studies are provocative. In their study of the Nigerian civil service,\r\nRasul and Rogger (2018) find that mid-level bureaucrats are more effective at carrying out small-scale\r\nproject implementation when their supervisors grant them more autonomy, as opposed to when they\r\nare subject to more monitoring. Given that the study does not rely on an experimental design, though,\r\nthe authors are unable to credibly rule out other potential causes, such as ambition, pro-social\r\nmotivation, and task complexity. A more credibly designed study concerning bureaucrats in highly\r\ndecentralized India suggests that bureaucrats perform more effectively when they report to only one\r\npolitician (Gulzar and Pasquale, 2017). If Armenia considers decentralization, which we caution against\r\nbelow, then it will need to find an effective way to organize reporting structures for its nearly 500\r\nmunicipalities.\r\nPerformance-Based Postings and Incentives. It is normally very difficult to disentangle whether\r\nsupervising bureaucrats award positions to their employees on the basis of merit, need, patronage, or a\r\ncombination of the three factors. To obviate these complicated measurement issues, a study of tax\r\ncollectors in Pakistan randomized whether employees participated in a scheme in which they could pick\r\ntheir postings based on their performance or maintained the bureaucratic supervisor’s authority to\r\ndecide postings (Khan, Khwaja and Olken, 2019). The study found a substantially positive effect on tax\r\nrevenue collected for the performance-based posting vis-à-vis the regular scheme. Given that the study\r\ntook place in a very weak institutional environment that is similar or worse to that of Armenia, it is\r\nUSAID.GOV                                                  GOVERNANCE IN ARMENIA: AN EVIDENCE REVIEW      | 26\r\n"                                                                                                                                                                                                                                                                                                                                                                                                                                                                                                                                                                                                                                                                                
## [30] "worth considering whether Armenia could experiment with performance-based posting. Perhaps\r\nperformance-based posting could induce more qualified individuals to join the civil service, too.\r\nE-Governance. Particularly since Armenia adopted an E-Government Strategy in 2014, and e-\r\ngovernance generally limits bureaucrats’ ability to bribe citizens, e-governance is area for an\r\nintervention. Below, we provide a summary of relevant e-governance measures, each time providing\r\nsome context to Armenia.\r\nE-Procurement. A well-done causal study on India and Indonesia, for example, suggests that introducing e-\r\nprocurement significantly reduced capture and leakage risks relative to the former non-digitized\r\nprocurement systems (Lewis-Faupel et al., 2016). Armenia has already adopted an e-procurement system, a\r\ntopic that we tackle in further detail in Integrity Systems and Rule of Law Evidence Review (USAID, 2019b).\r\nBiometric Smart Cards. Another method of reducing potential leakage of government funds entails\r\nsupplying citizens with smart, biometric ID cards that document fund transfers. For the massive NREGA\r\nworkfare program in India and a food subsidy program in Indonesia, sound causal evidence indicates that\r\nthese smartcards significantly reduced corruption by government officials (Muralidharan, Niehaus and\r\nSukhtankar, 2016; Banerjee et al., 2018.). The World Bank (2015) reports that some of Armenia’s\r\nrecent e-governance efforts concerns the use of smart cards, but they are a technology that\r\npolicymakers may wish to consider expanding to other areas as well.\r\nE-monitoring. Citizens may use technology to monitor government bureaucrats, which is exactly how\r\nparts of India combated teacher absenteeism. Specifically, the government linked teachers’ salaries to\r\ntheir appearance (or non-appearance) on cameras set-up at schools helped with the monitoring (Duflo,\r\nHanna and Ryan, 2012). However, e-monitoring can be disruptive. When Dhaliwal and Hanna (2017)\r\nintroduced e-monitoring of health workers in India, it produced staff dissatisfaction and highly-trained\r\ndoctors quit. If Armenia considers such an innovation for its civil service, it should weigh the costs and\r\nbenefits of introducing such a program, particularly given the staffing issues highlighted above.\r\nLimiting Petty Corruption. Although elite-level corruption is usually difficult to tackle (Bauhr and\r\nCharron, 2018), and may not be wise to address in the earliest stages of transition, governments may\r\nnonetheless focus on petty corruption: bribes to police, teachers (grades and national exams),\r\nbureaucrats who grant permits, doctors who insist on special payments, etc. These activities are easy to\r\ngather evidence on, such as when plain-clothes police or staff of the anti-corruption agency assume the\r\nposition of a person who is likely to be asked for a bribe. Petty corruption is also easy to prosecute\r\nsince the offenders hold no political power and there is little at stake for judges and prosecutors. If the\r\ngovernment is successful in stemming petty corruption it can win legitimacy and begin to change the\r\nculture of corruption.\r\nAsset Declaration. Asset declaration has proven informative for India, demonstrating private returns to\r\npublic office. As Fisman et al. (2014) show, elections winners in India gained 3-5% more assets per year\r\nthan the runners-up—even though government salaries are fixed. In line with India, Armenia requires\r\nhigh-ranking officials to declare their assets and, as of 2016, has accompanying legislation to penalize\r\nthose who do not truthfully declare their assets.\r\nTransparency of Tax Records. To create even more public trust in government, Armenia could go\r\nbeyond asset declaration. For example, the government could require applicants and current holders of\r\nUSAID.GOV                                                     GOVERNANCE IN ARMENIA: AN EVIDENCE REVIEW   | 27\r\n"                                                                                                                                                                                                                                                                                                                                                                                                                                                                                                                                                                                                                                                                                                                                                                                                                                                                                                                                                                                                                                                     
## [31] "all public-sector jobs (or at least all elective offices) to make their tax returns public, perhaps with\r\ncooperation from the Tax Service of the Republic of Armenia. The transparency of tax records would\r\nhave several beneficial effects. First, it would allow citizens to see how much money their public servants\r\nhave (on record), similar to asset declarations. Second, it would allow citizens and watchdog agencies to\r\nstep forward with additional information that may have been neglected in those returns (e.g., a house or\r\nbusiness that was not declared). Third, it would serve as an impetus to the Tax Service of the Republic\r\nof Armenia to take a close look at public servants, and counter any corruption within the agency to\r\nshelter “friends of the current regime” or people who may have paid off IRS agents. Prosecutions might\r\nfollow. Fourth, it would allow citizens, watchdog agencies, and the media to compare declared income\r\nagainst lifestyle. If someone declares a modest income but seems to enjoy a lavish lifestyle this could be\r\nnoted publicly, and perhaps become fodder in subsequent elections. Finally, it would discourage people with\r\nless pro-social motivation from pursuing a career in public sector service, and might prompt some of those\r\nwho currently hold office to resign—a method of lustration.\r\nLEARNING FROM REGIONAL/SIMILAR COUNTRY EXPERIENCE\r\nGeorgia undertook its “big bang” approach to civil service reform with very high-profile reforms:\r\ntelevising arrests of oligarchs and high-ranking officials, building a large-scale anti-corruption unit within\r\nthe executive branch, completely dismissing its police force, etc. (Devlin, 2009; Di Puppo, 2010; Light,\r\n2014). Given that Armenia is more tied to Russia than Georgia (see Section 1), 10 Armenian oligarchs\r\ncontrol about half of the country’s wealth (USAID CDCS, 2017), and Armenia already has trouble with\r\nstaffing its bureaucracy and enacting reforms (World Bank, 2015; Shanazarian and Light, 2018), it would\r\nbe unlikely that Armenia could replicate Georgia’s approach to civil service reform. Furthermore, as the\r\ninstitutionalist literature shows, powering through “big-bang” reforms at “critical junctures” tends\r\nproduce instability in weak institutional environments—something Armenia should avoid (Bersch, 2016,\r\n2019).\r\nIn 1997 the President of Kazakhstan set a long-term vision with “Kazakhstan 2030”, including reform of\r\ncivic service efforts in 1999. Kazakhstan limited the executive influence on civil servants by dividing\r\nhiring between political and career appointments in attempt to shield from political changes. It appears\r\nas though the Civil Service Council (CSC) of Armenia has already undertaken such steps, notably with\r\nsupport from donors and foreign governments (Nemec, 2014).\r\nMoldova’s civil service has recently undergone major revamping to meet European Union standards on\r\ngood government, with external support from the World Bank and the UK’s Department for\r\nInternational Development (DFID). According to the World Bank’s Independent Evaluation Group\r\n(2015), reforms in merit-based hiring, personnel policy, compensation, budget planning, strategic\r\nplanning were successful, though fraught with delays. Given the aforementioned issues with service\r\ndelivery in Moldova (see previous section), it indicates that civil service reforms do not immediately\r\ntranslate to better service delivery. Accordingly, Armenia should expect delays in output-based\r\nmeasures in governance improvement,29 even if it continues to improve its civil service.\r\nUkraine instituted some decrees and regulations related to its civil service in the period immediately\r\nfollowing the fall of the Soviet Union, stretching roughly from 1991-1997 (Nemec, 2014). However, the\r\ngovernment did not implement these decrees and regulations very thoroughly due to the chaotic nature\r\nof the democratic transition (EBRD Annual Report 2004, 2005). More recently, Ukraine has attempted to\r\n29 For more on output-based measures governance effectiveness, refer to Evans, Huber, and Stephens (2017).\r\nUSAID.GOV                                                        GOVERNANCE IN ARMENIA: AN EVIDENCE REVIEW |   28\r\n"                                                                                                                                                                                                                                                                                                                                                                                                                                                                                                                                                                                                                                                                                                                                                               
## [32] "overcome some its challenges through an open/transparent budgeting law in 2015 as well as some\r\nregional and city-wide initiatives, with some more successful than others (Walker, 2016; Anderson,\r\n2018). As to whether such a strategy is feasible for Armenia, it is unclear given its unitary government\r\nand apparent interest in decentralization, but it may be worth a try.\r\nTAKEAWAYS\r\nOne of the most frequent themes emerging from the literature on Armenia’s governance is that corruption is\r\na pervasive and endemic problem in the civil service. The identification of the problem is one matter;\r\nidentifying concrete steps for overcoming problems of corruption is quite another. As such, a critical factor in\r\napplying global or regional lessons is to identify concrete steps, and perhaps more critically proceed at the\r\nright pace and with the right expectations.\r\nOverall, it may be difficult to enact civil service reforms in Armenia to end corruption. In situations\r\nwhere corruption is the equilibrium behavior among large groups, there is generally a lack of “principled\r\nprincipals”: that is, supervising agents that can change behavior of others and stop the corruption\r\nentirely (Persson, Rothstein, and Teorell, 2013). Accordingly, it is best to conceive of the corruption and\r\npatronage in Armenia’s civil service as more of a collective action problem than a pure monitoring\r\nproblem (Rothstein, 2011). In other words, additional monitoring and transparency are likely not\r\nenough, and may even be harmful (Bauhr and Grimes, 2014). In such a setting, there is a need to\r\ndistinguish between “need corruption” (i.e., the need gain access to public services, which facilitates\r\ncollective action) and “greed corruption” (i.e., some individuals gaining illicit money at the expense of\r\nthe rest, which facilitates free-riding) (Bauhr, 2017). Armenia seems to have elements of both “need”\r\nand “greed” corruption. Not everyone has access to the unbiased access to the rule of law and public\r\nservices, but there are also some bureaucrats who are clearly enriching themselves for private gain.\r\nGiven Armenia’s political transition, there may be an opening for a measure that would not work under\r\nordinary times.\r\nAbove, we indicated that Armenia as well as international assistance providers are likely in a better position to\r\naddress petty corruption — bribes to police, teachers (grades and national exams), bureaucrats who grant\r\npermits, doctors, etc. Petty corruption is easier to address than grand corruption (Bauhr and Charron,\r\n2018), and addressing petty corruption was a measure that Georgia undertook successfully under\r\nSaakashvili (Di Puppo, 2010). In an effort to manage expectations appropriately, addressing petty\r\ncorruption provides some opportunity for a quick win. If the ambition is to eradicate grand corruption, then\r\nall parties are likely to be sorely disappointed. Curbing grand corruption generally happens only over a\r\nlonger period of time (Rothstein, 2011).\r\nOverall, it seems that Armenia should take a more gradual—or at least lower profile—course than\r\nGeorgia, while simultaneously seizing the momentum of its recent political transition. The reforms we\r\nsuggested above regarding random inspections for petty corruption, more e-government (e.g., biometric\r\nsmart cards), testing for pro-social motivations (but quietly), performance-based postings, and\r\ntransparency in hiring and civil servants’ tax records constitute some measures to consider.\r\n  2.5.      MISCELLANEOUS FACTORS THAT CONTRIBUTE TO GOVERNANCE\r\nSome factors that affect the quality of governance are difficult to change in that they are unlikely to be\r\nresponsive to policy initiatives emanating from USAID or the government of Armenia at least in the short-\r\nUSAID.GOV                                                  GOVERNANCE IN ARMENIA: AN EVIDENCE REVIEW        |   29\r\n"                                                                                                                                                                                                                                                                                                                                                                                                                                                                                                                                                                                                                                                                                                                                                                                                                                                                                                                                                                                                                                                                                             
## [33] "to medium-term. 30 Even so, they structure the political and policy sphere in important ways and thus\r\nform an essential backdrop to our report. The following factors may assist reform initiatives in Armenia,\r\nor at least not hinder them:\r\nSOCIAL COHESION\r\nSocial cohesion refers to the togetherness —or “sticking-togetherness” (Gross and Martin, 1952, 553)—\r\nof a community, i.e., the sense in which members identify and behave as members of a coherent, unitary\r\ngroup. Social cohesion has been argued to create higher levels of consensus on political matters, and are\r\nalso likely to foster a high level of social trust and —given the opportunity —of political engagement, as\r\nsuggested by work on social capital (Alesina et al., 2003; Alesina, Baqir and Easterly, 1999; Lieberman,\r\n2009) and demography (Gerring and Veenendaal, 2019).\r\nSocial cohesion is a hard thing to measure, but all reports suggest that Armenian society is one of the\r\nmost cohesive societies in the world. This may be interpreted as a product of Armenia’s long, unbroken\r\nhistorical trajectory, the impact of the genocide in reinforcing group identity, the ongoing war with\r\nAzerbaijan, a high level of linguistic and religious homogeneity along with a language and religion that\r\nuniquely identify Armenians from all other peoples, a very small population, one third of whom reside in\r\nthe capital city, and a compact geographic territory. These factors bring Armenians together, and are likely\r\nto foster consensus on matters of politics.\r\nDIASPORA\r\nDiaspora populations can play an important role in development and reconstruction (Mitra et al., 2007).\r\nAlthough most of those who identify as Armenian no longer reside in the country, they retain close ties,\r\nsend regular remittances home to family members, and could be engaged in partnerships—political, civic,\r\nor business. Since ex-patriots are often highly skilled and some are quite wealthy, Armenia’s diaspora is\r\nan important asset and should be fostered wherever possible. With that said, Armenian diasporas exist\r\nin multiple countries with very different interests, including the United States and Russia.\r\nEDUCATION\r\nAccording to the World Development Indicators, over 99% of Armenia’s population is literate, and\r\nschooling completion rates are high (World Bank, 2017a; UNESCO, 2018). According to UNESCO\r\n(2018), in 2015 the average Armenian benefited from 12.5 years of schooling. Thus, Armenian education\r\nshould be conducive to the emergence of good governance.\r\nGEOGRAPHY AND FOREIGN PRESSURES\r\nWe note that foreign pressures have some potential to impact the success of governance reform initiatives.\r\nArmenia is perhaps particularly susceptible. It has notable frictions with Azerbaijan, with which hostilities\r\ncontinue. There is no literature to suggest that these hostilities in themselves would hinder domestic\r\ngovernance reform within Armenia, but they would become relevant if domestic reforms affected the\r\nforeign policy status quo (Ambrosio, 2009; Giragosian, 2017; Jackson, 2010). For example, relations with\r\nAzerbaijan would become relevant if administrative reforms touched on the status of Nagorno-Karabakh.\r\nThey could also become relevant if administrative reforms touched on the postal service or\r\ntelecommunications infrastructure, given that both of these are severely limited between Armenia and\r\n30 Citations in the text are included for those who wish to read further on these subjects. Wherever possible, we\r\nhave sought to identify recent surveys of the literature on a subject that are accessible to those without advanced\r\ntraining. But readers should be aware that some of the papers cited here are rather technical and full of jargon.\r\nUSAID.GOV                                                       GOVERNANCE IN ARMENIA: AN EVIDENCE REVIEW         | 30\r\n"                                                                                                                                                                                                                                                                                                                                                                                                                                                                                                                                                                                                                                                                                                                                                                                                                                                                                                                                                                                                                                                                                                                                                  
## [34] "Azerbaijan. Similarly, the closed border between Armenia and Turkey could become relevant if, for\r\nexample, decentralization reforms affect the status quo administration of the border (Tocci 2007).\r\nRussia has also shown itself willing to intervene in the post-Soviet region when it sees its interests as\r\nthreatened (Zimmerman, 2014). We see no reason to expect domestic governance reform in Armenia to\r\ntrigger military intervention (Way, 2015). However, Russia has intervened in the domestic politics of its\r\nneighbors in response to various triggers, including debate over the Russian language (Goetz, 2017;\r\nSaari, 2014). Armenian is the sole state language in the country, but Russian remains the most common\r\nforeign language. Consistent with this mindset, it is possible that Russia would intervene diplomatically if,\r\nfor example, reforms had the effect of increasing English usage (Cheskin et al., 2018; Giragosian, 2015;\r\nToomet, 2011). The effect of Russia's foreign policy in recent years has been to claim authority over\r\n\"compatriots abroad,\" which includes non-ethnic Russian speaking individuals (Laruelle, 2015). This\r\npossibility could have an effect on agenda-setting, in that Armenian policymakers may de-prioritize\r\nreforms with implications for the official or unofficial status of foreign languages.\r\nTHE PRESS\r\nThe press plays a key role in democracy and in anti-corruption efforts (Stanig, 2015). While we think that\r\nattention to the media is critical, we refer readers to the Civil Society and Media Evidence Review being\r\nconducted separately under this tasking (USAID, 2019a).\r\n  2.6.       CONCLUSION\r\nWe provided a survey of the evidence on governance, with specific application to Armenia in the wake of\r\nits recent transition. We considered a broad array of factors that promote or hinder good governance,\r\ndistilling the lessons learned to a few key factors. We discussed a number of factors, summarized below,\r\nand also point to the other two evidence reviews that will be forthcoming — Civil Society and Media as\r\nwell as Integrity Systems and the Rule of Law — for discussion of what may also be key issues in\r\nstrengthening governance.\r\nWe note that on a structural level, Armenia is in a relatively good position, having recently adopted\r\ninstitutions (parliamentary and proportional representation with thresholds) that the literature indicates are\r\nimportant factors governance and democratization. These institutional factors are difficult to affect,\r\nespecially for international assistance providers, but it is nonetheless important to recognize the importance\r\nof maintaining such institutions, and providing whatever support may be possible or necessary, whatever\r\nfollows in the coming years.\r\nTurning to factors that may be more easily manipulated in the short- to medium-term, both by the\r\ngovernment or international assistance providers, we identify lessons learned:\r\n• Consolidation of State Institutions\r\n    - Exercise caution in moving forward with decentralization. There is cause for concern about the\r\n        efficacy of decentralization\r\n    - Address petty corruption, and carefully phase-in efforts to address higher-level corruption\r\n    - Institute measures such as e-governance in an effort to streamline many institutional functions\r\nUSAID.GOV                                                     GOVERNANCE IN ARMENIA: AN EVIDENCE REVIEW   |  31\r\n"                                                                                                                                                                                                                                                                                                                                                                                                                                                                                                                                                                                                                                                                                                                                                                                                                                                                                                                                                                                                                                                                                                                                                                                                                                                                                                                                                                                                                                                                                                                                                                                             
## [35] "   - Good institutional reform does not entail exporting the right strategy from one context and using\r\n      it in another one. Instead, good institutional reform first builds off robust political economy\r\n      analysis to inform feasible action. Robust political economy analysis should not only make use of\r\n      traditional indicators but also the dimensions of the political settlements framework, paying\r\n      particular attention to the implementation concerns stressed by the Problem-Driven Iterative\r\n      Adaptation (PDIA) approach. Thus, institutional reform is not about grand plans or overarching\r\n      solutions but about gradually undertaking reforms that are suitable to particular contexts.\r\n• Service Delivery\r\n   - Participatory programs are likely to be beneficial both for governance and citizen-level satisfaction\r\n      with democracy. Promising potential interventions include opening more citizen service centers\r\n      and improving the efficacy of existing ones, as well as perhaps initiating some combination of the\r\n      following: participatory/open budgeting, right to information laws, grievance redress mechanisms,\r\n      hotlines, social audits, and social programs. All potential interventions should have some form of\r\n      an e-governance component.\r\n• Civil Service Reform\r\n   - Armenia should be cautious about following Georgia’s “big bang” approach to civil service reform\r\n      especially given former Georgian President Mikhail Saakashvili’s failure to adequately tackle\r\n      corruption when he subsequently served as Governor of Odessa and advisor to President\r\n      Poroshenko in Ukraine.\r\n   - It could also pursue some relatively new interventions, including random inspections for petty\r\n      corruption, more e-government (e.g., biometric smart cards), testing for pro-social motivations\r\n      when hiring (but quietly), performance-based postings, and transparency in hiring and civil\r\n      servants’ tax records.\r\n• Comparisons and Process\r\n   - Regional comparisons are helpful but need to be considered carefully. Georgia, for example, offers\r\n      a model for e-governance reforms and addressing petty corruption, but is likely different than\r\n      Armenia concerning timing and scale.\r\n   - Sequencing is critical, but the literature offers little guidance on optimal sequencing. With that\r\n      said, reforms are likely to be most successful if they proceed at a moderate pace and do not\r\n      alienate important players even if those important players ultimately impede efforts to achieve\r\n      good governance.\r\nUSAID.GOV                                                   GOVERNANCE IN ARMENIA: AN EVIDENCE REVIEW   | 32\r\n"                                                                                                                                                                                                                                                                                                                                                                                                                                                                                                                                                                                                                                                                                                                                                                                                                                                                                                                                                                                                                                                                                                                                                                                                                                                                                                                                                                                                                                                                                                                                                                                                                                                                                                                                                                                                                                                                                                                                                                                                                                                                                                                                                                                                                                                                                                                                                      
## [36] "  3.      ANALYSIS OF ARMENIAN V-DEM GOVERNANCE\r\n          INDICATORS\r\n  3.1.       INTRODUCTION\r\nThis section presents a V-Dem analysis report on a number of factors related to Armenian democracy\r\nand governance over time and in regional context. We begin with summary measures for the key principles\r\nof democracy as well as a measure of corruption. We then turn to a series of time plots on overall\r\ndemocracy with sub-measures, judicial indices, liberal democracy indices, and corruption indices. We\r\nthen show some results for Armenia relative to other regional cases. Finally, since all of these results rely\r\non a single data set, we show that V-Dem tracks another prominent democracy measure, Polity, but\r\nprovide significantly more nuance, which lends more confidence to this study’s focus on V-Dem data.\r\nV-Dem is a database assembled by a worldwide team of professional social scientists. Indicators are\r\nbased on factual information, such as government records, where appropriate. The indicators we review\r\nhere draw heavily on subjective assessments made by country experts; typically five country experts\r\ncontribute to each rating. V-Dem does not document experts’ rationale for indicator changes; in what\r\nfollows, we infer explanations for changes based on events at the time. V-Dem indicators are intended\r\nto be comparable across countries and over time, although the reader should acknowledge that there is\r\na margin of error around any given point estimate. In general, higher numbers indicate more democratic\r\npractices, although important exceptions are noted in the text below.\r\n  3.2.       OVERALL MEASURES FOR ARMENIA\r\nFigure 5 summarizes data on different components of democracy in Armenia over time, with particular\r\npoints documenting 2002, 2007, 2012, and 2017. The overlap between polygons indicates a lack of change in\r\nmeasures on each indicator over time. Six concepts are presented. For the five democracy indicators, each\r\nindicator ranges from 0 to 1, where 0 represents a less democratic and 1 represents a more democratic\r\nscore. At the northern point of the figure is Electoral Democracy (index of freedom of association; clean\r\nelections; freedom of expression;          Figure 5. Armenia: Principles of Democracy Indices & Corruption\r\nelected executive; and suffrage); this\r\nwas somewhat higher in 2002 but has\r\ngenerally been stable since then.\r\nNortheast is Liberal Democracy\r\n(index of equality before the law and\r\nindividual liberties; judicial constraints\r\non the executive; and legislative\r\nconstraints on the executive); this has\r\nbeen very stable over time. Southeast\r\nis Deliberative Democracy (index of\r\nmeasures of how political elites\r\nreason on and justify public policy\r\nand engage in consultation); this was\r\nhigher in 2017. South is Egalitarian\r\nDemocracy (index based on measures\r\nof equal rights/freedoms for all people\r\nand equal distribution of resources\r\nUSAID.GOV                                                 GOVERNANCE IN ARMENIA: AN EVIDENCE REVIEW     |  33\r\n"                                                                                                                                                                                                                                                                                                                                                                                                                                                                                                                                                                                                                                                                                                                                                                                                                                                                                                                                                                                                                                                                                                                                                                                                                                                                                                                                                                                                                                                                                                                                                                                                                                                                                                                                                                                                                                                                                                                                                                                    
## [37] "across all social groups); this was highest in 2002 but then eroded over time. The change in this index indicates\r\nthat V-Dem experts evaluated that equal protection before the law and equal access to resources eroded over\r\ntime. For context, the magnitude of the decrease in the V-Dem measure of Egalitarian Democracy from 2007-2017\r\nwas on par with simultaneous decreases in Ukraine and Russia (V-Dem Annual Democracy Report 2018).\r\nSouthwest is Participatory Democracy (index of civil society participation, citizen initiatives, and direct\r\nvoting for officials at all levels of government). This was low and stable throughout much of the period and\r\nthen notably increased in 2017. This reflects the increase in civil society activity and citizen participation\r\nthat took place in the context of Armenia’s political transition. Northwest is Corruption (index of\r\nlegislative, judicial, executive, and public-sector corruption measures). This ranges from 0, or no corruption,\r\nto 1, or total corruption. This measure has been consistently high in Armenia over time, although V-Dem\r\nexperts identified an improvement in corruption over the 2007 to 2017 period (as indicated by the decrease\r\nin the corruption measure).\r\nFigure 6 displays the overall measure of electoral democracy (northern point in Figure 5) in V-Dem, and\r\nit breaks down the sub-indices that contribute to this overall measure. We plot scores for 1990-2017.\r\nFor all measures, higher values indicate more democratic outcomes. Electoral Democracy (also known\r\nas “polyarchy,” Dahl, 1973)\r\nintends to record the                      Figure 6. Armenia: Democracy with Sub-Measures, Corruption, and\r\nresponsiveness of rulers to                Violence\r\ncitizens, when this is “achieved\r\nthrough electoral competition for\r\nthe electorate’s approval under\r\ncircumstances when suffrage is\r\nextensive; political and civil society\r\norganizations can operate freely;\r\nelections are clean and not marred\r\nby fraud or systematic\r\nirregularities; and elections affect\r\nthe composition of the chief\r\nexecutive of the country”\r\n(Coppedge et al., 2018). Overall,\r\nElectoral Democracy was high\r\nimmediately following Armenia’s\r\nindependence from the Soviet\r\nUnion, but it fell meaningfully by\r\nthe mid-1990s. It hit new lows in\r\nthe mid-2000s although rose again\r\nin recent years. Taking into\r\naccount margins of error around\r\neach point estimate, the major\r\ntakeaway is that Electoral\r\nDemocracy by 2017 was lower\r\nthan levels in the early 2000s.\r\nIn Figure 6, the sub-indices that\r\ncontribute to the overall electoral\r\nUSAID.GOV                                                     GOVERNANCE IN ARMENIA: AN EVIDENCE REVIEW       |  34\r\n"                                                                                                                                                                                                                                                                                                                                                                                                                                                                                                                                                                                                                                                                                                                                                                                                                                                                                                                                                                                                                                                                                                                                                                                                                                                                                                                                                                                                                                                                                                                                                                                                                                                                                                                                                                                                                                                                                                                                                                                                                                                                                                                                                                                                                                                                                                                                   
## [38] "democracy score provide further context on the overall trend. Taken together, these variables measure\r\npolitical participation, the strength of rule of law and electoral institutions, and the threat of physical\r\nviolence. First, focus on the fact that this group of indicators are all around the same level on the overall\r\nscale from 0 (worse outcome) to 1 (better outcome). This suggests that the various aspects of political\r\nand civil life relevant to democracy in Armenia are all varying around a low beginning baseline. There are\r\nthree notable exceptions. First, Electoral Contestation was extremely high at independence, in the midst\r\nof undeveloped political parties, but it dropped significantly by the early 2000s. Second, Accountability\r\nfollowed the same trend. Accountability captures “constraints on the government’s use of power\r\nthrough requirements for justification for its actions and potential sanctions” (V-Dem Codebook). This\r\nincludes accountability through elections, checks and balances between institutions, and oversight by civil\r\nsociety and media. Armenia’s overall decline in electoral democracy from independence to the early\r\n2000s tracks the regime’s ability to limit these forms of accountability. The third exception is Civil\r\nLiberties, which includes “the absence of physical violence committed by government agents and the\r\nabsence of constraints of private liberties and political liberties by the government” (V-Dem Codebook).\r\nSince independence, Civil Liberties have been notably higher in Armenia than the other aspects of\r\npolitical life considered here, and Civil Liberties have been generally increasing over time. This increase\r\nparallels the indicator for Less Physical Violence, which indicates a considerably lower threat of physical\r\nviolence around 2010. (For definitions of other variables, see V-Dem Codebook.)\r\nFigure 7. Armenia: Judicial Democracy, with Sub-Measures                         3.3. JUDICIARY\r\n                                                                                 Figure 7 plots Armenia’s\r\n                                                                                 judicial scores for 1990-2017.\r\n                                                                                 Please see descriptions below\r\n                                                                                 to interpret high versus low\r\n                                                                                 scores. In general, V-Dem\r\n                                                                                 experts consider all courts in\r\n                                                                                 the judicial system at every\r\n                                                                                 level. The overall Fewer\r\n                                                                                 Judicial Constraints score in\r\n                                                                                 Figure 7 is based on experts’\r\n                                                                                 evaluations of the extent to\r\n                                                                                 which political actors respect\r\n                                                                                 the constitution and comply\r\n                                                                                 with court decisions, as well\r\n                                                                                 as the independence of the\r\n                                                                                 courts from political\r\n                                                                                 interference. In Armenia,\r\n                                                                                 there are constraints on the\r\n                                                                                 judiciary, and those\r\n                                                                                 constraints have been quite\r\n                                                                                 constant since independence.\r\n                                                                                 Meaningful movement has\r\n                                                                                 occurred in related judicial\r\n                                                                                 quality measures, however.\r\n                                                                                 First, note the variation in\r\n                                                                                 Judicial Purges, or whether\r\nUSAID.GOV                                                   GOVERNANCE IN ARMENIA: AN EVIDENCE REVIEW       |  35\r\n"
## [39] "judges were removed from their posts without cause. This was a particular problem around 2005-2010\r\n(lower values on Fewer Judicial Purges) but has been less of a problem in recent years. Second, note the\r\nvariation in the Less Judicial Reform measure. This marks periods in which more judicial reform was\r\nundertaken (negative values), specifically in the years after 1995 and 2010. However, these reforms did\r\nnot improve or change the overall quality of the judiciary in terms of Fewer Judicial Constraints. Third,\r\nnote the non-democratic trend in the Less Judicial Review measure over time; as of 2017, judicial review\r\nis particularly absent. The other indicators in Figure 7 capture government attacks on the judiciary,\r\ngovernment compliance with judicial rulings, and judicial accountability. While there has been some\r\nvariation in these measures, they have not followed a distinct pattern over time, and values in 2017 are\r\naround where they were in the mid 1990s. (For definitions of these variables, see V-Dem Codebook.) In\r\nsum, the major takeaway from Figure 7 is that the judiciary in Armenia has been marked by stability and\r\nhas not undergone meaningful reform that has removed constraints.\r\n  3.4.      LIBERAL DEMOCRACY\r\nFigure 8 plots the trajectory of Liberal Democracy in Armenia, from 1990-2017. Higher values indicate\r\nmore democratic outcomes. The Liberal Democracy measure “emphasizes the importance of protecting\r\nindividual and minority rights against\r\n                                         Figure 8. Armenia: Liberal Democracy, with Sub-Measures\r\nthe tyranny of the state and the\r\ntyranny of the majority” (V-Dem\r\nCodebook). Overall, Liberal\r\nDemocracy was high immediately\r\nupon independence but dropped\r\nmeaningfully by around 2000 and\r\ncontinued to decline thereafter. One\r\nkey takeaway is that the overall\r\nmeasure of Liberal Democracy\r\n(bolded line) has been improving\r\nsince 2010; in 2017 it was around the\r\nlevel of the early 2000s. Figure 8\r\nplots Judicial Constraints on the\r\nExecutive and Legislative Constraints\r\non the Executive, which are\r\ncomponents of the overall Liberal\r\nDemocracy measure. These have\r\ntrended similarly to the overall\r\nmeasure, meaning that constraints on\r\nthe executive were high following\r\nindependence and today are at\r\naround the levels they were in 2000.\r\nWhat is particularly notable in Figure\r\n8 is Equality Before the Law:\r\nIndividual Liberty. This indicator\r\nmeasures the extent to which laws\r\ntransparently, rigorously, and\r\nimpartially enforce the law and the extent to which the law protects citizens’ physical integrity rights,\r\nUSAID.GOV                                                  GOVERNANCE IN ARMENIA: AN EVIDENCE REVIEW      | 36\r\n"                                                                                                                                                                                                                                                                                                                                                                                                                                                                                                                                                                                                                                                                                                                                                                                                                                                                                                                                                                                                                                                                                                                                                                                                                                                                                                                                                                                                                                                                                                                                                                                                                                                                                                                                                                                                                                                                                                                                                                                                                                                                                                                                                                                                                                                                                                    
## [40] "property rights, freedom from forced labor, freedom of movement, and freedom of religion. First, the\r\nabsolute values of this indicator have been high in the whole period. This can be interpreted as\r\nArmenia’s scores being better than other countries at comparable levels of overall Liberal Democracy.\r\nFor example, in an analysis of 2017 data, V-Dem places the level of realized political equality in Armenia\r\nin the top 25% of a set of countries with similar overall democracy scores (V-Dem 2018 Democracy\r\nReport, p. 35, Figure 2.2). 31 Second, Individual Liberty has generally improved over time in Armenia,\r\nreaching a new high in 2017. The takeaway is that strong Individual Liberty has been a particularly\r\npositive outcome in Armenia even before its political transition.\r\n  3.5.       CORRUPTION IN ARMENIA\r\nFigure 9 captures corruption in Armenia, 1990-2017. Here, higher values indicate worse outcomes, in\r\nterms of more corruption. We report the overall Political Corruption measure as well as sub measures for\r\nExecutive Corruption and Public-Sector Corruption. The overall measure includes executive (with specific attention to\r\nboth bribery and embezzlement), legislative, and judicial corruption. It intends to cover both petty (low-level) corruption\r\nand grand (large-scale) corruption, bribery Figure 9. Armenia: Corruption, with Sub-Measures\r\nand theft, and corruption intending to\r\ninfluence law making and corruption\r\nintending to influence law implementation.\r\nTo remind the reader, these scores are\r\nbased on input from country experts and\r\nthus draw on more than observable\r\nevents in the news, for example, making\r\nthem particularly valuable. In general,\r\nPolitical Corruption grew sharply\r\nfrom independence until just after\r\n2000, and then leveled off at high\r\nrates. V-Dem experts’ perceptions of\r\nPolitical Corruption did meaningfully\r\ndecline from 2016-2017. The trend\r\nin Executive Corruption parallels the\r\noverall trend, including the decline\r\nfrom 2016-2017. While in the years\r\nimmediately after independence,\r\nPublic Sector Corruption was higher\r\nthan Executive Corruption, since\r\naround 2000 Public Sector\r\nCorruption has been meaningfully\r\nlower. It has also been declining in\r\nthe period from around 2007 to\r\n2017 and not just in the 2016-2017\r\nyear. As a point of comparison,\r\ncorruption levels today on all\r\nindicators are at or above where they were around 2000.\r\n31 For context, other countries in this set include Egypt (in the lowest 25% of de facto political equality) and Singapore (in the\r\nhighest 25% of de facto political equality).\r\nUSAID.GOV                                                             GOVERNANCE IN ARMENIA: AN EVIDENCE REVIEW                |   37\r\n"                                                                                                                                                                                                                                                                                                                                                                                                                                                                                                                                                                                                                                                                                                                                                                                                                                                                                                                                                                                                                                                                                                                                                                                                                                                                                                                                                                                                                                                                                                                                                                                                                                                                                                                                                                                                                                                                                                                                                                                                                                                                                                                                                                                                                                                                                                        
## [41] "  3.6.      DEMOCRACY & CORRUPTION ACROSS COUNTRIES\r\nFigure 10 compares Armenia with other countries in the post-Soviet space that experienced a political\r\ntransition – by which we mean a widescale, popular protest movement that led to significant electoral\r\nchanges. We compare each country in the year before the political transition to provide context on\r\ndemocratic indicators and corruption in the environment immediately preceding the political transition.\r\nThe significant overlap across the polygons suggests similarities in the characteristics of democracy in these\r\ncountries in the year before their respective political transitions. The indicators are the same as those in\r\nFigure 1. Armenia is notable for being in the “middle of the pack” on most of these indicators; it is at or\r\nbelow the different democracy measures in Moldova 2008 and Ukraine 2003, but above Georgia 2002 and\r\nKyrgyzstan 2004. When it comes to corruption, the level in Armenia 2017 is significantly lower than it was in\r\nGeorgia 2002, Ukraine 2003, or Kyrgyzstan 2004; it is about on par with Moldova 2008.\r\n          Figure 10. Armenia vs. Regional Comparisons: Principles of Democracy Indices &\r\n          Corruption\r\nFigure 11 compares Armenia with other countries in the post-Soviet space that experienced political\r\ntransitions by tracking their scores on democracy over time, from 1990–2017. Years are on the x-axis, and\r\noverall standardized democracy scores are on the y-axis. This is again the Electoral Democracy score, an\r\noverall measure of electoral democracy and democratic practices (see discussion above). The black\r\nshapes on the graph indicate the year in which their political transitions took place. We can see that the\r\nUSAID.GOV                                                   GOVERNANCE IN ARMENIA: AN EVIDENCE REVIEW        | 38\r\n"                                                                                                                                                                                                                                                                                                                                                                                                                                                                                                                                                                                                                                                                                                                                                                                                                                                                                                                                                                                                                                                                                                                                                                                                                                                                                                                                                                                                                                                                                                                                                                                                                                                                                                                                                                                                                                                                                                                                                                                                                                                                                                                                                                                                                                                                                                                                                                                                                                                                                                                                                                                                                                                                                                                                                                                                                                                                                                                                                                                                                                                                                                                                                                                                                                                                         
## [42] "Electoral Democracy score               Figure 11. Armenia and Regional Countries: Comparative Scores on\r\nincreases in each of Georgia,           Democracy\r\nKyrgyzstan, Ukraine, and Moldova\r\nafter their political transitions. The\r\nincrease only lasts a few years in\r\nMoldova and Ukraine, with\r\nUkraine’s score declining to around\r\npre-transition levels. Kyrgyzstan’s\r\nscore increases only slightly\r\nfollowing its political transition. For\r\ncomparison, political\r\ndevelopments in Kyrgyzstan that\r\ntook place within the structure of\r\nthat country’s institutions (and\r\nthus not in response to a political\r\ntransition related protest\r\nmovement) accounted for a\r\ngreater amount of democratic\r\nimprovement since around 2010.\r\nThe takeaway is that political\r\ntransitions have correlated with\r\nlarge improvements in democracy\r\nin general, but not in all cases, and\r\nthose improvements are not\r\nstable over time. Armenia’s\r\nElectoral Democracy trend is\r\nincluded for context. In 2017,\r\nbefore its political transition,\r\nArmenia’s Electoral Democracy\r\nscore was at or lower than the\r\nlevel in Georgia, Ukraine, and Moldova at the time of their political transitions (and higher than\r\nKyrgyzstan’s was at the time).\r\nFigure 12 again compares Armenia with other countries in the post-Soviet space that experienced a political\r\ntransition by tracking their scores on Political Corruption over time, from 1990-2017. Years are on the x-\r\naxis and Political Corruption scores are on the y-axis. Political Corruption captures both “petty” and\r\n“grand” corruption, including both bribery and theft, that influences law making and implementation (see\r\nagain Figure 9). The black shapes on the graph indicate the year in which the political transition took\r\nplace. The year in which the political transition took place in Kyrgyzstan and Ukraine does not precede\r\na notable improvement in the overall trend in either country. Political Corruption meaningfully declined\r\nfollowing events in Moldova in 2009, although it is now back at its previous level. Most notably, Political\r\nCorruption declined dramatically following Georgia’s political transition. Corruption in Georgia in recent\r\nyears has hovered around the level it dropped to following its transition, but that drop was so\r\nmeaningfully large as to put it on a totally different trajectory than the other political transition countries\r\nin the figure. The Georgian case gives a proof-of-concept that significant anti-corruption gains are\r\npossible in the immediate aftermath of a political transition, and those gains need not erode away over\r\nUSAID.GOV                                                   GOVERNANCE IN ARMENIA: AN EVIDENCE REVIEW       | 39\r\n"                                                                                                                                                                                                                                                                                                                                                                                                                                                                                                                                                                                                                                                                                                                                                                                                                                                                                                                                                                                                                                                                                                                                                                                                                                                                                                                                                                                                                                                                                                                                                                                                                                                                                                                                                                                                                                                                                                                                                                                                                                                                                                                                                                                                                                                                                                                            
## [43] "time. Armenia’s Political Corruption trend is included for context; it is among the lowest until the mid-\r\n2000s, and by 2017 its corruption level is clustered with the four countries other than Georgia.\r\n             Figure 12. Armenia and Neighbors: Comparative Scores on Corruption\r\n  3.7.     COMPARING V-DEM TO POLITY (WITH FREEDOM HOUSE\r\n           IMPUTATION)\r\nFigure 13 compares the V-Dem Electoral Democracy score for Armenia (1990–2017) to an alternative\r\nmeasure of democracy, which is a standardized measure from Polity and Freedom House. While both\r\nindices note a change in the years after 1995, the relative size of that change — both the drop-off and\r\nthe subsequent increase — is larger in the Polity indicator. Coming to recent years, the Polity indicator\r\ndeclines substantially from 2016 to 2017. The V-Dem indicator also shows a decline, but a much subtler\r\none. The V-Dem indicator places the score in 2017 around the level of the score in 2013, and still higher\r\nthan scores since around 2003; the Polity indicator drops the score to significantly lower than that\r\nrecorded in 2003. In our view, the V-Dem indicator levels and trends are more useful than the Polity\r\nindicator, because the V-Dem indicator is based on a wide variety of sub-indicators reviewed here that\r\nUSAID.GOV                                                  GOVERNANCE IN ARMENIA: AN EVIDENCE REVIEW      | 40\r\n"                                                                                                                                                                                                                                                                                                                                                                                                                                                                                                                                                                                                                                                                                                                                                                                                                                                                                                                                                                                                                                                                                                                                                                                                                                                                                                                                                                                                                                                                                                                                                                                                                                                                                                                                                                                                                                                                                                                                                                                                                                                                                                                                                                                                                                                                                                                                                                                                                                                                                                                                                                                                                                                                                                                                                                                                                                                                                                                                                                                                                                                                                                                                                                                                                                                                                                                                                                                                                                                                                                                                                                                                                                                                                                                                     
## [44] "allow it to capture nuance missing      Figure 13. Comparing V-Dem to Polity\r\nfrom the Polity indicator. In the\r\ncontext of Armenia, that nuance\r\nreveals that the Polity indicator\r\noverexaggerated changes in\r\ndemocracy in Armenia, particularly\r\nin the late 1990s and since 2015.\r\n  3.8.      CONCLUSION\r\nWe emphasize several key\r\ntakeaways from this analysis of V-\r\nDem indicators. First, in recent\r\nyears Armenia’s democracy scores\r\nhave been relatively stable and\r\nnear, albeit generally lower, than\r\nthose of other political transition\r\ncountries in the region. As a\r\ncomponent of this, we highlight the\r\nstability and low level of Armenia’s\r\nscores when it comes to judicial\r\nqualities. A notably good outcome\r\nin Armenia is the high level of\r\nindividual liberty in terms of\r\nindividual equality before the law\r\nand, particularly, the extent to\r\nwhich the law protects citizens\r\nfrom violence. We also highlight\r\nthat corruption in Armenia has\r\nbeen high and stable since the\r\n2000s, although there was a meaningful decline in 2016-2017, and public-sector corruption is lower and\r\nhas seen more improvement than executive corruption. As of 2017 Armenia has similar levels of\r\ncorruption to Ukraine, Moldova, and Kyrgyzstan, but the corruption in this group is of a meaningfully\r\ndifferent scale than that in Georgia since its political transition.\r\nUSAID.GOV                                                   GOVERNANCE IN ARMENIA: AN EVIDENCE REVIEW | 41\r\n"                                                                                                                                                                                                                                                                                                                                                                                                                                                                                                                                                                                                                                                                                                                                                                                                                                                                                                                                                                                                                                                                                                                                                                                                                                                                                                                                                                                                                                                                                                                                                                                                                                                                                                                                                                                                                                                                                                                                                                                                                                                                                                                                                                                                                                                                                                                                                                                                                                                                                                                                                                                                                                                                                                                                                                                                                                                                                                                                                                                                                                                                                                                                                                                                                                                                                                                                                                                                                                                                                                                                                                                                                                          
## [45] " 4.       REFERENCES\r\nAcemoglu, Daron, and Matthew O. Jackson. 2017. “Social Norms and the Enforcement of Laws.” Journal\r\n      of the European Economic Association 15(December 2017): 245–95.\r\nAcemoglu, Daron, Simon Johnson, and James A. Robinson. 2002. “Reversal of Fortune: Geography and\r\n         Institutions in the Making of the Modern World Income Distribution.” Quarterly Journal of\r\n         Economics 117(4): 1231–94.\r\nAcemoglu, Daron, Simon Johnson, and James A. Robinson. 2005. “Institutions as a Fundamental Cause of\r\n      Long-Run Growth.” Handbook of Economic Growth 1(SUPPL. PART A): 385–472.\r\nAcemoglu, Daron and James A. Robinson. 2006. Economic Origins of Dictatorship and Democracy.\r\n         Cambridge: Cambridge University Press.\r\nAcemoglu, Daron, and James A. Robinson. 2012. Why Nations Fail: The Origins of Power, Prosperity, and\r\n      Poverty. New York: Crown Business.\r\nAdes, Alberto, and Edward L. Glaeser. 1995. “Trade and Circuses: Explaining Urban Giants.” Quarterly\r\n         Journal of Economics 110(1): 195–227.\r\nAgarwal, Sanjay, David Post, and Varsha Venugopal. 2013. Citizen Report Cards: Monitoring Citizen\r\n         Perspectives to Improve Service Delivery. Washington, DC.\r\nAlesina, Alberto et al. 2003. “Fractionalization.” Journal of Economic Growth 8: 155–94.\r\nAlesina, Alberto, Arnaud Devleeschauwer, William Easterly, Sergio Kurlat and Romain Wacziarg. 2003.\r\n         “Fractionalization.” Journal of Economic Growth 8:155–194.\r\nAlesina, Alberto, Reza Baqir and William Easterly. 1999. “Public Goods and Ethnic Divisions.” Quarterly\r\n         Journal of Economics 114(4):1243–1284.\r\nAlesina, Alberto, Stelios Michalopoulos, and Elias Papaioannou. 2016. “Ethnic Inequality.” Journal of\r\n         Political Economy 124(2): 428–88.\r\nAmbrosio, T. 2009. Authoritarian Backlash. London: Routledge.\r\nAnderson, Terry. 2018. Local Government Anti-Corruption Initiative in Post-Soviet Georgia and\r\n         Ukraine. In The Routledge Handbook of International Local Government, eds. Richard Kerley, Joyce\r\n         Little, and Pamela Dunning. London and New York: Routledge, 435–49.\r\nAndreasyan, Zhanna and Georgi Derluguian. 2015. “Fuel Protests in Armenia.” New Masses. 11:1–20.\r\nAndrews, Matt. 2015. “Explaining Positive Deviance in Public Sector Reforms in Development.” World\r\n      Development 74: 197–208.\r\nAndrews, Matt, Lant Pritchett, and Michael Woolcock. 2013. “Escaping Capability Traps Through\r\n         Problem Driven Iterative Adaptation (PDIA).” World Development 51: 234–44.\r\nUSAID.GOV                                                  GOVERNANCE IN ARMENIA: AN EVIDENCE REVIEW  |   42\r\n"                                                                                                                                                                                                                                                                                                                                                                                                                                                                                                                                                                                                                                                                                                                                                                                                                                                                                                                                                                                                                                                                                                                                                                                                                                                                                                                                                                                                                                                                                                                                                                                                                                                                                                                                                                                                                                                                                                                                                                                                                                                                                                                                                                                                                                                                                                                                                                                                                
## [46] "Andrews, Matt, Lant Pritchett and Michael Woolcock. 2017. Building State Capability: Evidence, Analysis,\r\n         Action. New Haven, Connecticut: Oxford University Press.\r\nAshraf, Nava, Oriana Bandiera, and B. Kelsey Jack. 2014. “No Margin, No Mission? A Field Experiment\r\n         on Incentives for Public Service Delivery.” Journal of Public Economics 120: 1–17.\r\nAvis, Eric, Claudio Ferraz and Frederico Finan. 2018. “Do Government Audits Reduce Corruption?\r\n         Estimating the Impacts of Exposing Corrupt Politicians.” Journal of Political Economy 126(5):1912–\r\n         1964.\r\nBai, Ying, and Ruixue Jia. 2016. “Elite Recruitment and Political Stability: The Impact of the Abolition of\r\n         China’s Civil Service Exam.” Econometrica 84(2): 677–733.\r\nBallard-Rosa, Cameron. 2016. “Hungry for Change: Urban Bias and Autocratic Sovereign Default.”\r\n         International Organization 70(02): 313–46.\r\nBanerjee, Abhijit V. et al. 2010. “Pitfalls of Participatory Programs: Evidence from a Randomized\r\n         Evaluation in Education in India.” American Economic Journal: Economic Policy 2(1): 1–30.\r\nBanerjee, Abhijit V., Rema Hanna, Jordan Kyle, Benjamin A. Olken and Sudarno Sumarto. 2018. “Tangible\r\n         Information and Citizen Empowerment: Identification Cards and Food Subsidy Programs in\r\n         Indonesia.” Journal of Political Economy 126(2):451–491.\r\nBanful, Afua Branoah. 2011. “Do Formula-Based Intergovernmental Transfer Mechanisms Eliminate\r\n         Politically Motivated Targeting? Evidence from Ghana.” Journal of Development Economics\r\n         96(2):380–390.\r\nBardhan, Pranab. 2002. “Decentralization of Governance and Development.” Journal of Economic\r\n         Perspectives 16(4): 185–205.\r\nBardhan, Pranab, and Dilip Mookherjee. 2000. “Capture and Governance at Local and National Levels.”\r\n         American Economic Review: Papers and Proceedings 90(2): 135–39\r\nBardhan, Pranab, and Dilip Mookherjee.. 2006. “Pro-Poor Targeting and Accountability of Local\r\n         Governments in West Bengal.” Journal of Development Economics 79(2): 303–27.\r\nBates, Robert H. 1981. Markets and States in Tropical Africa: The Political Basis of Agricultural Policies.\r\n         Berkeley, California: University of California Press.\r\nBanuri, Sheheryar, and Philip Keefer. 2016. “Pro-Social Motivation, Effort and the Call to Public\r\n         Service.” European Economic Review 83: 139–64.\r\nBauhr, Monika. 2017. “Need or Greed? Conditions for Collective Action against Corruption.”\r\n         Governance 30(4): 561–81.\r\nBauhr, Monika, and Nicholas Charron. 2018. “Insider or Outsider? Grand Corruption and Electoral\r\n         Accountability.” Comparative Political Studies 51(4): 415–46.\r\nUSAID.GOV                                                    GOVERNANCE IN ARMENIA: AN EVIDENCE REVIEW      | 43\r\n"                                                                                                                                                                                                                                                                                                                                                                                                                                                                                                                                                                                                                                                                                                                                                                                                                                                                                                                                                                                                                                                                                                                                                                                                                                                                                                                                                                                                                                                                                                                                                                                                                                                                                                                                                                                                                                                                                                                                                                                                                                                                                                                                                                                                                  
## [47] "Bauhr, Monika, and Marcia Grimes. 2014. “Indignation or Resignation: The Implications of Transparency\r\n         for Societal Accountability.” Governance 27(2): 291–320.\r\nBeath, Andrew, Fotini Christia, and Ruben Enikolopov. 2017. “Direct Democracy and Resource\r\n         Allocation: Experimental Evidence from Afghanistan.” Journal of Development Economics 124: 199–\r\n         213.\r\nBehuria, Pritish, Lars Buur, and Hazel Gray. 2017. “Studying Political Settlements in Africa.” African Affairs\r\n         116(464): 508–25.\r\nBennet, Richard. 2011. Delivering on the Hope of the Rose Revolution: Public Sector Reform in Georgia, 2004–\r\n         2009. Innovations for Successful Sociceties, Princeton University. Princeton, New Jersey.\r\nBernauer, Thomas and Vally Koubi. 2009. “Effects of Political Institutions on Air Quality.” Ecological\r\n         Economics 68(5):1355–1365.\r\nBersch, Katherine. 2016. “The Merits of Problem-Solving over Powering Governance Reforms in Brazil\r\n         and Argentina.” Comparative Politics 48(2): 205–25.\r\nBersch, Katherine. 2019. When Democracies Deliver: Governance Reform in Latin America. Cambridge:\r\n         Cambridge University Press.\r\nBesley, Timothy, and Maitreesh Ghatak. 2018. “Prosocial Motivation and Incentives.” Annual Review of\r\n         Economics 10: 411–38.\r\nBesley, Timothy, Rohini Pande, and Vijayendra Rao. 2005. “Participatory Democracy in Action: Survey\r\n         Evidence from South India.” Journal of the European Economic Association 3(2): 648–57.\r\nBesley, Timothy and Masayuki Kudamatsu. 2006. “Health and Democracy.” American Economic Review\r\n         96(2):313–318.\r\nBesley, Timothy, and Torsten Persson. 2011. Pillars of Prosperity: The Political Economics of Development\r\n         Clusters. Princeton, New Jersey: Princeton University Press.\r\nBesley, Timothy and Torsten Persson. 2013. “Taxation and Development.” In Handbook of Public\r\n         Economics. Vol. 5 Elsevier B.V. pp. 51–110.\r\nBird, Richard M. and Michael Smart. 2002. “Intergovernmental Fiscal Transfers: International Lessons for\r\n         Developing Countries.” World Development 30(6):899–912.\r\nBizzarro, Fernando, John Gerring, Carl Henrik Knutsen, Allen Hicken, Michael Bernhard, Svend Erik\r\n         Skaaning, Michael Coppedge and Staffan I. Lindberg. 2018. “Party Strength and Economic\r\n         Growth.” World Politics 70(2):275–320.\r\nBjörkman, Martina, and Jakob Svensson. 2009. “Power to the People: Evidence from a Randomized Field\r\n         Experiment on Community-Based Monitoring in Uganda.” Quarterly Journal of Economics 124(2):\r\n         735–69.\r\nUSAID.GOV                                                 GOVERNANCE IN ARMENIA: AN EVIDENCE REVIEW      |  44\r\n"                                                                                                                                                                                                                                                                                                                                                                                                                                                                                                                                                                                                                                                                                                                                                                                                                                                                                                                                                                                                                                                                                                                                                                                                                                                                                                                                                                                                                                                                                                                                                                                                                                                                                                                                                                                                                                                                                                                                                                                                                                                                                                                                                                                                                                                                                                                                               
## [48] "Björkman, Martina, and Jakob Svensson. 2010. “When Is Community-Based Monitoring Effective?\r\n         Evidence from a Randomized Experiment in Primary Health in Uganda.” Journal of the European\r\n         Economic Association 8(2/3): 571–81.\r\nBjörkman, Martina, Damien de Walque, and Jakob Svensson. 2017. “Experimental Evidence on the Long-\r\n         Run Impact of Community-Based Monitoring.” American Economic Journal: Applied Economics 9(1):\r\n         33–69.\r\nBlaydes, Lisa and Mark Andreas Kayser. 2011. “Counting Calories: Democracy and Distribution in the\r\n         Developing World.” International Studies Quarterly 55:887–908.\r\nBlom-Hansen, Jens, Kurt Houlberg, Søren Serritzlew, and Daniel Treisman. 2016. “Jurisdiction Size and\r\n         Local Government Policy Expenditure: Assessing the Effect of Municipal Amalgamation.”\r\n         American Political Science Review 110(04): 812–31.\r\nBobonis, Gustavo J., Luis R. Ca´mara Fuertes and Rainer Schwabe. 2016. “Monitoring Corruptible\r\n         Politicians.” American Economic Review 106(8):2371–2405.\r\nBoffa, Federico, Amedeo Piolatto and Giacomo Ponzetto. 2016. “Political Centralization and\r\n         Government Accountability.” Quarterly Journal of Economics 130(2):381–422.\r\nBrinkerhoff, Derick W., Anna Wetterberg, and Erik Wibbels. 2018. “Distance, Services, and Citizen\r\n         Perceptions of the State in Rural Africa.” Governance 31(1): 103–24.\r\nBrinks, Daniel, Marcelo Leiras and Scott Mainwaring, eds. 2014. Reflections on Uneven Democracies: The\r\n         Legacy of Guillermo O’Donnell. Baltimore, Maryland: Johns Hopkins University Press.\r\nBrollo, Fernanda, Tommaso Nannicini, Roberto Perotti and Guido Tabellini. 2013. “The Political\r\n         Resource Curse.” American Economic Review 103(5):1759–1796.\r\nBueno de Mesquita, Bruce, Alastair Smith, Randolph M. Siverson and James D. Morrow. 2003. The Logic\r\n         of Political Survival. Cambridge: Cambridge University Press.\r\nBuntaine, Mark T., Ryan S. Jablonski, Daniel L. Nielson and Paula M. Pickering. 2018. “SMS Texts on\r\n         Corruption Help Ugandan Voters Hold Elected Councillors Accountable at the Polls.” Proceedings\r\n         of the National Academy of Sciences 115(26):6668–6673.\r\nBurgess, Robin et al. 2015. “The Value of Democracy: Evidence from Road Building in Kenya.” American\r\n         Economic Review 105(6): 1817–51.\r\nBurgess, Robin, Remi Jedwab, Edward Miguel, Ameet Morjaria and Gerard Padro´ i Miquel. 2015. “The Value\r\n         of Democracy: Evidence from Road Building in Kenya.” American Economic Review 105(6):1817–\r\n         1851.\r\nCai, Hongbin, and Daniel Treisman. 2004. “State Corroding Federalism.” Journal of Public Economics 88(3–\r\n         4): 819–43.\r\nUSAID.GOV                                                    GOVERNANCE IN ARMENIA: AN EVIDENCE REVIEW | 45\r\n"                                                                                                                                                                                                                                                                                                                                                                                                                                                                                                                                                                                                                                                                                                                                                                                                                                                                                                                                                                                                                                                                                                                                                                                                                                                                                                                                                                                                                                                                                                                                                                                                                                                                                                                                                                                                                                                                                                                                                                                                                                                                                                                                                                                                                   
## [49] "Cao, Xun and Hugh Ward. 2015. “Winning Coalition Size, State Capacity, and Time Horizons: An\r\n         Application of Modified Selectorate Theory to Environmental Public Goods Provision.”\r\n         International Studies Quarterly 59(2):264–279.\r\nCapoccia, Giovanni. 2016. “When Do Institutions ‘Bite’? Historical Institutionalism and the Politics of\r\n      Institutional Change.” Comparative Political Studies 49(8): 1095–1127.\r\nCapoccia, Giovanni, and R. Daniel Kelemen. 2007. “The Study of Critical Junctures: Theory, Narrative,\r\n         and Counterfactuals in Historical Institutionalism.” World Politics 1(3): 341–69.\r\nCarey, John M. and Matthew Soberg Shugart. 1995. “Incentives to Cultivate a Personal Vote: A Rank\r\n         Ordering of Electoral Formulas.” Electoral Studies 14(4):417–439.\r\nCasey, Katherine. 2018. “Radical Decentralization: Does Community Driven Development Work?” Annual\r\n         Review of Economics 10:139–163.\r\nChang, Eric C. C. and Miriam A. Golden. 2007. “Electoral Systems, District Magnitude, and Corruption.”\r\n         British Journal of Political Science 37(December 2006):115.\r\nCharron, Nicholas, Carl Dahlström, Mihály Fazekas, and Victor Lapuente. 2017. “Careers, Connections,\r\n         and Corruption Risks: Investigating the Impact of Bureaucratic Meritocracy on Public\r\n         Procurement Processes.” Journal of Politics 79(1): 89–104.\r\nChase, Robert S., and Julien Labonne. 2011. “Does Community-Driven Development Enhance Social\r\n         Capital? Evidence from the Philippines.” Journal of Development Economics 96(2011): 348–58.\r\nCheibub, José Antonio, Zachary Elkins, and Tom Ginsburg. 2014. “Beyond Presidentialism and\r\n         Parliamentarism.” British Journal of Political Science 44(03): 515–44.\r\nCheskin, Ammon, and Angela Kachuyevski. 2018. \"The Russian-Speaking Populations in the Post-Soviet\r\n         Space: Language, Politics and Identity.\" (2018): 1-23.\r\nCheung, Hoi Yan and Alex W. H. Chan. 2008. “Corruption across Countries: Impacts from Education and\r\n         Cultural Dimensions.” Social Science Journal 45:223–239.\r\nChong, Alberto, Ana De La O, Dean Karlan, and Leonard Wantchekon. 2015. “Does Corruption\r\n         Information Inspire the Fight or Quash the Hope? A Field Experiment in Mexico on Voter\r\n         Turnout.” Journal of Politics 29(1): 55–71.\r\nCoase, Ronald H. 1960. “The Problem of Social Cost.” Journal of Law and Economics 3(1): 1–44.\r\nCollier, David and Steven Levitsky. 1997. “Democracy with Adjectives: Conceptual Innovation in\r\n         Comparative Research.” World Politics 49(3):430–451.\r\nCoppedge, Michael, John Gerring, Carl Henrik Knutsen, Staffan I Lindberg, Svend-Erik Skaaning, Jan\r\n         Teorell, David Altman, Michael Bernhard, Agnes Cornell and M Steven Fish. 2018. “V-Dem\r\n         Codebook v8.”.\r\nUSAID.GOV                                                     GOVERNANCE IN ARMENIA: AN EVIDENCE REVIEW | 46\r\n"                                                                                                                                                                                                                                                                                                                                                                                                                                                                                                                                                                                                                                                                                                                                                                                                                                                                                                                                                                                                                                                                                                                                                                                                                                                                                                                                                                                                                                                                                                                                                                                                                                                                                                                                                                                                                                                                                                                                                                                                                                                                                                                            
## [50] "Copsey, Nathaniel. 2010. “Ukraine.” In The Colour Revolutions in the Former Soviet Republics, eds.\r\n         Donnacha Ó Beacháin and Abel Polese. New York: Routledge, 30–44.\r\nCruz, Cesi, and Philip Keefer. 2015. “Political Parties, Clientelism, and Bureaucratic Reform.”\r\n      Comparative Political Studies 48(14): 1942–73.\r\nDahl, Robert Alan. 1973. Polyarchy: Participation and Opposition. Yale University Press.\r\nDal Bó, Ernesto, Frederico Finan, and Martín A. Rossi. 2013. “Strengthening State Capabilities: The Role\r\n         of Financial Incentives in the Call to Public Service.” Quarterly Journal of Economics: 1169–1218.\r\nDee, Thomas S. 2004. “Are there Civic Returns to Education?” Journal of Public Economics 88(9-10):1697–\r\n         1720.\r\nDe La O, Ana. 2015. Crafting Policies to Policies to End Policies in Latin America: The Quiet Transformation.\r\n         New York: Cambridge University Press.\r\nDevlin, Matthew. 2009. Seizing the Reform Moment: Rebuilding Georgia’s Police, 2004-2006. Innovations for\r\n         Successful Societies, Princeton University. Princeton, New Jersey.\r\nDhaliwal, Iqbal, and Rema Hanna. 2017. “The Devil Is in the Details: The Successes and Limitations of\r\n         Bureaucratic Reform in India.” Journal of Development Economics 124: 1–21.\r\nDi Puppo, Lili. 2010. “Anti-Corruption Interventions in Georgia.” Global Crime 11(2):220– 236.\r\nDi Tella, Rafael and Ernesto Schargrodsky. 2003. “The Role of Wages and Auditing during a Crackdown on\r\n         Corruption in the City of Buenos Aires.” Journal of Law and Economics 46(1):269–292.\r\nDiamond, Larry, Francis Fukuyama, Donald L. Horowitz and Marc F. Plattner. 2014. “Re-considering the\r\n         Transition Paradigm.” Journal of Democracy 25(1):86–100.\r\nDíaz-Cayeros, Alberto. 2006. Federalism, Fiscal Authority, and Centralization in Latin America. New York:\r\n         Cambridge University Press.\r\nDixit, Avinash. 2018. Anti-Corruption Institutions: Some History and Theory. In Institutions, Governance,\r\n         and the Control of Corruption, ed. Kaushik Basu and Tito Cordella. Cham, Switzerland: Palgrave\r\n         Macmillan chapter 2, pp. 15–50.\r\nDizon-Ross, Rebecca, Pascaline Dupas, and Jonathan Robinson. 2017. “Governance and the Effectiveness\r\n      of Public Health Subsidies: Evidence from Ghana, Kenya and Uganda.” Journal of Public Economics\r\n      156: 150–69.\r\nDollar, David, and Jakob Svensson. 2000. “What Explains the Success or Failure of Structural Adjustment\r\n      Programmes?” Economic Journal 110(466): 894–917.\r\nDonaldson, Dave, and Adam Storeygard. 2016. “The View from Above: Applications of Satellite Data in\r\n         Economics.” Journal of Economic Perspectives 30(4): 171–98.\r\nUSAID.GOV                                                    GOVERNANCE IN ARMENIA: AN EVIDENCE REVIEW      | 47\r\n"                                                                                                                                                                                                                                                                                                                                                                                                                                                                                                                                                                                                                                                                                                                                                                                                                                                                                                                                                                                                                                                                                                                                                                                                                                                                                                                                                                                                                                                                                                                                                                                                                                                                                                                                                                                                                                                                                                                                                                                                                                                                                                                                                                                                                              
## [51] "Drozd, Oleksiy Yriyovych. 2017. “Civil Service Pattern in Germany and Ukraine: A Comparative\r\n          Aspect.” Journal of Advanced Research in Law and Economics VIII(5(27)): 1503–7.\r\nDuflo, Esther, and Rohini Pande. 2007. “Dams.” Quarterly Journal of Economics 122(2): 601–46.\r\nDuflo, Esther, Rema Hanna and Stephen P. Ryan. 2012. “Incentives Work: Getting Teachers to Come to\r\n          School.” American Economic Review 102(4):1241–1278.\r\nEasterly, William. 2006. The White Man’s Burden: Why the West’s Efforts to Aid the Rest Have Done So Much\r\n      Ill and So Little Good. New York: Penguin Books.\r\nEasterly, William, and Ross Levine. 2016. “European Origins of Economic Development.” Journal of\r\n          Economic Growth 21(2): 225–57.\r\nEasterly, William, Jozef Ritzen, and Michael Woolcock. 2006. “Social Cohesion, Institutions, and\r\n          Growth.” Economics and Politics 18(2): 103–20.\r\nEBRD. 2016. Life in Transition Survey. European Bank for Reconstruction and Development.\r\nEscaleras, Monica, Shu Lin and Charles Register. 2010. “Freedom of Information Acts and Public Sector\r\n          Corruption.” Public Choice 145(3):435–460.\r\nEvans, Peter. 1995. Embedded Autonomy: States and Industrial Transformation. Princeton, New Jersey: Princeton\r\n          University Press.\r\nEvans, Peter and James E. Rauch. 1999. “Bureaucracy and Growth: A Cross-National Analysis of the\r\n          Effects of “Weberian” State Structures on Economic Growth.” American Sociological Review\r\n          64(5):748–765.\r\nEvans, Peter, Evelyne Huber, and John D. Stephens. 2017. The Political Foundations of State\r\n          Effectiveness. In States in the Developing World, eds. Miguel Angel Centeno, Atul Kohli, and\r\n          Deborah Yashar. New York: Cambridge University Press, 382–409.\r\nFalkowski, Maciej. 2016. “From Apathy to Nationalist Mobilisation: Politics Makes a Come- back in\r\n          Armenia.” OSW Commentary (215):1–8.\r\nFerraz, Claudio and Frederico Finan. 2008. “Exposing Corrupt Politicians: The Effects of Brazil’s Publicly\r\n          Released Audits on Electoral Outcomes.” Quarterly Journal of Economics 123(2):703–745.\r\nFerraz, Claudio and Frederico Finan. 2011. “Electoral Accountability and Corruption: Evidence from the\r\n          Audits of Local Governments.” American Economic Review 101(4):1274– 1311.\r\nFerraz, Claudio and Frederico Finan. 2018. Fighting Political Corruption: Evidence from Brazil. In\r\n          Institutions, Governance, and the Control of Corruption, ed. Kaushik Basu and Tito Cordella. Cham,\r\n          Switzerland: Palgrave Macmillan chapter 9, pp. 253–284.\r\nUSAID.GOV                                                         GOVERNANCE IN ARMENIA: AN EVIDENCE REVIEW  | 48\r\n"                                                                                                                                                                                                                                                                                                                                                                                                                                                                                                                                                                                                                                                                                                                                                                                                                                                                                                                                                                                                                                                                                                                                                                                                                                                                                                                                                                                                                                                                                                                                                                                                                                                                                                                                                                                                                                                                                                                                                                                                                                                                                                                                                                                                                                                                                                                                    
## [52] "Finan, Frederico, Benjamin A. Olken and Rohini Pande. 2017. The Personnel Economics of the\r\n         Developing State. In Handbook of Field Experiments, ed. Abhijit V. Banerjee and Esther Duflo. Vol. 2\r\n         Amsterdam: Elsevier chapter 6, pp. 467–514.\r\nFish, M. Steven. 2006. “Stronger Legislatures, Stronger Democracies.” Journal of Democracy 17(1):5–20.\r\nFisman, Raymond, and Miriam A. Golden. 2017. Corruption: What You Need to Know. Oxford: Oxford\r\n         University Press.\r\nFisman, Raymond, Florian Schulz and Vikrant Vig. 2014. “The Private Returns to Public Office.” Journal of\r\n         Political Economy 122(4):806–862.\r\nFox, Jonathan A. 1994. “The Difficult Transition from Clientelism to Citizenship: Lessons from Mexico.”\r\n         World Politics 46(02): 151–84.\r\nFox, Jonathan. 2015. “Social Accountability: What Does the Evidence Really Say?” World Development 72:\r\n         346–61\r\nFrisk Jensen, Mette. 2018. The Building of the Scandinavian States: Establishing Weberian Bureaucracy\r\n         and Curbing Corruption from the Mid-Seventeenth to the End of the Nineteenth Century. In\r\n         Bureaucracy and Society in Transition: Comparative Perspectives. Emerald Publishing pp. 179–203.\r\nFukuyama, Francis. 2004. State-Building: Governance and World Order in the 21st Century. Ithaca, New York: Cornell\r\n         University Press.\r\nFukuyama, Francis. 2011. The Origins of Political Order: Prehuman Times to the French Revolution. New York:\r\n         Farrar, Straus and Giroux.\r\nFukuyama, Francis. 2013. “What is Governance?” Governance 26(3):347–368.\r\nFukuyama, Francis. 2014. Political Order and Political Decay: From the Industrial Revolution to the Globalization\r\n      of Democracy. New York: Farrar, Straus and Giroux.\r\nFukuyama, Francis. 2016. “Governance: What Do We Know, and How Do We Know It?” Annual Review\r\n         of Political Science 19(1):89–105.\r\nGehlbach, Scott and Alberto Simpser. 2015. “Electoral Manipulation as Bureaucratic Control.” American\r\n         Journal of Political Science 59(1):212–224.\r\nGehlbach, Scott and Philip Keefer. 2011. “Investment Without Democracy: Ruling-Party\r\n         Institutionalization and Credible Commitment in Autocracies.” Journal of Comparative Economics\r\n         39(2):123–139.\r\nGelman, Andrew, and John B. Carlin. 2017. “Some Natural Solutions to the P-Value Communication\r\n         Problem—and Why They Won’t Work.” Journal of the American Statistical Association 112(519):\r\n         899–901.\r\nUSAID.GOV                                                         GOVERNANCE IN ARMENIA: AN EVIDENCE REVIEW  |  49\r\n"                                                                                                                                                                                                                                                                                                                                                                                                                                                                                                                                                                                                                                                                                                                                                                                                                                                                                                                                                                                                                                                                                                                                                                                                                                                                                                                                                                                                                                                                                                                                                                                                                                                                                                                                                                                                                                                                                                                                                                                                                                                                                                                                                                                                                                                                                                                                                                                                                                                             
## [53] "Gerring, John and Strom C. Thacker. 2004. “Political Institutions and Corruption: The Role of Unitarism\r\n         and Parliamentarism.” British Journal of Political Science 34(2004):295–330.\r\nGerring, John and Strom C. Thacker. 2008. A Centripetal Theory of Democratic Governance. Cambridge: Cambridge\r\n         University Press.\r\nGerring, John and Wouten Veenendaal. 2019. Scale Effects: The Impact of Population on Politics. Cambridge:\r\n         Cambridge University Press.\r\nGingerich, Daniel W. 2013. Political Institutions and Party-Directed Corruption in South America: Stealing for the\r\n         Team. New York: Cambridge University Press.\r\nGiragosian, Richard. 12 August 2015. \"Soft Power in Armenia: Neither Soft, Nor Powerful.\" European\r\n         Council on Foreign Relations Wider Europe Forum.\r\nGiragosian, Richard, \"Armenian-Russian Relations: Diminishing Returns.\" 16 October 2017. Heinrich\r\n         Boell Stiftung, South Caucasus.\r\nGlaeser, Edward L. and Claudia Goldin. 2007. Corruption and Reform: Lessons from America’s Economic History.\r\n         Chicago: University of Chicago Press.\r\nGlaeser, Edward L., Rafael La Porta, Florencio Lopez-de Silanes and Andrei Shleifer. 2004. “Do Institutions\r\n         Cause Growth?” Journal of Economic Growth 9(3):271–303.\r\nGrigoryan, Stepan, and Hasmik Grigoryan. 2017. Transparency and Rule of Law as Key Priorities for\r\n         Armenia. In Eastern Voices: Europe’s East Faces an Unsettled West, eds. Daniel Hamilton and Stefan\r\n         Meister. Washington, DC: Center for Strategic and International Studies, 151–64.\r\nGrossman, Guy, Melina R. Platas, and Jonathan Rodden. 2018. “Crowdsourcing Accountability: ICT for\r\n         Service Delivery.” World Development 112: 74–87.\r\nGoetz, Elias. 2017. “Putin, the State, and War: The Causes of Russia’s Near Abroad Assertion\r\n         Revisited.” International Studies Review (2017) 19, 228–253\r\nGreene, Kenneth F. 2007. Why Dominant Parties Lose: Mexico’s Democratization in Comparative Perspective.\r\n         New York: Cambridge University Press.\r\nGrigoryan, Stepan, and Hasmik Grigoryan. 2017. “Transparency and Rule of Law as Key Priorities for\r\n         Armenia.” In Eastern Voices: Europe’s East Faces an Unsettled West, eds. Daniel Hamilton and\r\n         Stefan Meister. Washington, DC: Center for Strategic and International Studies, 151–64.\r\nGrillos, Tara. 2017. “Participatory Budgeting and the Poor: Tracing Bias in a Multi-Staged Process in\r\n         Solo, Indonesia.” World Development 96: 343–58.\r\nGrindle, Merilee S. 2012. Jobs for the Boys: Patronage and the State in Comparative Perspective. Cambridge,\r\n         Massachusetts: Harvard University Press.\r\nUSAID.GOV                                                     GOVERNANCE IN ARMENIA: AN EVIDENCE REVIEW          | 50\r\n"                                                                                                                                                                                                                                                                                                                                                                                                                                                                                                                                                                                                                                                                                                                                                                                                                                                                                                                                                                                                                                                                                                                                                                                                                                                                                                                                                                                                                                                                                                                                                                                                                                                                                                                                                                                                                                                                                                                                                                                                                                                                                                                                                                                                                                              
## [54] "Gross, Neil and William Martin. 1952. “On Group Cohesiveness.” American Journal of Sociology\r\n        57(6):546–564.\r\nGrossman, Guy and Kristin Michelitch. 2018. “Information Dissemination, Competitive Pressure, and\r\n        Politician Performance between Elections: A Field Experiment in Uganda.” American Political Science\r\n        Review pp. 1–22.\r\nGulzar, Saad, and Benjamin J. Pasquale. 2017. “Politicians, Bureaucrats, and Development: Evidence from\r\n        India.” American Political Science Review 111(1): 162–83.\r\nHabyarimana, James, Macartan Humphreys, Daniel N. Posner, and Jeremy M. Weinstein. 2007. “Why\r\n        Does Ethnic Diversity Undermine Public Goods Provision?” American Political Science Review\r\n        101(04): 709–25.\r\nHale, Henry E. 2015. Patronal Politics: Eurasian Regime Dynamics in Comparative Perspective. New York:\r\n        Cambridge University Press.\r\nHanna, Rema, and Shing Yi Wang. 2017. “Dishonesty and Selection into Public Service: Evidence from\r\n        India.” American Economic Journal: Economic Policy 9(3): 262–90.\r\nHarding, Robin and David Stasavage. 2014. “What Democracy Does (and Doesn’t Do) for Basic Services:\r\n        School Fees, School Inputs, and African Elections.” Journal of Politics 76(01):229–245.\r\nHasnain, Zahid, Nick Manning, and Jan H. Pierskalla. 2014. “The Promise of Performance Pay? Reasons\r\n        for Caution in Policy Prescriptions in the Core Civil Service.” World Bank Research\r\n        Observer 29(2): 235–64.\r\nHausmann, Ricardo, Dani Rodrik, and Andrés Velasco. 2008. “Growth Diagnostics.” In The Washington\r\n      Consensus Reconsidered: Towards a New Global Governance, eds. Narcis Serra and Joseph E. Stiglitz.\r\n      New York: Oxford University Press, 324–55.\r\nHerbst, Jeffrey. 2000. States and Power in Africa: Lessons in Authority and Control. Princeton, New Jersey:\r\n        Princeton University Press.\r\nHicken, Allen, Ken Kollman and Joel W. Simmons. 2016. “Party System Nationalization and the Provision\r\n        of Public Health Services.” Political Science Research and Methods 4(03):573–594.\r\nHidalgo, César A., Bailey Klinger, Albert-László Barabasi, and Ricardo Hausmann. 2007. “The Product\r\n        Space Conditions the Development of Nations.” Science 73(1): 181–93.\r\nHollyer, James R., B. Peter Rosendorff and James Raymond Vreeland. 2018. Information, Democracy, and\r\n        Autocracy: Economic Transparency and Political (In)Stability. Cambridge: Cambridge University Press.\r\nHummel, Calla, John Gerring and Thomas Burt. 2018. “Do Political Finance Laws Reduce Corruption?”.\r\nHumphreys, Macartan, and Jeremy M. Weinstein. 2012. Policing Politicians: Citizen Empowerment and\r\n        Political Accountability. Working Paper.\r\nUSAID.GOV                                                     GOVERNANCE IN ARMENIA: AN EVIDENCE REVIEW    | 51\r\n"                                                                                                                                                                                                                                                                                                                                                                                                                                                                                                                                                                                                                                                                                                                                                                                                                                                                                                                                                                                                                                                                                                                                                                                                                                                                                                                                                                                                                                                                                                                                                                                                                                                                                                                                                                                                                                                                                                                                                                                                                                                                                                                                                                                                            
## [55] "Hunter, Wendy, and Natasha Borges Sugiyama. 2014. “Transforming Subjects into Citizens: Insights from\r\n         Brazil’s Bolsa Família.” Perspectives on Politics 12(4): 829–45.\r\nHyde, Susan D. 2007. “The Observer Effect in International Politics: Evidence from a Natural\r\n         Experiment.” World Politics 60(1):37–63.\r\nHyde, Susan D. 2011. “Catch Us If You Can: Election Monitoring and International Norm Diffusion.”\r\n         American Journal of Political Science 55(2):356–369.\r\nImai, Kosuke, Gary King, and Carlos Velasco Rivera. 2019. “Do Nonpartisan Programmatic Policies Have\r\n         Partisan Electoral Effects? Evidence from Two Large Scale Experiments.” Journal of Politics.\r\nIskandaryan, Aleksandr. 2012. “Armenia Between Autocracy and Polyarchy.” Russian Politics and Law\r\n         50(4):23–36.\r\nIslam, Roumeen. 2006. “Does More Transparency Go Along with Better Governance?” Economics\r\n         and Politics 18(2):121–167.\r\nJackson, Nicole J. \"The Role of External Factors in Advancing Non-Liberal Democratic Forms of Political\r\n         Rule: A Case Study of Russia's Influence on Central Asian Regimes.\" Contemporary Politics 16, no.\r\n         1 (2010): 101-118.\r\nJoshi, Anuradha. 2013. “Do They Work? Assessing the Impact of Transparency and Accountability\r\n      Initiatives in Service Delivery.” Development Policy Review 31(S1): 29–48.\r\nKaufmann, Daniel, Aart Kraay, and Massimo Mastruzzi. 2015. “Worldwide Governance Indicators.”\r\nKeefer, Philip. 2007a. “Clientelism, Credibility, and the Policy Choices of Young Democracies.” American\r\n         Journal of Political Science 51(4):804–821.\r\nKeefer, Philip. 2007b. The Poor Performance of Poor Democracies. In Oxford Handbook of Comparative Politics,\r\n         ed. Carles Boix and Susan C. Stokes. Oxford: chapter 36, pp. 886– 909.\r\nKeefer, Philip and Razvan Vlaicu. 2008. “Democracy, Credibility, and Clientelism.” Journal of Law, Economics,\r\n         and Organization 24(2):371–406.\r\nKeefer, Philip and Stuti Khemani. 2005. “Democracy, Public Expenditures, and the Poor: Understanding\r\n         Political Incentives for Providing Public Services.” World Bank Research Observer 20(1):1–27.\r\nKennedy, Ryan. 2010. “Moldova.” In The Colour Revolutions in the Former Soviet Republics, eds. Donnacha\r\n         Ó Beacháin and Abel Polese. New York: Routledge, 62–81.\r\nKhan, Adnan Q., Asim Ijaz Khwaja and Benjamin A. Olken. 2016. “Tax Farming Redux: Experimental\r\n         Evidence on Performance Pay for Tax Collectors.” Quarterly Journal of Economics pp. 219–271.\r\nKhan, Adnan, Asim Ijaz Khwaja, and Benjamin A. Olken. 2019. “Making Moves Matter: Experimental\r\n         Evidence on Incentivizing Bureaucrats Through Performance-Based Transfers.” American\r\n         Economic Review 109(1): 237–70.\r\nUSAID.GOV                                                     GOVERNANCE IN ARMENIA: AN EVIDENCE REVIEW  |   52\r\n"                                                                                                                                                                                                                                                                                                                                                                                                                                                                                                                                                                                                                                                                                                                                                                                                                                                                                                                                                                                                                                                                                                                                                                                                                                                                                                                                                                                                                                                                                                                                                                                                                                                                                                                                                                                                                                                                                                                                                                                                                                                                                                                                                          
## [56] "Khan, Mushtaq H. 2017. “Introduction: Political Settlements and the Analysis of Institutions.” African\r\n      Affairs 117(469): 636–55.\r\nKhemani, Stuti et al. 2016. Making Politics Work for Development: Harnessing Transparency and Citizen\r\n         Engagement. Washington, DC: World Bank.\r\nKingdon, John W. 1995. Agendas, Alternatives and Public Policies. Ann Arbor, Michigan: University of\r\n      Michigan Press.\r\nKiser, Edgar and Steven M. Karceski. 2017. “Political Economy of Taxation.” Annual Review of Political Science\r\n         20(1):75–92.\r\nKitschelt, Herbert and Steven I. Wilkinson. 2007. Citizen-Politician Linkages: An Introduction. In Patrons,\r\n         Clients, and Policies: Patterns of Democratic Accountability and Political Competition, ed. Herbert Kitschelt\r\n         and Steven I. Wilkinson. New York: Cam- bridge University Press chapter 1, pp. 1–49.\r\nKlomp, Jeroen and Jakob de Haan. 2013. “Political Budget Cycles and Election Outcomes.” Public Choice\r\n         157(1-2):245–267.\r\nKosack, Stephen and Archon Fung. 2014. “Does Transparency Improve Governance?” Annual Review of\r\n         Political Science 17(1):65–87.\r\nKosec, Katrina, and Leonard Wantchekon. 2019. “Can Information Improve Rural Governance and\r\n         Service Delivery?” World Development 115.\r\nKudamatsu, Masayuki. 2012. “Has Democratization Reduced Infant Mortality in Sub- Saharan Africa?\r\n         Evidence from Micro Data.” Journal of the European Economic Association 10(6):1294–1317.\r\nLabonne, Julien, and Robert S. Chase. 2009. “Who Is at the Wheel When Communities Drive\r\n         Development? Evidence from the Philippines.” World Development 37(1): 219–31.\r\nLake, David A. and Matthew A. Baum. 2001. “The Invisible Hand of Democracy: Political Control and the\r\n         Provision of Public Services.” Comparative Political Studies 34(6):587– 621.\r\nLessmann, Christian and Gunther Markwardt. 2010. “One Size Fits All? Decentralization, Corruption, and\r\n         the Monitoring of Bureaucrats.” World Development 38(4):631–646.\r\nLevitsky, Steven, and Lucan A. Way. 2010. Competitive Authoritarianism: Hybrid Regimes after the Cold War.\r\n         New York: Cambridge University Press.\r\nLevitsky, Steven, and Daniel Ziblatt. 2018. How Democracies Die. New York: Crown.\r\nLevy, Brian. 2014. Working with the Grain: Integrating Governance and Growth in Development Strategies.\r\n      New York: Oxford University Press.\r\nLewis, David. 2010. Kyrgyzstan. In The Colour Revolutions in the Former Soviet Republics, eds. Donnacha Ó\r\n         Beacháin and Abel Polese. New York: Routledge, 45–61.\r\nUSAID.GOV                                                      GOVERNANCE IN ARMENIA: AN EVIDENCE REVIEW         |  53\r\n"                                                                                                                                                                                                                                                                                                                                                                                                                                                                                                                                                                                                                                                                                                                                                                                                                                                                                                                                                                                                                                                                                                                                                                                                                                                                                                                                                                                                                                                                                                                                                                                                                                                                                                                                                                                                                                                                                                                                                                                                                                                                                                                                                                                                                                                                                                                                           
## [57] "Lewis, David. 2017. The Contested State in Post-Soviet Armenia. In The Logic of Weak States, ed. John\r\n         Heathershaw and Ed Schatz. Pittsburgh, Pennsylvania: Pittsburgh University Press chapter 8, pp.\r\n         120–135.\r\nLewis-Faupel, Sean, Yusuf Neggers, Benjamin A. Olken and Rohini Pande. 2016. “Can Electronic\r\n         Procurement Improve Infrastructure Provision? Evidence from Public Works in India and\r\n         Indonesia.” American Economic Journal: Economic Policy 8(3):258–283.\r\nLieberman, Evan S. 2009. Boundaries of Contagion: How Ethnic Politics Have Shaped Government Responses to\r\n         AIDS. Princeton, New Jersey: Princeton University Press.\r\nLieberman, Evan S. 2018. The Comparative Politics of Service Delivery in Developing Countries. In\r\n         Handbook on Politics in Developing Countries, eds. Carol Lancaster and Nicolas van de Walle.\r\n         Oxford: Oxford University Press.\r\nLight, Matthew. 2014. “Police Reforms in the Republic of Georgia: The Convergence of Domestic and\r\n         Foreign Policy in an Anti-Corruption Drive.” Policing and Society 24(3):318– 345.\r\nLindberg, Staffan I., Michael Coppedge, John Gerring and Jan Teorell. 2014. “V-Dem: A New Way to\r\n         Measure Democracy.” Journal of Democracy 25(3):159–169.\r\nLindert, Kathy, Anja Linder, Jason Hobbs, and Bénédicte de la Brière. 2007. SP Discussion Paper The\r\n         Nuts and Bolts of Brazil’s Bolsa Família Program: Implementing Conditional Cash Transfers in a\r\n         Decentralized Context. Washington, DC: World Bank.\r\nLinz, Juan J. 1990. “The Perils of Presidentialism Reconsidered.” Journal of Democracy 1(1):51–69.\r\nLinz, Juan J. and Arturo Valenzuela, eds. 1994. The Failure of Presidential Democracy. Baltimore, Maryland:\r\n         Johns Hopkins University Press.\r\nLizzeri, Alessandro and Nicola Persico. 2004. “Why Did the Elites Extend the Suffrage? Democracy and\r\n         the Scope of Government, with an Application to Britain’ s Age of Reform.” Quarterly Journal of\r\n         Economics 119(2):707–765.\r\nLucas, Robert E. 2015. “Human Capital and Growth.” American Economic Review: Papers and Proceedings 105(5).\r\nLutsevych, Orysia. 2013. How to Finish a Revolution: Civil Society and Democracy in Georgia, Moldova, and\r\n         Ukraine. London.\r\nMachin, Stephen, Olivier Marie and Suncica Vujic. 2011. “The Crime Reducing Effect of Education.”\r\n         Economic Journal 121(552):463–484.\r\nMagaloni, Beatriz. 2010. “The Game of Electoral Fraud and the Ousting of Authoritarian Rule.” American\r\n         Journal of Political Science 54(3):751–765.\r\nMann, Michael. 1984. “The Autonomous Power of the State: Its Origins, Mechanisms and Results.” Archives\r\n         europe´ennes de sociologie 25(1984):185–213.\r\nUSAID.GOV                                                    GOVERNANCE IN ARMENIA: AN EVIDENCE REVIEW  |  54\r\n"                                                                                                                                                                                                                                                                                                                                                                                                                                                                                                                                                                                                                                                                                                                                                                                                                                                                                                                                                                                                                                                                                                                                                                                                                                                                                                                                                                                                                                                                                                                                                                                                                                                                                                                                                                                                                                                                                                                                                                                                                                                                                                                                                                                            
## [58] "Markarov, Alexander. 2016. Semi-Presidentialism in Armenia. In Semi-Presidentialism in the Caucasus and Central\r\n         Asia, ed. Robert Elgie and Sophia Moestrup. London: Palgrave Macmillan chapter 3.\r\nMauro, Paolo. 1995. “Corruption and Growth.” Quarterly Journal of Economics 110(3): 681–712.\r\nMcCargo, Duncan. 2015. “Transitional Justice and Its Discontents.” Journal of Democracy 26(2):5–20.\r\nMcMann, Kelly, Daniel Pemstein, Brigitte Seim, Jan Teorell and Staffan I. Lindberg. 2018. “A Measurement\r\n         Assessment Approach: Assessing The Varieties of Democracy Corruption Measures.” Working\r\n         Paper.\r\nMenaldo, Victor. 2016. The Institutions Curse: Natural Resources, Politics, and Development. Cambridge:\r\n      Cambridge University Press.\r\nMichalopoulos, Stelios, Alireza Naghavi, and Giovanni Prarolo. 2016. “Islam, Inequality and Pre-Industrial\r\n         Comparative Development.” Journal of Development Economics 120: 86–98.\r\nMiguel, Edward, and Mary Kay Gugerty. 2005. “Ethnic Diversity, Social Sanctions, and Public Goods in\r\n         Kenya.” Journal of Public Economics 89(11–12): 2325–68.\r\nMin, Brian. 2015. Power and the Vote: Elections and Electricity in the Developing World. New York: Cambridge\r\n         University Press.\r\nMitra, Saumya, Douglas Andrew, Gohar Gyulumyan, Paul Holden, Bart Kaminski, Yevgeny Kuznetsov and\r\n         Ekaterina Vashakmadze. 2007. The Caucasian Tiger: Sustaining Economic Growth in Armenia. Washington,\r\n         DC: World Bank.\r\nMonroe, Burt L., and Amanda G. Rose. 2002. “Electoral Systems and Unimagined Consequences:\r\n         Partisan Effects of Districted Proportional Representation.” American Journal of Political Science\r\n         46(1): 67–89.\r\nMookherjee, Dilip. 2015. “Political Decentralization.” Annual Review of Economics 7:231– 249.\r\nMungiu-Pippidi, Alina. 2015. The Quest for Good Governance: How Societies Develop Control of Corruption.\r\n         New York: Cambridge University Press.\r\nMungiu-Pippidi, Alina. 2016. “Learning from Virtuous Circles.” Journal of Democracy 27(1):95–109.\r\nMungiu-Pippidi, Alina, and Igor Munteanu. 2009. “Moldova’s Twitter Revolution.” Journal of Democracy\r\n         20(3): 136–42.\r\nMunteanu, Igor. 2018. “Moldova Should Reinvent Itself to Fight Corruption.” Emerging Europe: 1–5.\r\nMuralidharan, Karthik, Paul Niehaus and Sandip Sukhtankar. 2016. “Building State Capacity: Evidence\r\n         from Biometric Smartcards in India.” American Economic Review 106(10):2895–2929.\r\nUSAID.GOV                                                      GOVERNANCE IN ARMENIA: AN EVIDENCE REVIEW   |  55\r\n"                                                                                                                                                                                                                                                                                                                                                                                                                                                                                                                                                                                                                                                                                                                                                                                                                                                                                                                                                                                                                                                                                                                                                                                                                                                                                                                                                                                                                                                                                                                                                                                                                                                                                                                                                                                                                                                                                                                                                                                                                                                                                                                                                                                                                                                                                                                                                                                                                                                              
## [59] "Nemec, Juraj. 2014. “Comparative Analysis of Public Administration Reforms in Former Socialist\r\n        Countries of Central and Eastern Europe.” International Journal of Civil Service Reform and Practice\r\n        4(December 2014): 93–113.\r\nNemtsova, Anna. 2016. “The West Is About to Lose Moldova.” Foreign Policy.\r\nNooruddin, Irfan. 2009. “Good Governance.” APSA-CP Newsletter 20(2 (Summer 2009)):6.\r\nNorth, Douglass C. 1981. Structure and Change in Economic History. New York: W.W. Norton and\r\n      Company.\r\nNorth, Douglass C., John Joseph Wallis, and Barry R. Weingast. 2009. Violence and Social Orders: A\r\n      Conceptual Framework for Interpreting Recorded Human History. New York: Cambridge University\r\n      Press.\r\nOates, Wallace E. 2005. “Toward a Second-Generation Theory of Fiscal Federalism.” International Tax and\r\n        Public Finance 12(4):349–373.\r\nO’Donnell, Guillermo. 1998. “Horizontal Accountability in New Democracies.” Journal of Democracy\r\n        9(3): 1112–26.\r\nO’Donnell, Guillermo. 2004. “Why the Rule of Law Matters.” Journal of Democracy 15(4):32–46.\r\nO’Donnell, Guillermo and Philippe C. Schmitter. 1986. Transitions from Authoritarian Rule: Tentative\r\n        Conclusions about Uncertain Democracies. Baltimore, Maryland: Johns Hopkins University Press.\r\nOECD. 2018. Anti-Corruption Reforms in Armenia: Fourth Round of Monitoring of the Istanbul Anti-\r\n        Corruption Action Plan. Technical Report July Organization for Economic Cooperation and\r\n        Development Paris: .\r\nOlken, Benjamin A. 2007. “Monitoring Corruption: Evidence from a Field Experiment in Indonesia.”\r\n        Journal of Political Economy 115(21):200–249.\r\nOlken, Benjamin A.. 2010. “Direct Democracy and Local Public Goods: Evidence from a Field\r\n        Experiment in Indonesia.” American Political Science Review 104(2): 243–67.\r\nOstrom, Elinor. 1990. Governing the Commons: The Evolution of Institutions for Collective Action. Cambridge:\r\n        Cambridge University Press.\r\nPandey, Priyanka. 2010. “Service Delivery and Corruption in Public Services: How Does History\r\n        Matter?” American Economic Journal: Applied Economics 2(3): 190–204.\r\nPande, Rohini. 2011. “Can Informed Voters Enforce Better Governance? Experiments in Low-Income\r\n        Democracies.” Annual Review of Economics 3(1):215–237.\r\nPaturyan, Yevgenya J. and Christoph H. Stefes. 2017. Doing Business in Armenia: The Art of Manoeuvring a\r\n        System of Corruption. In State Capture, Political Risks and International Business: Cases from Black Sea\r\nUSAID.GOV                                                     GOVERNANCE IN ARMENIA: AN EVIDENCE REVIEW          | 56\r\n"                                                                                                                                                                                                                                                                                                                                                                                                                                                                                                                                                                                                                                                                                                                                                                                                                                                                                                                                                                                                                                                                                                                                                                                                                                                                                                                                                                                                                                                                                                                                                                                                                                                                                                                                                                                                                                                                                                                                                                                                                                                                                                                                                                                                                                                                                                                                                                                      
## [60] "         Region Countries, ed. Johannes Leitner and Hannes Meissner. New York: Routledge chapter 3, pp.\r\n         43–56.\r\nPeisakhin, Leonid, and Paul Pinto. 2010. “Is Transparency an Effective Anti-Corruption Strategy?\r\n         Evidence from a Field Experiment in India.” Regulation and Governance 4(3): 261–80.\r\nPersson, Anna, Bo Rothstein, and Jan Teorell. 2013. “Why Anticorruption Reforms Fail--Systemic\r\n         Corruption as a Collective Action Problem.” Governance 26(3): 449–71.\r\nPeters, B. Guy, ed. 2008. Mixes, Matches, and Mistakes: New Public Management in Russia and the Former\r\n       Soviet Republics. Budapest: Open Society Institute.\r\nPeters, B. Guy, and Jon Pierre. 2018. The Next Public Administration: Debates and Dilemmas. London: Sage\r\n       Publications.\r\nPfeil, Hélène, Berenike Laura Schott, and Sanjay Agarwal. 2016. Citizen Service Centers: Pathways Toward\r\n         Improved Public Service Delivery. World Bank: Washington, DC.\r\nPierskalla, Jan H. 2016. “The Politics of Urban Bias: Rural Threats and the Dual Dilemma of Political\r\n         Survival.” Studies in Comparative International Development 51(3): 286–307.\r\nPierskalla, Jan H., Anna Schultz, and Erik Wibbels. 2017. “Order, Distance, and Local Development over\r\n         the Long-Run.” Quarterly Journal of Political Science 12(4): 375–404.\r\nPindyck, Robert, and Daniel L. Rubinfeld. 2009. Microeconomics. Seventh. Upper Saddle River, New\r\n         Jersey: Prentice Hall.\r\nPritchett, Lant, and Michael Woolcock. 2004. “Solutions When the Solution Is the Problem: Arraying the\r\n       Disarray in Development.” World Development 32(2): 191–212.\r\nRaffler, Pia, Daniel N. Posner, and Doug Parkerson. 2018. The Weakness of Bottom Up Accountability:\r\n         Experimental Evidence from the Ugandan Health Sector. Working Paper.\r\nRauch, James E. and Peter Evans. 2000. “Bureaucratic Structure and Bureaucratic Performance in Less\r\n         Developed Countries.” Journal of Public Economics 75(1):49–71.\r\nRecanatini, Francesca. 2011. Anti-Corruption Authorities: An Effective Tool to Curb Corruption? In\r\n         Internal Handbook on the Economics of Corruption, Volume Two, ed. Susan Rose-Ackerman and Tina\r\n         Søreide. Northampton, Massachusetts: Edward Elgar Publishing Limited chapter 19, pp. 528–569.\r\nRemmer, Karen L. 2007. “The Political Economy of Patronage: Expenditure Patterns in the Argentine\r\n         Provinces, 1983-2003.” Journal of Politics 69(2):363–377.\r\nRobinson, James A., and Thierry Verdier. 2013. “The Political Economy of Clientelism.” Scandinavian\r\n         Journal of Economics 115(2): 260–91.\r\nRodden, Jonathan A. 2002. “The Dilemma of Fiscal Federalism: Grants and Fiscal Performance Around\r\n         the World.” American Journal of Political Science 46(3): 670-687.\r\nUSAID.GOV                                                     GOVERNANCE IN ARMENIA: AN EVIDENCE REVIEW | 57\r\n"                                                                                                                                                                                                                                                                                                                                                                                                                                                                                                                                                                                                                                                                                                                                                                                                                                                                                                                                                                                                                                                                                                                                                                                                                                                                                                                                                                                                                                                                                                                                                                                                                                                                                                                                                                                                                                                                                                                                                                                                                                                                                                                                          
## [61] "Rodden, Jonathan A. 2004. “Comparative Federalism and Decentralization: On Meaning and\r\n         Measurement.” Comparative Politics 36(4):481–500.\r\nRodrik, Dani. 2006. “Goodbye Washington Consensus, Hello Washington Confusion? A Review of the\r\n         World Bank’s Economic Growth in the 1990s: Learning from a Decade of Reform.” Journal of\r\n         Economic Literature XLIV(December): 973–87.\r\nRodrik, Dani. 2007. One Economics, Many Recipes: Globalization, Institutions, and Economic Growth.\r\n      Princeton, NJ: Princeton University Press.\r\nRoss, Michael L. 2006. “Is Democracy Good for the Poor?” American Journal of Political Science 50(4):860–\r\n         874.\r\nRoss, Michael L. 2015. “What Have We Learned about the Resource Curse?” Annual Review of Political Science\r\n         18(1):239–259.\r\nRothstein, Bo. 2011. The Quality of Government: Corruption, Social Trust, and Inequality in International\r\n         Perspective. Chicago: University of Chicago Press.\r\nSaari, Sinikukka. 2014. “Russia’s Post-Orange Revolution Strategies to Increase its Influence in Former\r\n         Soviet Republics: Public Diplomacy po russkii.” Europe-Asia Studies Vol. 66, No. 1, January 2014,\r\n         50–66.\r\nSachs, Jeffrey D. 2005. The End of Poverty: Economic Possibilities for Our Time. New York: Penguin Books.\r\nSamuels, David, and Matthew Shugart. 2010. Presidents, Parties and Prime Ministers: How Separation of\r\n         Powers Affects Party Organization and Behavior. New York: Cambridge University Press.\r\nSamuels, David, and Richard Snyder. 2001. “The Value of a Vote: Malapportionment in Comparative\r\n         Perspective.” British Journal of Political Science 31(04): 651–71.\r\nScartascini, Carlos, Pablo Spiller, Ernesto Stein and Mariano Tommasi. 2010. El Juego Poli´tico en Ame´rica\r\n         Latina ¿Co´mo Se Deciden Las Poli´ticas Públicas? Inter-American Development Bank: Washington,\r\n         DC.\r\nSchalkwyk, Andrew. 2010. Rejuvenating the Public Registry: Republic of Georgia, 2006-2008. Innovations for\r\n         Successful Society, Princeton University. Princeton, New Jersey.\r\nSchmitter, Philippe C. and Terry Lynn Karl. 1991. “What Democracy Is... and Is Not.” Journal of\r\n         Democracy 2(3):75–88.\r\nScott, James C. 1998. Seeing Like a State: How Certain Schemes to Improve the Human Condition Have Failed.\r\n      New Haven, Connecticut: Yale University Press.\r\nSelway, Joel Sawat. 2011. “The Measurement of Cross-Cutting Cleavages and Other Multidimensional\r\n         Cleavage Structures.” Political Analysis 19(1): 48–65.\r\nUSAID.GOV                                                     GOVERNANCE IN ARMENIA: AN EVIDENCE REVIEW   | 58\r\n"                                                                                                                                                                                                                                                                                                                                                                                                                                                                                                                                                                                                                                                                                                                                                                                                                                                                                                                                                                                                                                                                                                                                                                                                                                                                                                                                                                                                                                                                                                                                                                                                                                                                                                                                                                                                                                                                                                                                                                                                                                                                                                                                                                                                                                                                                                                                                                                              
## [62] "Shahnazarian, Nona. N.d. “Sashik-50 per cents and Samvel-two per cents: Anti-Corruption Trends in New\r\n          Armenia.” Forthcoming.\r\nShahnazarian, Nona and Matthew Light. 2018. “Parameters of Police Reform and Non- Reform in Post-\r\n          Soviet Regimes: The Case of Armenia.” Demokratizatsiya: The Journal of Post-Soviet Democratization\r\n          1(Winter):83–108.\r\nSo, Sokbuntheoen et al., eds. 2018. Alternative Paths to Public Financial Management and Public Sector\r\n          Reform Experiences from East Asia. Washington, DC: World Bank.\r\nSole´-Olle´, Albert and Pilar Sorribas-Navarro. 2008. “The Effects of Partisan Alignment on the Allocation of\r\n          Intergovernmental Transfers: Differences-in-Differences Estimates for Spain.” Journal of Public\r\n          Economics 92(12):2302–2319.\r\nSøreide, Tina. 2014. Drivers of Corruption: A Brief Review. Washington, DC: World Bank.\r\nStanig, Piero. 2015. “Regulation of Speech and Media Coverage of Corruption: An Empirical Analysis of the\r\n          Mexican Press.” American Journal of Political Science 59(1):175–193.\r\nStokes, Susan C., Thad Dunning, Marcelo Nazareno and Valeria Brusco. 2013. Brokers, Voters, and Clientelism:\r\n          The Puzzle of Distributive Politics. New York: Cambridge University Press.\r\nSugiyama, Natasha Borges. 2016. “Pathways to Citizen Accountability: Brazil’s Bolsa Família.” Journal of\r\n          Development Studies 52(8): 1192–1206.\r\nSugiyama, Natasha Borges, and Wendy Hunter. 2013. “Whither Clientelism? Good Governance and\r\n          Brazil’s Bolsa Família Program.” Comparative Politics 46(1): 43–62.\r\nSvolik, Milan W. 2012. The Politics of Authoritarian Rule. New York: Cambridge University Press.\r\nTeorell, Jan and Bo Rothstein. 2015. “Getting to Sweden, Part I: War and Malfeasance, 1720-1850.”\r\n          Scandinavian Political Studies 38(3):217–237.\r\nThe Economist. 2018. “The Economist’s Country of the Year 2018.”\r\nTheriault, Sean M. 2003. “Patronage, the Pendleton Act, and Power of the People.” Journal of\r\n          Politics 65(1): 485–86.\r\nTimmons, Jeffrey F. and Francisco Garfias. 2015. “Revealed Corruption, Taxation, and Fiscal\r\n          Accountability: Evidence from Brazil.” World Development 70:13–27.\r\nTocci, Nathalie. \"The Closed Armenia-Turkey Border: Economic and Social Effects, Including Those on\r\n          The People; And Implications for the Overall Situation In The Region.\" 2007. European\r\n          Parliament, Policy Department External Policies.\r\nToomet, Ott. \"Learn English, Not the Local Language! Ethnic Russians in the Baltic states.\" American\r\n          Economic Review101.3 (2011): 526-31.\r\nUSAID.GOV                                                       GOVERNANCE IN ARMENIA: AN EVIDENCE REVIEW |   59\r\n"                                                                                                                                                                                                                                                                                                                                                                                                                                                                                                                                                                                                                                                                                                                                                                                                                                                                                                                                                                                                                                                                                                                                                                                                                                                                                                                                                                                                                                                                                                                                                                                                                                                                                                                                                                                                                                                                                                                                                                                                                                                                                                                                                                                                                                                                                            
## [63] "Touchton, Michael, Natasha Borges Sugiyama, and Brian Wampler. 2017. “Democracy at Work: Moving\r\n         Beyond Elections to Improve Well-Being.” American Political Science Review 111(01): 68–82.\r\nTouchton, Michael, and Brian Wampler. 2013. “Improving Social Well-Being Through New Democratic\r\n         Institutions.” Comparative Political Studies 47(10): 1442–69.\r\nTransparency International. 2013. Integrity Pacts in Public Procurement: An Implementation Guide. Berlin:\r\n         Transparency International.\r\nTreisman, Daniel. 2007. The Architecture of Government: Rethinking Political Decentralization. Cambridge:\r\n         Cambridge University Press.\r\nUNESCO. 2018. “Education Statistics.” URL: http://uis.unesco.org/\r\nUSAID. 2019a. Civil Society and Media in Armenia: An Evidence Review for Learning, Evaluation, and\r\n         Research Activity II (LER II). URL:\r\n         https://dec.usaid.gov/dec/content/Detail.aspx?vID=47&ctID=ODVhZjk4NWQtM2YyMi00YjRmL\r\n         TkxNjktZTcxMjM2NDBmY2Uy&rID=NTE3Mzgw\r\nUSAID. 2019b. Integrity Systems and the Rule of Law in Armenia: An Evidence Review for Learning,\r\n         Evaluation and Research Activity II (LER II). URL:\r\n         https://dec.usaid.gov/dec/content/Detail.aspx?vID=47&ctID=ODVhZjk4NWQtM2YyMi00YjRmL\r\n         TkxNjktZTcxMjM2NDBmY2Uy&rID=NTE5NDk5\r\nUslaner, Eric M. 2017. The Historical Roots of Corruption: Mass Education, Economic Inequality, and State\r\n         Capacity. Cambridge: Cambridge University Press.\r\nvan der Ploeg, Frederick. 2011. “Natural Resources: Curse or Blessing?” Journal of Economic Literature\r\n         49(2):366–420.\r\nVeenendaal, Wouten. 2015. Politics and Democracy in Microstates. London: Routledge.\r\nVeiga, Linda Gonc¸alves and Francisco Jose´ Veiga. 2013. “Intergovernmental Fiscal Transfers as Pork Barrel.”\r\n         Public Choice 155(3-4):335–353.\r\nWalker, Shaun. 2016. “Mikheil Saakashvili Quits as Governor of Ukraine’s Odessa Region.” The Guardian.\r\nWay, Lucan 2015. “The Limits of Autocracy Promotion: The Case of Russia in the ‘Near Abroad’,\r\n         European Journal of Political Research 54: 691–706, 2015\r\nWeber, Max. 1978. Economy and Society: An Outline of Interpretive Sociology. Berkeley, California: University of\r\n         California Press.\r\nWeingast, Barry R. 2014. “Second Generation Fiscal Federalism: Political Aspects of De- centralization\r\n         and Economic Development.” World Development 53:14–25. URL:\r\n         http://dx.doi.org/10.1016/j.worlddev.2013.01.003\r\nUSAID.GOV                                                    GOVERNANCE IN ARMENIA: AN EVIDENCE REVIEW      |  60\r\n"                                                                                                                                                                                                                                                                                                                                                                                                                                                                                                                                                                                                                                                                                                                                                                                                                                                                                                                                                                                                                                                                                                                                                                                                                                                                                                                                                                                                                                                                                                                                                                                                                                                                                                                                                                                                                                                                                                                                                                                                                                                                                                                                                                                                                                                                                                                                                                                                                                                       
## [64] "Wicksberg, Sofia and Varuzhan Hoktanyan. 2013. Overview of Corruption and Anti- Corruption in\r\n         Armenia. Transparency International: Yerevan, Armenia.\r\nWilliamson, John. 1993. “Democracy and the Washington Consensus.” World Development 21(8): 1329–\r\n         36.\r\nWilliamson, John. 2000. “What Should the World Bank Think about the Washington Consensus?”\r\n         World Bank Research Observer 15(2): 251–64.\r\nWilliamson, Oliver. 1985. The Economic Institutions of Capitalism. New York: Simon and Schuster.\r\nWorld Bank. 2004. World Development Report: Making Services Work for Poor People. Washington,\r\n         DC: World Bank.\r\nWorld Bank. 2008. Public Sector Reform: What Works and Why. Washington, DC: World Bank\r\n      Independent Evaluation Group.\r\nWorld Bank. 2011. How-to Notes: Dealing with Governance and Corruption Risks in Project Lending.\r\n         Citizen Service Centers: Enhancing Access, Improving Service Delivery, and Reducing Corruption.\r\n         Washington, DC.\r\nWorld Bank. 2013. Global Review of Grievance Redress Mechanisms. Washington, DC.\r\nWorld Bank. 2015a. Project Appraisal Document on a Proposed Loan to the Republic of Armenia for a\r\n         Third Public Sector Modernization Project. Washington, DC.\r\nWorld Bank. 2015b. Independent Evaluation Group Implementation Completion Report Review for\r\n         Moldova’s Government Central Public Administration Reform (CPAR) Project. Washington,\r\n         DC.\r\nWorld Bank. 2017a. “World Development Indicators.” URL: http://data.worldbank.org/data-catalog/world-\r\n         development-indicators\r\nWorld Bank. 2017b. World Development Report: Governance and the Law. Washington, DC: World Bank.\r\nWorld Bank. 2017c. Independent Evaluation Group Implementation Completion Report Review of\r\n         Moldova’s ETRANSFORMATION Project (P121231). Washington, DC.\r\nWorld Bank. 2017d. Project Appraisal Document for Moldova’s Modernization of Government Services\r\n         Project (P148537). Washington, DC.\r\nWorld Bank. 2018. Implementation Status Report for Armenia’s Third Public Sector Modernization\r\n         Project. Washington, DC.\r\nXu, Guo. 2018. “The Costs of Patronage: Evidence from the British Empire.” American Economic\r\n         Review 108(11): 3170–98.\r\nZiblatt, Daniel. 2017. Conservative Parties and the Birth of Democracy. Cambridge: Cambridge University Press.\r\nUSAID.GOV                                                      GOVERNANCE IN ARMENIA: AN EVIDENCE REVIEW  |  61\r\n"                                                                                                                                                                                                                                                                                                                                                                                                                                                                                                                                                                                                                                                                                                                                                                                                                                                                                                                                                                                                                                                                                                                                                                                                                                                                                                                                                                                                                                                                                                                                                                                                                                                                                                                                                                                                                                                                                                                                                                                                                                                                                                                                                                                                                                                                                                                                                                                                                                                                                                                                                                                         
## [65] "Zimmerman, William. 2014. Ruling Russia: Authoritarianism from the Revolution to Putin. Princeton, New Jersey:\r\n        Princeton University Press.\r\nUSAID.GOV                                                  GOVERNANCE IN ARMENIA: AN EVIDENCE REVIEW      |  62\r\n"
\end{verbatim}

\hypertarget{question-12}{%
\subsection{Question 12}\label{question-12}}

convert the text from the pdf file to a dataframe

\begin{Shaded}
\begin{Highlighting}[]
\NormalTok{armeniatext=}\KeywordTok{as.data.frame}\NormalTok{(armeniatext, }\DataTypeTok{stringsAsFactors =} \OtherTok{FALSE}\NormalTok{)}
\end{Highlighting}
\end{Shaded}

\hypertarget{question-13}{%
\subsection{Question 13}\label{question-13}}

tokenize data by word and then remove the stop words

\begin{Shaded}
\begin{Highlighting}[]
\NormalTok{armeniatext=armeniatext }\OperatorTok
\StringTok{  }\KeywordTok{unnest_tokens}\NormalTok{(word, text)}
\end{Highlighting}
\end{Shaded}

\begin{verbatim}
## Error in check_input(x): Input must be a character vector of any length or a list of character
##   vectors, each of which has a length of 1.
\end{verbatim}

\begin{Shaded}
\begin{Highlighting}[]
\KeywordTok{data}\NormalTok{(stop_words)}

\NormalTok{armeniatext <-}\StringTok{ }\NormalTok{armenia }\OperatorTok\StringTok{ }
\StringTok{  }\KeywordTok{anti_join}\NormalTok{(stop_words)}
\end{Highlighting}
\end{Shaded}

\begin{verbatim}
## Error in eval(lhs, parent, parent): object 'armenia' not found
\end{verbatim}

\hypertarget{question-14}{%
\subsection{Question 14}\label{question-14}}

figuring out the op 5 most used word in the report

\begin{Shaded}
\begin{Highlighting}[]
\NormalTok{armtextfreq <-}\StringTok{ }\NormalTok{armeniatext }\OperatorTok
\KeywordTok{count}\NormalTok{(word, }\DataTypeTok{sort =} \OtherTok{TRUE}\NormalTok{)}
\end{Highlighting}
\end{Shaded}

\begin{verbatim}
## Error: Must group by variables found in `.data`.
## * Column `word` is not found.
\end{verbatim}

\begin{Shaded}
\begin{Highlighting}[]
\KeywordTok{head}\NormalTok{(armtextfreq)}
\end{Highlighting}
\end{Shaded}

\begin{verbatim}
## Error in h(simpleError(msg, call)): error in evaluating the argument 'x' in selecting a method for function 'head': object 'armtextfreq' not found
\end{verbatim}

\hypertarget{question-15}{%
\subsection{Question 15}\label{question-15}}

load the billboard hot 100 webpage and name the list

\begin{Shaded}
\begin{Highlighting}[]
\KeywordTok{library}\NormalTok{(rvest)}
\end{Highlighting}
\end{Shaded}

\begin{verbatim}
## Loading required package: xml2
\end{verbatim}

\begin{verbatim}
## 
## Attaching package: 'rvest'
\end{verbatim}

\begin{verbatim}
## The following object is masked from 'package:purrr':
## 
##     pluck
\end{verbatim}

\begin{verbatim}
## The following object is masked from 'package:readr':
## 
##     guess_encoding
\end{verbatim}

\begin{Shaded}
\begin{Highlighting}[]
\NormalTok{hot100page <-}\StringTok{ "https://www.billboard.com/charts/hot-100"}
\NormalTok{hot100exam <-}\StringTok{ }\KeywordTok{read_html}\NormalTok{(hot100page)}

\NormalTok{hot100exam}
\end{Highlighting}
\end{Shaded}

\begin{verbatim}
## {html_document}
## <html class="" lang="">
## [1] <head>\n<meta http-equiv="Content-Type" content="text/html; charset=UTF-8 ...
## [2] <body class="chart-page chart-page-" data-trackcategory="Charts-TheHot100 ...
\end{verbatim}

\hypertarget{question-16}{%
\subsection{Question 16}\label{question-16}}

using rvest to identify all the nodes in the webpage

\begin{Shaded}
\begin{Highlighting}[]
\NormalTok{body_nodes <-}\StringTok{ }\NormalTok{hot100exam }\OperatorTok
\KeywordTok{html_node}\NormalTok{(}\StringTok{"body"}\NormalTok{) }\OperatorTok
\KeywordTok{html_children}\NormalTok{()}
\NormalTok{body_nodes}
\end{Highlighting}
\end{Shaded}

\begin{verbatim}
## {xml_nodeset (36)}
##  [1] <div class="header-wrapper ">\n<header id="site-header" class="site-head ...
##  [2] <div class="site-header__placeholder"></div>
##  [3] <script>\n        var PGM = window.PGM || {};\n        PGM.config = PGM. ...
##  [4] <div class="chart-piano-overlay__attachment-point"></div>
##  [5] <main id="main" class="page-content"><div id="charts" data-page-title="T ...
##  [6] <div class="ad_desktop dfp-ad dfp-ad-promo " data-position="promo" data- ...
##  [7] <div class="ad-container footerboard footerboard--bottom">\n    <div cla ...
##  [8] <footer id="site-footer" class="site-footer"><div class="container foote ...
##  [9] <div class="biz-modal">\n    <div class="biz-modal__content">\n        < ...
## [10] <script>\n    window.CLARITY = window.CLARITY || [];\n</script>
## [11] <div class="ad_clarity" data-out-of-page="true" style="display: none;">< ...
## [12] <script>\n    var darkMatterCMD = function() {\n        this.darkMatterC ...
## [13] <script src="https://www.billboard.com/assets/1593527595/js/vendors_/art ...
## [14] <script src="https://www.billboard.com/assets/1593527595/js/vendors_/clo ...
## [15] <script src="https://www.billboard.com/assets/1593527595/js/vendors_/rea ...
## [16] <script src="https://www.billboard.com/assets/1593527595/js/vendors_/rea ...
## [17] <script src="https://www.billboard.com/assets/1593527595/js/vendors_/rea ...
## [18] <script src="https://www.billboard.com/assets/1593527595/js/vendors_/rea ...
## [19] <script src="https://www.billboard.com/assets/1593527595/js/default_/art ...
## [20] <script src="https://www.billboard.com/assets/1593527595/js/default_/rea ...
## ...
\end{verbatim}

\begin{Shaded}
\begin{Highlighting}[]
\NormalTok{body_nodes }\OperatorTok\StringTok{ }
\KeywordTok{html_children}\NormalTok{()}
\end{Highlighting}
\end{Shaded}

\begin{verbatim}
## {xml_nodeset (9)}
## [1] <header id="site-header" class="site-header " role="banner"><div class="s ...
## [2] <div class="header-wrapper__secondary-header">\n<nav class="site-header-l ...
## [3] <div id="charts" data-page-title="THE HOT 100" data-chart-code="HSI" data ...
## [4] <div class="footerboard-wrapper">\n        <div class="ad_desktop_placeho ...
## [5] <div class="container footer-content">\n\t\t\t\t\t<div class="cover-image ...
## [6] <div class="container">\n\t\t<p class="copyright__paragraph">© 2020 Billb ...
## [7] <div class="container">\n\t\t<p class="station-identification">\n\t\t\tBi ...
## [8] <div class="container">\n\t\t\n\n\n    <div class="ad_desktop dfp-ad dfp- ...
## [9] <div class="biz-modal__content">\n        <button class="biz-modal__close ...
\end{verbatim}

\hypertarget{question-17}{%
\subsection{Question 17}\label{question-17}}

using google chrome to identify the necessary tags and pull the data on
it

\begin{Shaded}
\begin{Highlighting}[]
\NormalTok{rank <-}\StringTok{ }\NormalTok{hot100exam }\OperatorTok
\NormalTok{rvest}\OperatorTok{::}\KeywordTok{html_nodes}\NormalTok{(}\StringTok{'body'}\NormalTok{) }\OperatorTok
\NormalTok{xml2}\OperatorTok{::}\KeywordTok{xml_find_all}\NormalTok{(}\StringTok{"//span[contains(@class,}
\StringTok{'chart-element__rank__number')]"}\NormalTok{) }\OperatorTok
\NormalTok{rvest}\OperatorTok{::}\KeywordTok{html_text}\NormalTok{()}

\NormalTok{artist <-}\StringTok{ }\NormalTok{hot100exam }\OperatorTok
\NormalTok{rvest}\OperatorTok{::}\KeywordTok{html_nodes}\NormalTok{(}\StringTok{'body'}\NormalTok{) }\OperatorTok
\NormalTok{xml2}\OperatorTok{::}\KeywordTok{xml_find_all}\NormalTok{(}\StringTok{"//span[contains(@class,}
\StringTok{'chart-element__information__artist')]"}\NormalTok{) }\OperatorTok
\NormalTok{rvest}\OperatorTok{::}\KeywordTok{html_text}\NormalTok{()}

\NormalTok{title <-}\StringTok{ }\NormalTok{hot100exam }\OperatorTok
\NormalTok{rvest}\OperatorTok{::}\KeywordTok{html_nodes}\NormalTok{(}\StringTok{'body'}\NormalTok{) }\OperatorTok
\NormalTok{xml2}\OperatorTok{::}\KeywordTok{xml_find_all}\NormalTok{(}\StringTok{"//span[contains(@class,}
\StringTok{'chart-element__information__song')]"}\NormalTok{) }\OperatorTok
\NormalTok{rvest}\OperatorTok{::}\KeywordTok{html_text}\NormalTok{()}

\NormalTok{lastweek <-}\StringTok{ }\NormalTok{hot100exam }\OperatorTok\StringTok{ }
\StringTok{  }\NormalTok{rvest}\OperatorTok{::}\KeywordTok{html_nodes}\NormalTok{(}\StringTok{'body'}\NormalTok{) }\OperatorTok\StringTok{ }
\StringTok{  }\NormalTok{xml2}\OperatorTok{::}\KeywordTok{xml_find_all}\NormalTok{(}\StringTok{"//span[contains(@class, 'chart-element__information__delta__text text--last')]"}\NormalTok{) }\OperatorTok\StringTok{ }
\NormalTok{rvest}\OperatorTok{::}\KeywordTok{html_text}\NormalTok{()}
\end{Highlighting}
\end{Shaded}

\hypertarget{last-question}{%
\subsection{Last Question}\label{last-question}}

\href{https://github.com/RolongAlong/exam3.git}{Link to GitHub Repo}

\end{document}
